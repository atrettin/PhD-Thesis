\tikzsetnextfilename{sterile_metric_scan}%
\begin{tikzpicture}

% Define new pgfmath function for the logarithm to base 10 that also works with fpu library
\pgfmathdeclarefunction{lg10}{1}{%
    \pgfmathparse{ln(#1)/ln(10)}%
}

\tikzstyle{bfp}=[very thick, orange, shape=diamond, fill=white, fill opacity=0.8, scale=0.8]


\begin{loglogaxis}[
	%view={0}{90},
	%enlargelimits,
    width=6.5cm,
    height=6.5cm,
    %height=0.9\linewidth,
    colorbar,
    colorbar style={
        %ytick={-2, -1, 0, 1, 2},
        yticklabel={$10^{\pgfmathprintnumber{\tick}}$},
        ylabel=$\Delta \chi^2_{\mathrm{mod}}$,
        tick align=outside,
        tick pos=right,
        tick style={color=black}
    },
    colorbar/width=0.4 cm,
    colormap/viridis,
    point meta min=-2.2,
    point meta max=2.2,
    clip marker paths=true,
    legend columns=2,
    xmin=8e-4, xmax=0.2,
    ymin=8e-4, ymax=0.5,
    ylabel={$|U_{\tau 4}|^2 = \sin^2\theta_{34}\cos^2\theta_{24}$},
    xlabel={$|U_{\mu 4}|^2 = \sin^2\theta_{24}$},
    xmajorgrids, ymajorgrids,
    tick align=outside,
    tick pos=left,
    tick style={color=black}
]
    \addplot [
        scatter,
        only marks,
        scatter src=explicit,
        mark size=3,
        forget plot
    ] table [
    	col sep=comma,
        x=Umu4_sq,
        y=Utau4_sq,
        %meta=metric_diff
        meta expr=lg10(\thisrow{metric_diff})
    ] {figures/measurement/sterile_analysis/results/contour_data/real_data_v12_metric_diff.csv};
	
    \addplot [black, very thick] table [col sep=comma,x=Umu4_sq,y=Utau4_sq] {figures/measurement/sterile_analysis/results/contour_data/real_data_v12_wilks_contour_90pct.csv};
    \addlegendentry{90\% C.L.}
    \addplot [black, very thick, dashed] table [col sep=comma,x=Umu4_sq,y=Utau4_sq] {figures/measurement/sterile_analysis/results/contour_data/real_data_v12_wilks_contour_99pct.csv};
    \addlegendentry{99\% C.L.}

    \addplot [scatter, only marks, mark=*, black] coordinates {
        (0.063, 0.001)
        (0.023, 0.059)
        (0.001, 0.066)
    };
    \addlegendentry{test points}

\node[draw, bfp] at (0.004578826816037697, 0.0029866619869177565) {};
\addlegendimage{legend image code/.code={
	\node[draw, bfp, scale=0.8] at (0cm,0cm) {};
    }
}
\addlegendentry{best fit point}

% One could also use TeX to plot the contours directly from the likelihood as below, but that 
% requires LuaLaTeX.

% level68: 2.27886856637673, lg10 = 0.357719278
% level90: 4.605170185988092, lg10 = 0.6632456844
% level99: 9.21034037197618, lg10 = 0.96427568
% level5sig: 28.743702426548186, lg10 = 1.4585427081

%	\addplot3 [
%        contour lua={
%        	%levels={4.605170185988092, 9.21034037197618, 28.743702426548186}
%			levels={0.6632456844, 0.96427568, 1.4585427081}
%		},
%		thick,
%        mesh/ordering=y varies,
%        mesh/rows=20,
%        mesh/cols=21,
%        %point meta=42
%    ] table [
%    	col sep=comma,
%		x=Umu4_sq,
%		y=Utau4_sq,
%		z expr=lg10(\thisrow{metric_diff})
%	] {contours/real_data_v12_metric_diff.csv};
    
\end{loglogaxis}
\end{tikzpicture}
