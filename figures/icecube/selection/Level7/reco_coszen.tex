\tikzsetnextfilename{reco_coszen_l7}%
\begin{tikzpicture}
    \pgfplotstableread{figures/icecube/selection/Level7/reco_coszen.csv}\table
    \begin{groupplot}[
        xmin=-1.1,xmax=1.1,
        xmode=normal,
        xmajorgrids, ymajorgrids,
        width=0.45\linewidth,
        ylabel style={at={(-0.15,0.5)}},
        group/.cd,
        group size=1 by 2,
        xticklabels at=edge bottom,
        vertical sep=10pt
        ]
    \nextgroupplot[
        height=0.3\linewidth,
        legend cell align={left},
        legend columns=-1,
        legend to name=reco_coszen_legend,
        ymode=log,
        ymin=2e-7, ymax=5e-4,
        ylabel=rate (Hz),
        % add magic filter to correctly handle empty bins in logarithmic y-axes:
        % If a bin-count is too low or zero, it would cause the line to be
        % interrupted, which creates artefacts and ugliness. Instead, we replace
        % these bin-counts with values that are just below the axis limit.
        % Because of the way pgfplots works, the input is the raw number but the
        % output has to be the log. Weird, I know.
        y filter/.expression={y < 8.7587858920556e-09 ? ln(8.758785892055601e-10) : ln(y)}
    ]

    \ploterrorband[muon_color]{muon}{1}
    \addlegendentry{atm. muons}

    \ploterrorband[nue_color]{nuenuebar}{1}
    \addlegendentry{$\nu_e + \bar{\nu}_e$}

    \ploterrorband[numu_color]{numunumubar}{1}
    \addlegendentry{$\nu_\mu + \bar{\nu}_\mu$}

    \ploterrorband[nutau_color]{nutaunutaubar}{1}
    \addlegendentry{$\nu_\tau + \bar{\nu}_\tau$}


    % alternative event breakdown by interaction
    % \ploterrorband[nue_color]{nue_ccnuebar_cc}{1}
    % \addlegendentry{$\nu_e + \bar{\nu}_e$, CC}
    %
    % \ploterrorband[numu_color]{numu_ccnumubar_cc}{1}
    % \addlegendentry{$\nu_\mu + \bar{\nu}_\mu$, CC}
    %
    % \ploterrorband[nutau_color]{nutau_ccnutaubar_cc}{1}
    % \addlegendentry{$\nu_\tau + \bar{\nu}_\tau$, CC}
    %
    % \ploterrorband[nc_color]{nuall_ncnuallbar_nc}{1}
    % \addlegendentry{all $\nu$, NC}

    \ploterrorband{total_mc}{1}
    \addlegendentry{total MC}

    \ploterrorbar{data}
    \addlegendentry{data}


    % draw cut
    % use "axis cs" to give coordinates in the data coordinate system!
    \draw[thick,dashed] (axis cs:0.1,1e-9) -- (axis cs:0.1,5e-4);
    \draw[-stealth, very thick] (axis cs:0.1,1e-4)  -- node[anchor=south]{\footnotesize\sffamily signal} (axis cs:-0.4,1e-4);

    \nextgroupplot[
        height=0.2\linewidth,
        ymin=-0.1, ymax=1.3,
        ylabel=rate/total MC,
        xlabel=reconstructed $\cos(\theta_{\mathrm{zenith}})$\strut
    ]

    \ploterrorbar{data_mc_ratio}
    \plotratioerrorband[muon_color]{muon}{total_mc}
    \plotratioerrorband[nue_color]{nuenuebar}{total_mc}
    \plotratioerrorband[numu_color]{numunumubar}{total_mc}
    \plotratioerrorband[nutau_color]{nutaunutaubar}{total_mc}
    \end{groupplot}
\end{tikzpicture}
