\begin{tikzpicture}

\pgfplotsset{set layers}

% Declare a Gaussian function that takes three arguments: (x, mu, sigma)
\pgfmathdeclarefunction{gauss}{3}{%
    \pgfmathparse{1/(#3*sqrt(2*pi))*exp(-((#1-#2)^2)/(2*#3^2))}%
}

\begin{axis}[
    scale only axis,
    width=0.6\linewidth,
    height=0.4\linewidth,
    xmin=-3.5, xmax=3.5,
    ymin=-0.05, ymax=0.5,
    xmajorgrids,
    axis y line*=left,
    ylabel=probability density,
    xlabel=$E_\mathrm{reco} - E_\mathrm{true}$ (a.u.),
    legend columns=-1
]

    \addplot[
        domain=-4:4,
        samples=100,
        skyblue,
        very thick
    ] (x, {gauss(x, 0, 1.2)});
    \label{plot:nominal-domeff}
    \addlegendentry{nominal}

    \addplot[
        domain=-4:4,
        samples=100,
        orange,
        very thick
    ] (x, {gauss(x, 0.6, 1)});
    \label{plot:increased-domeff}
    \addlegendentry{+DOM eff.}
    
    \addplot[black, thick, draw=none, mark=none] coordinates {(0,0) (1,1)};
    \label{plot:density-ratio}
    \addlegendentry{ratio}

\end{axis}

\begin{axis}[
    scale only axis,
    width=0.6\linewidth,
    height=0.4\linewidth,
    xmin=-3.5, xmax=3.5,
    ymin=-0.28, ymax=2.9,
    axis y line*=right,
    ylabel=density ratio,
    ymajorgrids
]

    \addplot[
       domain=-4:4,
       samples=100,
       black,
       thick
    ] (x, {gauss(x, 0.6, 1) / gauss(x, 0, 1.2)});

\end{axis}
\end{tikzpicture}
