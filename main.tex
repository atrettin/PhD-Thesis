\documentclass[
a4paper, % Page size
fontsize=10pt, % Base font size
twoside=true, % Use different layouts for even and odd pages (in particular, if twoside=true, the margin column will be always on the outside)
%open=any, % If twoside=true, uncomment this to force new chapters to start on any page, not only on right (odd) pages
%chapterentrydots=true, % Uncomment to output dots from the chapter name to the page number in the table of contents
numbers=noenddot, % Comment to output dots after chapter numbers; the most common values for this option are: enddot, noenddot and auto (see the KOMAScript documentation for an in-depth explanation)
]{kaobook}

%----------------------------------------------------------------------------------------
%	PACKAGES AND OTHER DOCUMENT CONFIGURATIONS
%----------------------------------------------------------------------------------------

% Choose the language
\ifxetexorluatex
\usepackage{polyglossia}
\setmainlanguage{english}
\else
\usepackage[english]{babel} % Load characters and hyphenation
\fi
\usepackage[english=british]{csquotes}	% English quotes

% Load packages for testing
\usepackage{blindtext}
\usepackage{adjustbox}

\usepackage{pifont}% http://ctan.org/pkg/pifont
\newcommand{\cmark}{\ding{51}}%
\newcommand{\xmark}{\ding{55}}%
\usepackage{longtable}
\usepackage{booktabs}
\usepackage{tabularx}
\usepackage{siunitx}

% Plots and graphs
\usepackage{pgfplots}
\usepackage{pgfplotstable}
\DeclareUnicodeCharacter{2212}{−}
\usepgfplotslibrary{groupplots,dateplot}
\usetikzlibrary{patterns,shapes,arrows}
\pgfplotsset{compat=newest}
\usepgfplotslibrary{fillbetween}

% global color definitions
\definecolor{black}{RGB}{0,0,0}
\definecolor{orange}{RGB}{230, 159, 0}
\definecolor{skyblue}{RGB}{86, 180, 233}
\definecolor{bluishgreen}{RGB}{0, 158, 115}
\definecolor{yellow}{RGB}{240, 228, 66}
\definecolor{blue}{RGB}{0, 114, 178}
\definecolor{vermilion}{RGB}{213, 94, 0}
\definecolor{reddishpurple}{RGB}{204, 121, 167}

\definecolor{lightgray204}{RGB}{204,204,204}

\tikzstyle{nue_color}=[orange]
\tikzstyle{numu_color}=[bluishgreen]
\tikzstyle{nutau_color}=[skyblue]
% color for any "all NC" plots
\tikzstyle{nc_color}=[reddishpurple]
\tikzstyle{noise_color}=[blue]
\tikzstyle{muon_color}=[vermilion]

% plot an error band assuming that a table has been loaded.
% Syntax: \ploterrorband[style]{column_name}{scale_factor}
% The column name and the column "column_name__err" must exist. The
% scale factor scales the entire error band at once.
\newcommand{\ploterrorband}[3][black]{
	\addplot[const plot, #1, thick] table [x=bin_edges, y expr=\thisrow{#2} * #3] from \table;
    \addplot[const plot, #1, thin, name path=err_lo, forget plot] table[x=bin_edges, y expr=(\thisrow{#2}-\thisrow{#2__err}) * #3] from \table;
    \addplot[const plot, #1, thin, name path=err_hi, forget plot] table[x=bin_edges, y expr=(\thisrow{#2}+\thisrow{#2__err}) * #3] from \table;
    \addplot[#1, opacity=0.5, forget plot] fill between[of = err_lo and err_hi];
}

% Plot the ratio beteween two histograms with correct error propagation.
% Syntax: \plotratioerrorband[style]{nominator column}{denominator column}
% If the two columns are "x" and "y", then the ratio is x/y and the error
% on the ratio is:
%   sigma = sqrt((x * y__err)^2 + (y * x__err)^2) / y^2
% \newcommand{\plotratioerrorband}[3][black]{
% 	\addplot[const plot, #1, thick] table [x=bin_edges, y expr=\thisrow{#2} / \thisrow{#3}] from \table;

%     \addplot[const plot, #1, thin, name path=err_lo, forget plot] table[x=bin_edges,
%     y expr=
%     	\thisrow{#2} / \thisrow{#3}
% 		-
% 		sqrt(
%             (\thisrow{#2} * \thisrow{#3__err})^2 + (\thisrow{#3} * \thisrow{#2__err})^2
% 		) / \thisrow{#3}^2
% 	] from \table;

% 	\addplot[const plot, #1, thin, name path=err_hi, forget plot] table[x=bin_edges,
%     y expr=
%     	\thisrow{#2} / \thisrow{#3}
% 		+
% 		sqrt(
%             (\thisrow{#2} * \thisrow{#3__err})^2 + (\thisrow{#3} * \thisrow{#2__err})^2
% 		) / \thisrow{#3}^2
% 	] from \table;

%     \addplot[#1, opacity=0.5, forget plot] fill between[of = err_lo and err_hi];
% }

% Plot the ratio error band between two histograms, assuming that the denominator is the sum
% of the nominator and some background. This applies to any ratio where the denominator is the total MC
% and the nominator is some component. The difference w.r.t. \plotratioerrorband is that the two
% histograms are *dependent*. This reduces the error in particular for cases where the component
% for which the ratio is plotted also dominantly makes up the total.
% The error calculation is derived by expressing the ratio x / y, where x and y are dependent, as
%     f = x / (x + b),
% where now x and b are independent. The variance of b is
%     sigma_b^2 = sigma_y^2 - sigma_x^2.
% Applying the error propagation and then replacing b = y - x back into the result we find
%    sigma_f = sqrt( (y * sigma_x)^2 + (x * sigma_y)^2 - 2*x*y*sigma_y^2) / y^2
\newcommand{\plotratioerrorband}[3][black]{
	\addplot[const plot, #1, thick] table [x=bin_edges, y expr=\thisrow{#2} / \thisrow{#3}] from \table;

    \addplot[const plot, #1, thin, name path=err_lo, forget plot] table[x=bin_edges,
    y expr=
    	\thisrow{#2} / \thisrow{#3}
		-
		sqrt(
            (\thisrow{#2} * \thisrow{#3__err})^2 + (\thisrow{#3} * \thisrow{#2__err})^2 - (2 * \thisrow{#2} * \thisrow{#3} * \thisrow{#2__err}^2)
		) / \thisrow{#3}^2
	] from \table;

	\addplot[const plot, #1, thin, name path=err_hi, forget plot] table[x=bin_edges,
    y expr=
    	\thisrow{#2} / \thisrow{#3}
		+
		sqrt(
            (\thisrow{#2} * \thisrow{#3__err})^2 + (\thisrow{#3} * \thisrow{#2__err})^2 - (2 * \thisrow{#2} * \thisrow{#3} * \thisrow{#2__err}^2)
		) / \thisrow{#3}^2
	] from \table;

    \addplot[#1, opacity=0.5, forget plot] fill between[of = err_lo and err_hi];
}


\pgfplotsset{error bar legend/.style={%
    /pgfplots/legend image code/.prefix code={%
      \pgfkeysgetvalue{/pgfplots/error bars/error mark}{\pgfplotserrorbarsmark}%
      \draw[%
        /pgfplots/every error bar,
        mark=\pgfplotserrorbarsmark,
        /pgfplots/error bars/error mark options,
        sharp plot,
        ##1
      ] plot coordinates {(0.3cm, -0.1cm) (0.3cm, 0.1cm)};%
    }
  }
}

\newcommand{\ploterrorbar}[2][black]{
    \addplot[
        mark=*,
        mark options={scale=0.5, fill=black},
        #1,
        only marks,
        error bar legend,
        error bars/.cd,
        x dir=none,
        y dir=both,
        y explicit
    ] table [x=bin_midpoints, y=#2, y error=#2__err]  from \table;
}

\usepackage{helvet}
\usepackage[eulergreek]{sansmath}
\pgfplotsset{
    tick label style = {font=\footnotesize\sansmath\sffamily},
    every axis label = {font=\footnotesize\sansmath\sffamily},
    % the eulergreek style for math looks oddly different from the
    % rest of the document, so we omit \sansmath here
    label style = {font=\footnotesize\sffamily},
    % global legend style applied to all plots (set per plot to override)
    legend style = {
        font=\footnotesize\sffamily,
        fill opacity=0.8,
        draw opacity=1,
        text opacity=1,
        at={(0.5,0.95)},
        anchor=north,
        draw=lightgray204,
        % hack to get better legend spacing, see
        % https://tex.stackexchange.com/questions/18152/how-can-i-adjust-the-horizontal-spacing-between-legend-entries-in-pgfplots
        /tikz/every even column/.append style={column sep=0.1cm}
    },
    legend cell align={left}
}

\tikzstyle{plot annotation}=[
    font=\footnotesize\sffamily,
    fill opacity=0.8,
    draw opacity=1,
    text opacity=1,
    draw=lightgray204,
    fill=white
]


% Load the bibliography package
\usepackage{aas_macros}
\usepackage[giveninits=true]{kaobiblio}
\addbibresource{refs.bib} % Bibliography file

% Colors
\usepackage[dvipsnames]{xcolor}

% Load mathematical packages for theorems and related environments
\usepackage[framed=true]{kaotheorems}
\usepackage{amsmath}
\usepackage{nicefrac}
\renewcommand{\rm}{\mathrm}
% Load the package for hyperreferences
\usepackage{kaorefs}

\usepackage{caption}
\usepackage{subcaption}

\graphicspath{{images/}{./}{figures/}{tikz/}} % Paths in which to look for images

\makeindex[columns=3, title=Alphabetical Index, intoc] % Make LaTeX produce the files required to compile the index

%----------------------------------------------------------------------------------------

\newcommand{\numucc}{$\nu_{\mu,\,\rm CC}$}

% command to simplify writing total differentials
\newcommand{\drm}{\mathrm{d}}

% Reset sidenote counter at chapters
%\counterwithin*{sidenote}{chapter}

%%%%********************************************************************
% fancy quotes
\definecolor{quotemark}{gray}{0.7}
\makeatletter
\def\fquote{%
	\@ifnextchar[{\fquote@i}{\fquote@i[]}%]
}%
\def\fquote@i[#1]{%
	\def\tempa{#1}%
	\@ifnextchar[{\fquote@ii}{\fquote@ii[]}%]
}%
\def\fquote@ii[#1]{%
	\def\tempb{#1}%
	\@ifnextchar[{\fquote@iii}{\fquote@iii[]}%]
}%
\def\fquote@iii[#1]{%
	\def\tempc{#1}%
	\vspace{1em}%
	\noindent%
	\begin{list}{}{%
			\setlength{\leftmargin}{0.1\textwidth}%
			\setlength{\rightmargin}{0.1\textwidth}%
		}%
		\item[]%
		\begin{picture}(0,0)%
		\put(-15,-5){\makebox(0,0){\scalebox{3}{\textcolor{quotemark}{``}}}}%
		\end{picture}%
		\begingroup\itshape}%
	%%%%********************************************************************
	\def\endfquote{%
		\endgroup\par%
		\makebox[0pt][l]{%
			\hspace{0.8\textwidth}%
			\begin{picture}(0,0)(0,0)%
			\put(15,15){\makebox(0,0){%
					\scalebox{3}{\color{quotemark}''}}}%
			\end{picture}}%
		\ifx\tempa\empty%
		\else%
		\ifx\tempc\empty%
		\hfill\rule{100pt}{0.5pt}\\\mbox{}\hfill\tempa,\ \emph{\tempb}%
		\else%
		\hfill\rule{100pt}{0.5pt}\\\mbox{}\hfill\tempa,\ \emph{\tempb},\ \tempc%
		\fi\fi\par%
		\vspace{0.5em}%
	\end{list}%
}%
\makeatother
%----------------------------------------------------------------------------------------
\renewcommand{\thefootnote}{\roman{footnote}}



% For some reason this *must* be placed here, right before the start of the
% document!
% externalization to avoid repeated compilation of pgfplots
\usepgfplotslibrary{external}
% Use a sub-directory for these intermediate files. It will look empty in the
% web-interface, but if you compile locally, they will be there.
\tikzexternalize[prefix=tikz/]
% very important: ONLY do this for figures for which we have given a name!
% otherwise, there will be a bunch of tiny unnamed output PDFs and errors.
\tikzset{external/only named=true}

%%%% UN-COMMENT TO FORCE REMAKE ALL TEX FIGURES %%%%
%\tikzset{external/force remake}

\begin{document}

%----------------------------------------------------------------------------------------
\frontmatter % Denotes the start of the pre-document content, uses roman numerals

%----------------------------------------------------------------------------------------
%	BOOK INFORMATION
%----------------------------------------------------------------------------------------
\KOMAoptions{twoside=false}
\begin{titlepage}
	\begin{center}
	\vspace*{1cm}

	\LARGE
	\textbf{Search for eV-scale sterile neutrinos with IceCube DeepCore}
	\large

	\vspace{0.8cm}

	\textbf{Dissertation}\\
	zur Erlangung des akademischen Grades\\
	doctor rerum naturalium \\
	(Dr. rer. nat.) \\

	\vspace{0.5cm}

	im Fach: Physik \\
	Spezialisierung: Experimentalphysik\\

	\vspace{0.5cm}

	eingereicht an der \\
	Mathematisch-Naturwissenschaftlichen Fakultät\\
	der Humboldt-Universität zu Berlin\\

	\vspace{0.5cm}

	von\\
	\textbf{Alexander Trettin M. Sc}\\
%	\vspace{0.8cm}
	geboren am 07. Mai 1993\\
	in Kiel

	\vspace{0.5cm}

	Präsidentin der Humboldt-Universität zu Berlin\\
	Prof. Dr.-Ing. Dr. Sabine Kunst\\

	\vspace{0.5cm}

	Dekan der Mathematisch-Naturwissenschaftlichen Fakultät\\
	Prof. Dr. Elmar Kulke\\

	\end{center}
	\newpage

	\vspace*{8cm}

	\textbf{No copyright}\\
	\cczero\ This book is released into the public domain using the CC0 code. To the extent possible under law, I waive all copyright and related or neighbouring rights to this work.

	To view a copy of the CC0 code, visit: \\\url{http://creativecommons.org/publicdomain/zero/1.0/}

	\medskip

	\textbf{Colophon} \\
	This document was typeset with the help of \href{https://sourceforge.net/projects/koma-script/}{\KOMAScript} and \href{https://www.latex-project.org/}{\LaTeX} using the open-source \href{https://github.com/fmarotta/kaobook/}{kaobook} template class.\\

	% TODO: upload to GitHub, sync
	\todo[inline, noinlinepar]{Link to code }
	% The source code of this thesis is available % at:\\\url{https://github.com/robertdstein/kaobook}, \\while the scripts used to generate the % plots are available at: \\\url{https://github.com/robertdstein/thesis_code}\\

	\medskip

	\textbf{Publisher} \\
	\todo[inline, noinlinepar]{add when first printed, by whom}
	%First printed in Nov 2022 by Humboldt Universität zu Berlin

	% \newpage
	%
	% \vspace*{5.0cm}
	%
	% \large
	% A neutrino is not a big thing to be hit by. \\
	% In fact it's hard to think of anything much smaller by which one could reasonably hope to be % hit. And it's not as if being hit by neutrinos was in itself a particularly unusual event for % something the size of the Earth. Far from it. It would be an unusual nanosecond in which the % Earth was not hit by several billion passing neutrinos.\\
	% \flushright --\textit{The Hitchhiker's Guide to The Galaxy}
	%
	% \afterpage{\blankpage}

\end{titlepage}
\title[Search for eV-scale sterile neutrinos with IceCube DeepCore]



%----------------------------------------------------------------------------------------
%	COPYRIGHT PAGE
%----------------------------------------------------------------------------------------
% \makeatletter
% \uppertitleback{\@titlehead} % Header
% \lowertitleback{

% 	\textbf{No copyright}\\
% 	\cczero\ This book is released into the public domain using the CC0 code. To the extent possible under law, I waive all copyright and related or neighbouring rights to this work.

% 	To view a copy of the CC0 code, visit: \\\url{http://creativecommons.org/publicdomain/zero/1.0/}

% 	\medskip

% 	\textbf{Colophon} \\
% 	This document was typeset with the help of \href{https://sourceforge.net/projects/koma-script/}{\KOMAScript} and \href{https://www.latex-project.org/}{\LaTeX} using the open-source \href{https://github.com/fmarotta/kaobook/}{kaobook} template class.\\

% 	The source code of this thesis is available at:\\\url{https://github.com/robertdstein/kaobook}, \\while the scripts used to generate the plots is available at: \\\url{https://github.com/robertdstein/thesis_code}\\

% 	\medskip

% 	\textbf{Publisher} \\
% 	First printed in Nov 2021 by Humboldt Universität zu Berlin
% }
% \makeatother
%----------------------------------------------------------------------------------------
%	DEDICATION
%----------------------------------------------------------------------------------------
% \dedication{
% 	A neutrino is not a big thing to be hit by. \\

% 	In fact it's hard to think of anything much smaller by which one could reasonably hope to be hit. And it's not as if being hit by neutrinos was in itself a particularly unusual event for something the size of the Earth. Far from it. It would be an unusual nanosecond in which the Earth was not hit by several billion passing neutrinos.\\
% 	\flushright --\textit{The Hitchhiker's Guide to The Galaxy}
% }
%----------------------------------------------------------------------------------------
%	OUTPUT TITLE PAGE AND PREVIOUS
%----------------------------------------------------------------------------------------
% Note that \maketitle outputs the pages before here
% If twoside=false, \uppertitleback and \lowertitleback are not printed
% To overcome this issue, we set twoside=semi just before printing the title pages, and set it back to false just after the title pages
\KOMAoptions{twoside=semi}
\KOMAoptions{twoside=false}
%----------------------------------------------------------------------------------------
%	PREFACE
%----------------------------------------------------------------------------------------
%\input{chapters/preface.tex}
%\index{preface}
\chapter{Abstract}

Here be the abstract.\todo{write abstract}

\cleardoubleoddpage
\chapter{Zusammenfassung}

Eine Zusammenfassung sollte ebenfalls auf Deutsch geschrieben werden.
Wenn du jemals dachtest, du seist nutzlos, so denke daran, dass es diese Zusammenfassungen gibt.
\cleardoubleoddpage
%----------------------------------------------------------------------------------------
%	TABLE OF CONTENTS & LIST OF FIGURES/TABLES
%----------------------------------------------------------------------------------------
\begingroup % Local scope for the following commands
% Define the style for the TOC, LOF, and LOT
%\setstretch{1} % Uncomment to modify line spacing in the ToC
%\hypersetup{linkcolor=olive} % Uncomment to set the colour of links in the ToC
\setlength{\textheight}{230\hscale} % Manually adjust the height of the ToC pages
% Turn on compatibility mode for the etoc package
\etocstandarddisplaystyle % "toc display" as if etoc was not loaded
\etocstandardlines % "toc lines as if etoc was not loaded
\tableofcontents % Output the table of contents
\listoffigures % Output the list of figures
% Comment both of the following lines to have the LOF and the LOT on different pages
%\let\cleardoublepage\bigskip
%\let\clearpage\bigskip
\listoftables % Output the list of tables

\listoftodos

\endgroup
%----------------------------------------------------------------------------------------
%	MAIN BODY
%----------------------------------------------------------------------------------------
\mainmatter % Denotes the start of the main document content, resets page numbering and uses arabic numbers
\setchapterstyle{kao} % Choose the default chapter heading style

\setchapterstyle{kao}
\setchapterpreamble[u]{\margintoc}
\chapter{Introduction}
\labch{intro}
% \begin{fquote}[Douglas Adams][The Restaurant at the End of the Universe][1980] In the beginning the Universe was created. This has made a lot of people very angry and been widely regarded as a bad move. 
% \end{fquote}

The introduction is a text without further sub-dividing chapters. It gives an overview over the topic and links to the following chapters. 

\setchapterpreamble[u]{\margintoc}
\setchapterstyle{kao}
% \setchapterimage[6.5cm]{figures/artwork/Alexander_Trettin_illustrations_of_wave_functions_and_particle__21cc39d2-0bda-4230-9b29-9bfd8707df01.png}
\chapter{Neutrinos in the Standard Model}

\labch{stdmodel}

The Standard Model (SM) of particle physics is a relativistic quantum field theory based on the gauge symmetry group $\mathrm{SU}(3)_C \times \mathrm{SU}(2)_L \times \mathrm{U}(1)_Y$, where the sub-scripts $C$, $L$ and $Y$ correspond to the conserved quantities \emph{color}, \emph{left-handed chirality} and \emph{weak hypercharge}, respectively. In this model, all matter particles are described as fermions, that is, excitations of Dirac-type fermion fields permeating space-time. The forces acting between fermions are mediated by an exchange of bosons, and all interactions must preserve the over-all symmetry of the theory. Since its completion in the early 1970s, it has been shown to an impressive degree of precision that it accurately describes the interactions between elementary particles due to the Strong Force, the Weak Force and the electromagnetic force. It can also explain how quarks and leptons acquire their masses via the Higgs mechanism, whose by-product, the Higgs boson, was detected at the LHC in 2012\todo{cite Higgs discovery}. Despite its success, the Standard Model has some known shortcomings such as its incompatibility with General Relativity and inability to explain cosmological phenomena most commonly interpreted as Dark Matter and Dark Energy.  Most relevantly for this work, it predicts that neutrinos should be massless and therefore does not allow for neutrino oscillations. Since neutrino oscillations can be experimentally observed at very high statistical significance\todo{cite nobel prize SuperK}, it is clear that the SM has to be extended in such a way that neutrino masses can be accommodated. There are several candidate theories for such an extension, but none of them could so far be confirmed experimentally. This chapter will first describe the properties and interactions of neutrinos as they are described by the SM. Then, it will show the mathematical formulation of neutrino oscillations, and finally it will describe some of the simplest extensions to the SM that could explain how neutrinos acquire their mass.

\section{Standard Model particles}

The elementary particles of the SM are organized into fermions and bosons, where fermions make up the observable matter while bosons are the particles that mediate forces. The number of force-mediating bosons is determined by the generators of the symmetry groups that all interactions must obey, while the strength of each force is determined by a \emph{coupling constant} that has to be estimated experimentally. There are eight massless gluons that correspond to the generators of the $\mathrm{SU}(3)_C$ group and mediate the Strong force. All Strong interactions conserve the so-called \emph{color} charge of the involved particles. The symmetry group $\mathrm{SU}(2)_L \times \mathrm{U}(1)_Y$ is the combined symmetry of the \emph{electroweak} force and produces the gauge boson fields $W_1$, $W_2$, $W_3$ and $B$. The electroweak symmetry group is broken into the $\mathrm{U}(1)_Q$ group by interactions of fermions with the Higgs field (further described below) that mixes the $W$ and $B$ fields into massive $W^\pm$ and $Z^0$ bosons and a massless photon $\gamma$ such that
\begin{align}
    Z &= \cos \theta_W W_3 - \sin \theta_W B \\
    \gamma &= \sin \theta_W W_3 + \cos \theta_W B\\
    W^\pm &= \frac{1}{\sqrt{2}} (W_1 \mp iW_2)\;,\label{eq:ew-boson-definitions}
\end{align}
where $\theta_W$ is the so-called \emph{Weinberg angle}.
\begin{margintable}
    \caption{Fermions in the Standard Model. The electric charge, Q, is the conserved charge of the $\mathrm{U}(1)_Q$ symmetry group.}
    \label{tab:fermions-sm}
    \centering
    \begin{tabular}{ccccc} \toprule
    & \multicolumn{3}{c}{generation} & \\ \cmidrule{2-4}
    & 1 & 2 & 3 & Q \\ \midrule
    \multirow{4}{*}{\rotatebox[origin=c]{90}{quarks}}\\
    & u & c & t & $+\nicefrac{2}{3}$ \\
    & d & s & b & $-\nicefrac{1}{3}$ \\
    \\ \midrule
    \multirow{4}{*}{\rotatebox[origin=c]{90}{leptons}}\\
    & $\nu_e$ & $\nu_\mu$ & $\nu_\tau$ & 0 \\
    & $e$ & $\mu$ & $\tau$ & $-1$ \\
    \\ \bottomrule
    \end{tabular}
\end{margintable}
The fermions of the SM are divided into quarks and leptons. Quarks participate in all strong, weak and electromagnetic interactions and are always found in combinations that form baryons (protons, neutrons) or mesons (kaons, pions). The leptons, on the other hand, do not participate in strong interactions. Charged leptons are massive and participate in both the weak and the electromagnetic force, while neutral leptons, the neutrinos, are massless and participate only in weak interactions. All fermions can be grouped into three \emph{generations} of quarks and leptons that are only distinguished by their masses, leading to a convenient arrangement of quarks and leptons into a $3\times4$ scheme as shown in \reftab{fermions-sm}. For each (massive) fermion, there exists a left-handed and a right-handed component. The left-handed components of each generation form a doublet of the $\mathrm{SU}(2)_L$ group with weak isospin $\frac{1}{2}$, while the right-handed components are singlets. The right-handed and left-handed fields for one generation and their charges are summarized in \reftab{fermions-one-generation}.
\begin{margintable}
    \caption{Eigenvalues of the weak isospin $I$, of its third component $I_3$ and the hypercharge $Y = 2(Q - I_3)$ for one generation of fermions. Reproduced from \cite{giunti-kim-neutrino}.}
    \label{tab:fermions-one-generation}
    \centering
    \begin{tabular}{cccc} \toprule
    & $I$ & $I_3$ & $Y$ \\ \midrule
    \multirow{2}{*}{$L_L \equiv \begin{pmatrix} \nu_{eL} \\ e_L \end{pmatrix}$} & \multirow{2}{*}{$\nicefrac{1}{2}$} & $\nicefrac{1}{2}$ & \multirow{2}{*}{-1}\\
    & & $-\nicefrac{1}{2}$ & \\ \midrule
    $e_R$ & 0 & 0 & -2 \\ \midrule
    \multirow{2}{*}{$Q_L \equiv \begin{pmatrix} u_L \\ d_L \end{pmatrix}$} & \multirow{2}{*}{$\nicefrac{1}{2}$} & $\nicefrac{1}{2}$ & \multirow{2}{*}{$\nicefrac{1}{3}$}\\
    & & $-\nicefrac{1}{2}$ & \\ \midrule
    $u_R$ & 0 & 0 & $\nicefrac{4}{3}$ \\
    $d_R$ & 0 & 0 & $-\nicefrac{2}{3}$ \\\bottomrule
    \end{tabular}
\end{margintable}

\subsection{Electroweak Symmetry Breaking}
\labsec{ew-symmetry-breaking}
The process of breaking the $\mathrm{SU}(2)_L \times \mathrm{U}(1)_Y$ symmetry group deserves special attention for the purposes of this work, because it is the process by which the exchange bosons of the Weak force acquire their mass. If the symmetry was unbroken, as it is the case for the $\mathrm{SU(3)}$ group of the Strong force, then the exchange bosons would all remain massless, just like the gluons. To simplify the discussion, the process can be illustrated using only the first generation of SM fermions. The starting point is to introduce the Higgs doublet of complex scalar fields
\begin{equation}
    \Phi = \begin{pmatrix}
        \Phi^+ \\
        \Phi^0
    \end{pmatrix}\;,\label{eq:higgs-doublet}
\end{equation}
where $\Phi^+$ is charged and $\Phi^0$ is neutral\sidenote{In a more general discussion, the Higgs doublet would be written down without assigning the charges a priori, they would be derived later. See \cite{schwartz_2013} for a more rigorous derivation.}. The Lagrangian describing the dynamics of this field,
\begin{equation}
    \mathcal{L}_\mathrm{Higgs} = (D_\mu \Phi^\dag)(D^\mu \Phi) - \lambda \left( \Phi^\dag \Phi - \frac{v^2}{2} \right)^2\;,\label{eq:higgs-lagrangian}
\end{equation}
with the covariant derivative
\begin{equation}
    D_\mu \Phi= \partial_\mu \Phi - ig W_\mu^a \tau^a \Phi - \frac{1}{2}ig'B_\mu \Phi
\end{equation}
is invariant under $\mathrm{SU}(2)_L \times \mathrm{U}(1)_Y$ symmetry and adds a quartic self-interaction potential with the parameters $\lambda$ and $v$, where $\lambda$ is taken to be positive, such that the potential is bounded from below. The fields $W_\mu^a$ in the covariant derivative correspond to the gauge bosons of the $\mathrm{SU}(2)_L$ group whose generators are $\tau^a = \frac{\sigma^a}{2}$, where $\sigma^a$ are the Pauli matrices. The $B_\mu$ field is the boson of the $\mathrm{U}(1)_Y$ group. The factors $g$ and $g'$ are the $\mathrm{SU}(2)_L$ and $\mathrm{U}(1)_Y$ coupling constants, respectively, and are related to the Weinberg angle by
\begin{equation}
    \tan \theta_W = \frac{g'}{g}\;.\label{eq:weinberg-angle}
\end{equation}
Because the potential has a minimum that is not at zero, the field $\Phi$ acquires a non-zero \emph{vacuum expectation value} (VEV) where $\Phi^\dag \Phi = \frac{v^2}{2}$. Since the vacuum is electrically neutral, this VEV can only come from the neutral part, $\Phi^0$, of the doublet and can be written as
\begin{equation}
    \Phi_\mathrm{VEV} = \frac{1}{\sqrt{2}}\begin{pmatrix}
        0\\
        v
    \end{pmatrix}\;.
\end{equation}
This vacuum expectation value is no longer symmetric under the $\mathrm{SU}(2)_L \times \mathrm{U}(1)_Y$ group, but it still is symmetric under the $\mathrm{U}(1)_Q$ group in which electric charge is conserved. To see what happens to the Lagrangian, the field $\Phi$ can be expressed in the unitary gauge as a variation around the VEV such that
\begin{equation}
    \Phi(x) = \begin{pmatrix}
        0 \\
        v + H(x)
    \end{pmatrix}\;.
\end{equation}
Plugging this into the Lagrangian in \refeq{higgs-lagrangian} and re-writing the $W_\mu^i$ and $B_\mu$ fields in terms of $Z$ and $W^\pm$ using the relationships given in \refeq{ew-boson-definitions} and \refeq{weinberg-angle} we find
\begin{align}
    \mathcal{L}_\mathrm{Higgs} = &\hspace{1em}\frac{1}{2}(\partial H)^2 - \lambda v^2 H^2 - \lambda v H^3 - \frac{\lambda}{4}H^4 \\
    &+ \frac{g^2v^2}{4} W_\mu^\dag W^\mu + \frac{g^2 v^2}{8\cos^2\theta_W}Z_\mu Z^\mu \label{eq:boson-mass-terms}\\
    &+ \mathrm{Higgs\;vertices}\;,
\end{align}
where Higgs vertices are 3-vertices and 4-vertices between the Higgs field and the $W$ and $Z$. The notable part is that the $W$ and $Z$ bosons have acquired a kinetic term in \refeq{boson-mass-terms} with a mass that is proportional to the VEV of the Higgs field, giving massive exchange bosons to the Weak force\sidenote{The massless photon field is found by expanding the full electroweak Lagrangian in the same way, which we neglect here for the sake of brevity.}.

\subsection{Charged Fermion Masses}
\label{sec:charged-fermion-masses}
In Quantum Electrodynamics, a Lorentz\todo{check spelling}-invariant mass term for spin-$\frac{1}{2}$ fermions can be written as a product of left-handed and right-handed Weyl spinors, also known as the Dirac mass
\begin{equation}
    \mathcal{L}_\mathrm{Dirac} = m (\bar{\Psi}_R \Psi_L - \bar{\Psi}_L \Psi_R)\;.
\end{equation}
However, such a term is not invariant under $\mathrm{SU}(2)_L \times \mathrm{U}(1)_Y$ and therefore cannot be added to the SM Lagrangian directly. Fortunately, masses for fermions can be recovered if we add a Yukawa coupling term between the fermions and the Higgs field, such as
\begin{equation}
    \mathcal{L}_\mathrm{Yuk} = -y \bar{L}_L \Phi e_R + \mathrm{h.c.}\;,
\end{equation}
where $L_L$ denotes the $\mathrm{SU}(2)_L$ doublet listed in \reftab{fermions-one-generation} and $y$ is the Yukawa coupling constant. When the VEV is inserted into this term, it produces a mass term $-m_e (\bar{e}_L e_R + \bar{e}_R e_L)$ with $m_e = \frac{y}{\sqrt{2}}v$ for the charged leptons and the down-type quarks $d$, $s$, and $b$.  A similar term that is also invariant under $\mathrm{SU}(2)_L$ and generates masses for the up-type quarks is $-y \bar{L}_L \tilde{\Phi} u_R$, where we defined $\tilde{\Phi} \equiv i \sigma_2 \Phi$. When all possible terms of this form for all generations of quarks are put together, the generations mix among each other in a way that is very similar to neutrino mixing, which is beyond the scope of this discussion.
%Putting all possible terms of this form for all generations of quarks together, the Lagrangian for quark masses is
%\begin{equation}
%   \mathcal{L}_\mathrm{mass} = -Y_{ij}^d \bar{Q}^i H d_R^j - Y_{ij}^u \bar{Q}^i \tilde{H} u_R^j + \mathrm{h.c.}\;,
%\end{equation}
%which contains two matrices of coupling constants, $Y^d$ and $Y^u$ with indices $i$ and $j$ enumerating the generations. After symmetry breaking, the mass terms become
%\begin{align}
%   \mathcal{L}_\mathrm{mass} &= -\frac{v}{\sqrt{2}}\left[  Y^d_{ij} \bar{d}_L^i d_R^j +  Y^u_{ij} \bar{u}_L^i u_R^j \right] + \mathrm{h.c.} \\
%   &=  -\frac{v}{\sqrt{2}}\left[ \bar{d}_L Y_d d_R + \bar{u}_L Y_u u_R \right] + \mathrm{h.c.}\;.
%\end{align}
%These terms are not the masses of the physically observed quarks, because the matrices $Y_{d,u}$ can in principle have off-diagonal terms. To get well-defined quark masses, therefore, the $Y_{d,u}$ matrices have to be diagonalized with a unitary transformation such that
%\begin{equation}
%    Y_{u,d} Y_{u,d}^\dag = U_{u,d} M_{u,d}^2 U_{u,d}^\dag\;,
%\end{equation}
%where the $M_{u,d}$ are diagonal mass matrices. This transformation introduces two different bases in which the Lagrangian can be written: In the \emph{mass basis}, the mass terms are diagonal

\subsection{Neutrino Masses}

The Higgs mechanism described in \refsec{charged-fermion-masses} necessitates both left-handed and right-handed Weyl spinors to interact with the Higgs field. Since there are no right-handed neutrinos in the SM, it predicts that they should be massless, in contradiction to experimental evidence. However, if we add right-handed neutrino fields into the model, then neutrino masses can be generated in a way that is tantalizingly similar to that of up-type quarks by adding interactions of the form $Y_{ij}^\nu \bar{L}^i \tilde{\Phi}\nu_R^j$ to the Lagrangian. Such a right-handed field would be uncharged with respect to all symmetry groups of the SM and would therefore not interact with any other particle and is hence referred to as a \emph{sterile neutrino}. Because neutrinos are electrically neutral, another possibility for a mass term that is allowed by the symmetry of the SM is the so-called \emph{Majorana mass}, $m \nu_R^c \nu_R$, in which $\nu_R^c=\nu_R^T \sigma_2$ is the charge conjugate Weyl spinor. The most general Lagrangian including all Yukawa couplings and Majorana mass terms of the lepton sector is
\begin{equation}
    \mathcal{L}_\mathrm{mass} = -Y_{ij}^e \bar{L}^i \Phi e_R^j - Y_{ij}^\nu \bar{L}^i \tilde{\Phi} \nu_R^j - iM_{ij}(\nu_R^i)^c \nu_R^j + \mathrm{h.c.}\;,
\end{equation}
where the indices $i$ and $j$ run over the generations $e$, $\mu$, and $\tau$ and the matrix $Y_{ij}^e$ contains the Yukawa coupling constants.

%\section{Neutrino Properties}

%Neutrinos in the Standard Model are spin-$\nicefrac{1}{2}$ particles and interact solely via the Weak force, which is mediated by the exchange of $W^{\pm}$ and $Z^0$ bosons with masses of 80.4 and 91.2~GeV, respectively. Due to the large mass of the exchange bosons, the Weak force can only act on extremely short distances and interactions are generally much more rare than for the Strong and electromagnetic forces. For this reason, neutrinos can pass undisturbed over cosmological distances and penetrate enormous amounts of matter, which makes them excellent messenger particles for astronomy. On the other hand, this also means that they are very difficult to detect. Indeed, when their existence was first proposed by Pauli in the 1930s to explain the  continuous energy spectrum of electrons produced in  radioactive beta decays, it was thought that they might be entirely unobservable. It was not until 1956 that the first neutrino was detected, by Frederick Reines and Clyde Cowan, using a nuclear reactor as a source.
%%Since then, neutrinos have been detected from a variety of sources, including the Sun, the atmosphere, nuclear reactors, and accelerators.
%
%\subsection{Quantum Numbers and Helicity}
%
%Neutrinos are electrically neutral and have a lepton number that is defined empirically and results in the three known neutrino flavors $\nu_e$, $\nu_\mu$ and $\nu_e$. The neutrino fields of each flavor form a doublet with the charged lepton of the same lepton number, that is, the electron, the muon and the tauon, and all Weak interactions conserve the lepton number.

\section{Weak Interactions After Symmetry-Breaking}
\label{sec:ew-interactions}

Neutrino interactions with matter are described by Weak force interactions after electroweak symmetry-breaking described in \refsec{ew-symmetry-breaking}.
The Lagrangian for these interactions can be written as the sum of the neutral-current (NC) and charged-current (CC) interactions. The NC part describes the exchange of neutral $Z^0$ bosons, which couples to all quarks and leptons except for right-handed neutrinos. For leptons, the NC Lagrangian reads
\begin{equation}
  \mathcal{L}_\mathrm{NC,L} = -\frac{g}{2 c_W^2} \sum_{\mathcal{l}=e,\mu,\tau} (\bar{\nu}_{\mathcal{l}, L} \gamma^\mu \nu_{\mathcal{l}, L} + (2 s_W^2 - 1) \bar{e}_{\mathcal{l}, L} \gamma^\mu e_{\mathcal{l}, L} + 2s_w^2 \bar{e}_{\mathcal{l}, R} \gamma^\mu e_{\mathcal{l}, R}) Z^0_\mu\;, \label{eq:ew-nc-lagrangian}
\end{equation}
where $\nu$ denotes a neutrino field, $e$ a lepton field, and the subscripts $L$ and $R$ denote left-handed and right-handed fields, respectively. Since the Lagrangian is written in the flavor basis, the neutrino fields are superpositions of mass eigenstates. The coefficient $s_W$ ($c_W$) is the sine (cosine) of the Weinberg angle and $g$ is the coupling constant that determines the overall strength of the electroweak force. This Lagrangian leads to the trilinear couplings shown in \reffig{nc-vertices}. The couplings to quarks have the same form as those to the charged leptons up to a difference in coupling strength.
\begin{figure}
\centering
\begin{subfigure}{0.3\linewidth}
    \begin{tikzpicture}
        \begin{feynman}[small]
        \vertex (a) {\(\pbar{\nu}_\mathcal{l}\)};
        \vertex [below right=of a.south] (center);
        \vertex [above right=of center.south] (b) {\(\pbar{\nu}_\mathcal{l}\)};
        \vertex [below=of center] (c) {\(Z^0\)};

        \diagram* {
          (a) -- [fermion] (center) -- [fermion] (b),
          (center) -- [boson] (c)
        };
        \end{feynman}
    \end{tikzpicture}
    \caption{neutrinos}
\end{subfigure}
\begin{subfigure}{0.3\linewidth}
    \begin{tikzpicture}
        \begin{feynman}[small]
        \vertex (a) {\(\mathcal{l}^\pm\)};
        \vertex [below right=of a.south] (center);
        \vertex [above right=of center.south] (b) {\(\mathcal{l}^\pm\)};
        \vertex [below=of center] (c) {\(Z^0\)};

        \diagram* {
          (a) -- [fermion] (center) -- [fermion] (b),
          (center) -- [boson] (c)
        };
        \end{feynman}
    \end{tikzpicture}
    \caption{charged leptons}
\end{subfigure}
\caption{Neutral-current lepton interaction vertices.}
\label{fig:nc-vertices}
\end{figure}
Neutral-current interactions conserve both the electric charge and lepton number, such that a neutral-current interaction of a neutrino will always produce a neutrino of the same flavor.

The charged-current (CC) part of the Weak Lagrangian in the flavor basis is
\begin{equation}
    \mathcal{L}_\mathrm{CC} = -\frac{g}{\sqrt{2}} \sum_{\mathcal{l}=e,\mu,\tau} \bar{\nu}_{\mathcal{l},L} \gamma^\mu e_{\mathcal{l},L} W^+_\mu + \bar{e}_{\mathcal{l},L} \gamma^\mu \nu_{\mathcal{l},L} W^-_\mu\;.\label{eq:ew-cc-lagrangian}
\end{equation}
In contrast to neutral current interactions, the charged current interactions couple exclusively to left-handed fields\sidenote{The left-handed fields in the flavor basis are superpositions of mass eigenstates that may contain a (charge-conjugated) right-handed Majorana component as described in \refsec{neutrino-masses}}. The associated lepton interaction vertices are shown in \reffig{cc-vertices}.
\begin{figure}
\centering
\begin{subfigure}{0.3\linewidth}
    \begin{tikzpicture}
        \begin{feynman}[small]
        \vertex (a) {\(\bar{\nu}_\mathcal{l}\)};
        \vertex [below right=of a.south] (center);
        \vertex [above right=of center.south] (b) {\(\mathcal{l}^{+}\)};
        \vertex [below=of center] (c) {\(W^-\)};

        \diagram* {
          (a) -- [fermion] (center) -- [fermion] (b),
          (center) -- [boson] (c)
        };
        \end{feynman}
    \end{tikzpicture}
    \caption{Coupling to $W^-$}
\end{subfigure}
\begin{subfigure}{0.3\linewidth}
    \begin{tikzpicture}
        \begin{feynman}[small]
        \vertex (a) {\(\nu_\mathcal{l}\)};
        \vertex [below right=of a.south] (center);
        \vertex [above right=of center.south] (b) {\(\mathcal{l}^{-}\)};
        \vertex [below=of center] (c) {\(W^+\)};

        \diagram* {
          (b) -- [fermion] (center) -- [fermion] (a),
          (center) -- [boson] (c)
        };
        \end{feynman}
    \end{tikzpicture}
    \caption{Coupling to $W^+$}
\end{subfigure}
\caption{Charged-current lepton interaction vertices.}
\label{fig:cc-vertices}
\end{figure}
The weak CC interactions with quarks couple up-type quarks to down-type quarks with the Lagrangian
\begin{equation}
    \mathcal{L}_\mathrm{CC,Q} = \frac{g}{\sqrt{2}}\bar{u}_L\gamma^\mu d_L W_\mu\;.
\end{equation}

\subsection{Neutrino Cross-Sections}
\label{sec:neutrino-xsec}


\section{Neutrino Sources}
\section{Neutrino Sources}

\subsection{Solar neutrinos}
\label{sec:solar-nu}

Here we describe how solar neutrinos come to be.


\setchapterstyle{kao}
\setchapterpreamble[u]{\margintoc}
\chapter{Neutrino Masses and Oscillations}
\labch{massoscillations}

\section{Neutrino Oscillations}
\section{Atmospheric Neutrino Oscillations}
\section{Current Measurements of three-flavor Oscillations}
\section{Neutrino Mass Generation and Sterile Neutrinos}
\section{Current Status of Sterile Neutrino Searches}

\setchapterimage[6.5cm]{icecube}
\setchapterpreamble[u]{\margintoc}
\chapter{Neutrinos in IceCube and DeepCore}
\labch{icecube}
% \begin{fquote}[Roald Amundsen][The South Pole][1912] We must always remember with gratitude and admiration the first sailors who steered their vessels through storms and mists, and increased our knowledge of the lands of ice in the South.
% \end{fquote}

The IceCube Neutrino Observatory is a gigaton-scale Cherenkov detector located at the geographic South Pole in close proximity to the Amundsen-Scott South Pole Station.
Constructed over the course of several deployment seasons between 2006 and 2011, it instruments approximately one cubic-kilometer of Antarctic glacier with optical sensors that can detect faint flashes of light that are produced when charged particles travel through the ice, such as those produced by neutrino interactions.
The detector serves as both a telescope to study the astrophysical origin of neutrinos and an instrument to measure their fundamental properties.

This chapter describes the instrumentation and layout of the IceCube detector, the interactions that particles undergo when they interact with the ice, and finally the signals that these interactions produce in the detector.

\section{The IceCube in-ice Array and DeepCore}

The IceCube Neutrino Observatory consists of the so-called \emph{in-ice} array, optimized for astrophysical neutrino observations, the \emph{DeepCore} array, used primarily for the observation of atmospheric neutrinos, and the \emph{IceTop} surface array that can be used to study air showers from cosmic rays.

\subsection{The Antarctic Ice}

The detection medium of the IceCube detector is the Antarctic glacier that has formed from layers of snow being deposited top of each other over the course the past $\sim\num{100000}$ years\sidecite{iceage}.
The weight of the upper layers compresses the lower layers into a dense, crystalline structure.
As a result, the optical properties of the ice change mostly in the direction perpendicular to the layers, forming a geological record of the atmospheric conditions of the Earth.
The transmission of light through the ice is primarily characterized by the scattering and absorption length. Within the volume of IceCube, scattering lengths vary between \SI{20}{\meter} and \SI{100}{\meter}, while absorption lengths range from \SI{100}{\meter} to \SI{400}{\meter}. Both quantities are highly correlated, such that the absorption length is approximately four times as large as the scattering length\sidecite{tc-2022-174}.
This stratigraphy was traced at millimeter resolution using a laser dust logger deployed down seven IceCube drill holes as described by \cite{dustlogger}.
The most notable feature of the stratigraphy is the \emph{dust layer} at depths between 2000~m and 2100~m as shown in \reffig{icecube-schematic}.
The optical properties of the ice within the dust layer are particularly poor.
The ice below the dust layer where the DeepCore fiducial volume is located has the best optical properties of the entire IceCube volume.

\subsection{In-Ice Array}
The 5160 Digital Optical Modules (DOMs) that make up the IceCube in-ice array are distributed over 86 strings.
Of these, 78 are arranged on a hexagonal grid spanning an area of approximately one square-kilometer with a horizontal spacing of $\sim$150~m with respect to their closest neighboring strings\sidecite{icecube_detector_17}. Each of these strings holds 60 DOMs at depths between 1450~m and 2450~m with a 17~m vertical spacing.
The volume and instrumentation density of this array is optimized for astrophysical neutrinos that are found at energies above 1~TeV\cite{icecube_detector_17}.
The electric signals measured in each DOM are digitized  and sent to the \emph{IceCube lab (ICL)}, where the signal is processed, compressed, and sent North via satellite for offline processing.
\reffig{ic_detector} gives an overview of the detector including the IceCube Lab, the ice surface and the bedrock, and a schematic of the layout of the strings is shown in \reffig{icecube-schematic}.

\begin{figure}
	\centering \includegraphics{figures/icecube/IceCubeArray_slim.png}
	\caption{An overview of the IceCube detector}
	\label{fig:ic_detector}
\end{figure}

\subsubsection{DeepCore}
The remaining 8 strings that are not part of the hexagonal grid are located near the center of the IceCube detector and form the \emph{DeepCore} sub-array\sidecite{DeepCore}.
The DOMs on the DeepCore strings have a higher quantum efficiency than those in the rest of the detector and are placed more closely together to lower the minimum energy threshold for neutrino observations to a few GeV.
Of the 60 DOMs on each DeepCore string, 50 are placed at depths between 2100~m and 2500~m, where the ice is the most transparent compared to the rest of the IceCube's volume (see also the side band in the bottom panel of Figure~\ref{fig:icecube-schematic}).
Together with 7 strings from the in-ice array, the DeepCore strings instrument the DeepCore 20~MT \emph{fiducial volume} as shown in the upper panel of Figure~\ref{fig:icecube-schematic}.
The remaining 10 DOMs are located at depths between 1750~m and 1850~m and are used as a veto cap to reject atmospheric muons entering the detector directly from above.
In addition, the larger hexagonal IceCube array also serves as a veto for observations inside the DeepCore fiducial volume.
\begin{figure}
    \includegraphics[width=0.9\linewidth]{figures/icecube/DeepCore_geometry.pdf}
    \caption{Schematic view of the IceCube detector as seen from the top (upper panel) and the side(lower panel). The DeepCore fiducial volume is indicated by the hexagon in the upper panel and the green shaded area in the bottom panel. The side-band on the lower panel shows the scattering and absorption coefficients as a function of depth.}
    \label{fig:icecube-schematic}
\end{figure}

\subsection{IceTop}

In addition to the in-ice array, IceCube also contains a surface array called \emph{IceTop}, consisting of 81 stations spread across an area of 1~km$^2$ that is used to detect muons from air showers.
It is typically used as a veto against atmospheric muons, but also functions as a detector in its own right measuring the spectrum and composition of cosmic particles.
However, it is not relevant to the measurement presented in this thesis.

\subsection{Digital Optical Modules}
\label{sec:dom-daq}
The Cherenkov radiation produced by charged particles in the ice is detected and digitized by Digital Optical Modules (DOMs).
Each module consists of a photo-multiplier tube (PMT)\sidecite{Abbasi_2010} and electronics housed in a transparent, spherical glass vessel that can withstand the enormous pressure below a water column of 2.5~km\cite{icecube_detector_17}.
They are each held in place by a harness attached to chains that allows the string cable to pass beside the DOM as shown in \reffig{dom-cable-assembly}.
\begin{marginfigure}[*-20]
    \includegraphics[width=\textwidth]{figures/icecube/domfig1a-DOM3DModel.pdf}
    \caption{Schematic of a DOM, taken from \cite{icecube_detector_17}.}
    \label{fig:dom-schematic}
\end{marginfigure}
\begin{marginfigure}
    \includegraphics[width=\textwidth]{figures/icecube/domfig2a-CableAssembly.pdf}
    \caption{Schematic of the cable assembly of a DOM. Figure taken from \cite{icecube_detector_17}.}
    \label{fig:dom-cable-assembly}
\end{marginfigure}
The PMTs have a diameter of 10 inches and are sensitive to photons with wavelengths between 300~nm and 650~nm, with a maximum quantum efficiency of about 25\% at 390~nm.
Inside the DeepCore array, the peak efficiency reaches 34\%.
They are shielded from external magnetic fields with a mu-metal grid as shown in the schematic in Figure~\ref{fig:dom-schematic}.
The voltage at the PMT is measured and digitized by the on-board electronics\sidecite{icecube_daq} of the DOM in two separate readouts that are activated when the measured voltage rises above the equivalent of 0.25 photo-electrons (PE).
The first readout is the \emph{fast Analog-Digital Converter (fADC)} and measures the waveform continuously at a rate of 40~MHz.
The second readout, the \emph{Analog Transient Waveform Digitizer (ATWD)}, records the PMT voltage at a rate of 300~MHz in three channels with different gain levels to ensure that a large range of voltages can be recorded without saturation of the output.
The readout frequency of the ATWD is too high to be directly digitized and sent to the surface.
Instead, the ATWD voltage readout is buffered in 128 analog capacitors, corresponding to a readout time of $\sim$420~ns.
The buffered voltages are only digitized when at least one of the nearest or next-to-nearest DOMs on the same string also measures a signal within a 1~$\mu$s time window, which is referred to as the \emph{hard local coincidence (HLC)} condition.
The recorded waveforms are sent to the ICL on the surface, where they are compressed by applying the \emph{wavedeform} algorithm\sidecite{ic_spe_20}.
The output of this algorithm are reconstructed times and charges of single photo-electrons, which are taken as input by all further data processing steps described in section~\ref{sec:data-processing}.

The DOMs also contain a flasher board with 12 LEDs that can be used to emit pulsed light for the purpose of \emph{in-situ} detector calibration during special \emph{flasher runs} of the detector.
During such runs, the charge and time distributions of the observed pulses in the DOMs in response to the LED flashes are measured.
Since the light is emitted at known locations and at known times, the measured distributions allow inference on the absorption and scattering properties of the ice.
Because the total amplitude of the emitted light is less well known, this calibration method is less well suited for calibrating the total optical efficiency of the DOMs.
Instead, this property of the detector is calibrated more accurately from measurements of minimally-ionizing atmospheric muons, for which the energy loss is well known\sidecite{domeff_nick}.


\section{Propagation of Particles in Ice}
\label{sec:particle-interactions}
Neutrinos interacting with the ice mostly interact via Deep Inelastic Scattering (DIS), creating muons, electromagnetic showers, and hadronic showers, depending on the flavor of the neutrino and interaction type.
The secondary particles produced by those interactions travel through the ice at highly relativistic velocities and lose energy primarily through ionization, bremsstrahlung, pair production and photo-nuclear interactions. The fraction that each of these mechanisms contributes to the total energy loss of the particle depends on the type of particle and its energy. When they are electrically charged, they also give off Cherenkov radiation that is then measured by IceCube.

\subsection{Cherenkov Effect}

The IceCube Neutrino Observatory relies entirely on the Cherenkov effect to detect particle interactions. It is created by any electrically charged particle travelling through a transparent medium faster then the speed of light in that medium (i.e. $v>c/n$, where $n$ is the refractive index of the medium) and produces a cone of light moving with the particle similar to a super-sonic shock that is produced by an object travelling through a gas at a velocity above the speed of sound. The effect can be most easily understood according to Huygen's principle as a superposition of spherical light emissions that are produced every time that the particle displaces the charges in the dielectric medium in its closest vicinity as shown in Figure~\ref{fig:cherenkov-sketch}. When the particle is over-taking its own light emissions, they overlap coherently and form a conical light front as illustrated in the bottom panel of Figure~\ref{fig:cherenkov-sketch}.
\begin{marginfigure}
    \includegraphics[width=\textwidth]{figures/icecube/cherenkov/cherenkov_slow.jpeg}
    \includegraphics[width=\textwidth]{figures/icecube/cherenkov/cherenkov_fast.jpeg}
    \caption{An electrically charged particle emitting light while travelling below (upper panel) and above (lower panel) the speed of light in a medium. Image taken from \cite{jackson2012classical}.}
    \label{fig:cherenkov-sketch}
\end{marginfigure}
When the velocity of the particle is very close to the speed of light, as is the case for all (known) particles observable by IceCube, the opening angle of the cone only depends on the refractive index of the medium with
\begin{equation}
    \cos(\vartheta_c)=\nicefrac{1}{n}\;,
\end{equation}
where $n$ is the index of refraction and $\vartheta_c$ is the Cherenkov opening angle.

The frequency spectrum of the Cherenkov emissions of highly relativistic particles depends only on the charge of the particle, $q$, and the (wavelength-dependent) index of refraction, $n(\omega)$, and permeability, $\mu(\omega)$, of the medium. The emitted energy per unit of distance and frequency is given by the Frank-Tamm-Equation, which simplifies in the case of $v\approx c$ to 
\begin{equation}
    \frac{\drm E}{\drm x \drm \omega}=\frac{q^2}{4\pi}\mu(\omega)\omega\left(1-\frac{1}{n^2(\omega)}\right)\,.
\end{equation}
The equation shows that the intensity of the Cherenkov emission generally increases with frequency, and indeed the strongest emissions are in the ultraviolet part of the spectrum.

\subsection{Muons}
\label{sec:muon-propagation}

At energies below 100~GeV, the dominant energy loss for muons is via ionization, and has only a weak dependence on energy. Because the ionization loss is continuous and nearly constant, muons at these energies produce long, track-like signatures in the detector. Above 100~GeV, the losses due to bremsstrahlung, pair production and photo-nuclear interactinos rise quickly in their amplitude and become dominant over ionization at $\sim$1~TeV. The total average energy loss per unit distance, $\left<\drm E/\drm x\right>$, can be approximated combining all radiative energy losses (i.e. all losses except for ionization) into one component and adding it to the ionization loss such that
\begin{equation}
    \left<-\frac{\drm E}{\drm x}\right> = a_I(E) + b_R(E)E\,,
\end{equation}
where $a_I(E)$ and $b_R(E)$ are slowly changing functions describing the ionization loss and the radiative losses, respectively\cite{muonstoppingpower}. For the energy ranges relevant for this work, the energy dependence of $a_I(E)$ and $b_R(E)$ is weak enough such that they can be approximated as constant numbers with $a_I(E)\approx 2\;\mathrm{MeV/cm}$ and $b_R(E)\approx3.4\times10^{-6}\;\mathrm{cm^{-1}}$\cite{muonstoppingpower}. In this approximation, we can calculate the average length of a muon track, $\left<L\right>$, as a function of energy with 
\begin{equation}
    \left<L\right>=\frac{1}{b_R}\log\left(\frac{b_R}{a_I}E + 1\right)\,,
\end{equation}
which gives an average travel distance of 50~m at 10~GeV and 460~m at 100~GeV.

\subsection{Electromagnetic Showers}
\label{sec:em-showers}

In contrast to muons, electrons and positrons lose their energy very quickly by emitting very highly energetic photons due to bremsstrahlung. The energy of the emitted photons is high enough that they spontaneously produce pairs of electrons and positrons. This process is repeated until the electrons and positrons reach their critical energy, which is approximately 78~MeV in water ice\sidecite{pdg}. Below the critical energy, ionization takes over as the predominant mechanism of energy loss, which produces no new shower particles. Another important quantity is the \emph{radiation length}, $X_0$, defined as the distance at which the energy of an electron is reduced to $\nicefrac{1}{e}$ of its initial energy via bremsstrahlung, which is 36~cm in ice\cite{pdg}. The radiation length also determines the scale of the longitudinal development of the shower. Expressing distances in units of radiation length as $t=x/X_0$, the shower intensity follows roughly a gamma distribution parametrized as 
\begin{equation}
    \frac{\drm E}{\drm t} = E_0 b \frac{(bt)^{a-1}e^{-bt}}{\Gamma(a)}\;,
\end{equation}
where the parameters $a$ and $b$ need to be fit empirically\cite{pdg}. Their values for electrons, positrons and photons interacting in ice have been determined from GEANT4\sidecite{geant4} shower simulations in\sidecite{RADEL2013102} to be 
\begin{align}
    a &\approx 2.01 + 1.46 \log_{10}(E_0/\mathrm{GeV}),\; b\approx 0.63\; & (e^+,e^-), \\
    a &\approx 2.84 + 1.34 \log_{10}(E_0/\mathrm{GeV}),\; b\approx 0.65\; & (\gamma).\label{eq:shower-params}
\end{align}
The shower reaches its maximum intensity at a distance of
\begin{equation}
    t_{\mathrm{max}}=\frac{a-1}{b}\;,
\end{equation}
which corresponds to a logarithmic growth of the size of the cascade according to eq.~\ref{eq:shower-params}. The electrically charged components of the electromagnetic shower produce Cherenkov light, where the emissions peak at the Cherenkov angle since the secondary particles are emitted very close to the forward direction.

\subsection{Hadronic Showers}
\label{sec:had-showers}

As discussed in section~\ref{sec:neutrino-xsec}, neutrino interactions above 10~GeV happen almost exclusively via Deep-Inelastic Scattering (DIS). These interactions always produce a hadronic cascade in addition to any leptons in the final state. Hadronic cascades are also the only visible part of the final state of neutral-current interactions. Hadrons (mostly Pions) that are produced in neutrino-nucleon interactions interact strongly with the surrounding ice to create secondary particles and then decay to form additional photons and leptons. Charged secondary particles produce Cherenkov radiation, while neutral secondary particles are invisible to the detector. Because part of the energy deposited in a hadronic shower is not measurable, the inherent uncertainty on the true energy of the primary particle that initiated the interaction is larger. The average visible electromagnetic fraction of a hadronic shower can be parametrized\cite{RADEL2013102} as a function of the initial energy with 
\begin{equation}
    F(E_0) = 1 - (1-f_0)\left(\frac{E_0}{E_s}\right)^{-m}
\end{equation}
with a variance of 
\begin{equation}
    \sigma_F(E_0) = \sigma_0 \log(E_0)^{-\gamma}\,.
\end{equation}
The parameters $f_0$, $E_s$, $m$, $\sigma_0$, and $\gamma$ are fit to GEANT4 simulation results for hadronic showers induced by different primary particles in\cite{RADEL2013102}.
The Cherenkov emissions from the charged components of the shower still peak around the Cherenkov angle as they do for electromagnetic showers, but the emission profile is more smeared out due to the larger variations in particle kinematics.

\section{Particle Signatures in IceCube}

The characteristic signature left by a particle in the IceCube detector depends on its type, the type of interaction and the energy. Given the spacial resolution of the detector array, individual particles that are produced by these interactions cannot be resolved individually, with the exception of the muon. Instead, signatures of hadronic and electromagnetic cascades (see sections~\ref{sec:had-showers} and \ref{sec:em-showers}) are summarised as \emph{cascades}, while the elongated signature of a muon travelling through the detector (see section~\ref{sec:muon-propagation}) is called a \emph{track}.


% \subsection{Systematic uncertainties not included in the measurement}
\label{sec:other-uncertainties}

Several other potential sources of systematic uncertainties were investigated during the development of the analysis presented in this work whose impact was found to be negligible.


\subsubsection{Other oscillation parameters}
\label{sec:other-oscillation-syst}

The impact of other oscillation parameters besides $\theta_{23}$ and $\Delta m^2_{13}$ on the observed signal was assessed within the bounds set by other experiments and was found to be negligible. One exception is the CP-violating phase $\delta_{\mathrm{CP}}$, which had the potential to cause a small bias in the analysis as described in the assessment of parameter impact in \refch{measurement-three-flavor}. The analyses presented in this work only test the hypothesis under which $\delta_{\mathrm{CP}}=0$ for simplicity and to produce a result that is directly comparable to that of other experiments.

\subsubsection{Depth-dependent ice properties}
\label{sec:depth-dependent-ice-properties}

In the parametrization of the uncertainties of the detector properties described in \refsec{detector-unc}, variations of the scattering and absorption coefficients are only described by global, depth-independent scaling factors.
In principle, the error on the properties of the ice could also change as a function of depth.
For instance, one would expect that the uncertainty on the ice absorption is larger in regions with increased dust deposition, because the dust will absorb the LED light that is used to calibrate the ice model.
Of particular interest for the analysis presented in this work are variations of the ice properties at length scales of the DeepCore fiducial volume located within DeepCore.
Variations at much longer scales would be indistinguishable from uniform variations given the size of the event signatures observed below 100~GeV, while variations at much shorter scales are expected to average out.
To test how significantly such a variation would impact the final level histograms, two MC sets are produced in which the scattering and absorption coefficients vary following a sigmoid function function centered in DeepCore with an amplitude of $\pm 2\%$ in opposing directions as shown in \reffig{step-function-ice-model}.
\begin{figure}
    \centering
    \missingfigure[figwidth=0.8\linewidth]{Show step function variations, see  \href{https://drive.google.com/file/d/1TV0r1VzRbRPxlQeeCuq8DaZzeQloJZ_J/view}{presentation on lowen call}.}
    \caption{Perturbation of the scattering and absorption coefficients with respect to the nominal ice model applied in additional MC sets.}
    \label{fig:step-function-ice-model}
\end{figure}
The size of this variation corresponds approximately a $1\sigma$-allowed variation according to flasher calibration data. Then, for every bin in the final analysis histogram, a linear regression is fit to the bin counts of the nominal MC set and the two variations assuming that they correspond to $\pm 2\%$ variations of a parameter.
The resulting slopes were found to be indistinguishable from pure statistical variation and it was concluded that the impact of such a hypothetical new systematic uncertainty would be insignificant.\todo{This was only done for the high-stats sample: redo for verification sample?}

%In previous oscillation studies using track-like events in the TeV energy range \sidecite{MEOWS}, this effect was found to be non-negligible. To model depth-dependent uncertainties of the ice properties, a method was developed in which the depth-dependence of the scale factor for scattering and absorption coefficients is decomposed into Fourier-modes, and the impact of each mode on the final analysis histogram can be calculated\sidecite{snowstorm}. In the analysis presented in \cite{MEOWS}, the impact of high-frequency Fourier modes was found to be negligible and therefore the uncertainties of the bulk ice was modeled using the lowest fourth modes.

\subsubsection{Atmospheric density}
\todo[inline]{Describe test of atmospheric density impact (or get feedback if that is even necessary}

\subsubsection{$K/\pi$-air interactions}
\todo[inline]{Describe this part, although it might also be just too much.}

% \section{Data Processing}
\label{sec:data-processing}

\subsection{Trigger}
As described in section \ref{sec:dom-daq}, the high-frequency ATWD waveform digitization in each DOM is triggered when it and its adjacent or next-to-adjacent neighbors on the same string record a voltages of at least 0.25 PE-equivalent within a $\pm$1~$\mu$s time window, which is referred to as the Hard Local Coincidence (HLC) condition. Data acquisition for DeepCore is triggered when this condition is fulfilled for at least three DOMs inside the DeepCore fiducial volume within a $\pm$2.5~$\mu$s window. If this condition is met, the waveforms for all DOMs that have observed voltages of at least 0.25~PE within a $\pm$10~$\mu$s time window centered around the trigger time are recorded. A DOM that is included in this readout but for which the HLC condition has not been met is said to fulfill the \emph{Soft Local Coincidence} (SLC) condition. The DeepCore trigger rate is less than 10~Hz and will trigger on \~70\% of $\nu_\mu$ events at 10~GeV and >90\% of $\nu_\mu$ events at 100~GeV\cite{DeepCore}.

\subsection{Online Filter}

Once the trigger condition is met, the recorded waveforms within the trigger window are converted into reconstructed pulses and are then passed into a set of \emph{online} filters (i.e. filters running on hardware at the Pole). These filters are each designed to select events that are relevant to different physics measurements that are performed within the IceCube collaboration. For the purposes of the analysis presented in this thesis, events are selected using the \emph{DeepCore filter}. This filter is designed to select events that start inside the DeepCore fiducial volume and to reject those that are consistent with muons entering the detector from the outside.
\begin{marginfigure}
    \includegraphics[width=\textwidth]{figures/icecube/eventviews/FilterDiagram.pdf}
    \caption{Example of an event that would be rejected by the online filter algorithm. DOMs that have observed light are highlighted in color depending on time from red (early hits) to blue (late hits). DOMs that have not observed any light are shown as black dots. Figure taken from \cite{DeepCore}.}
    \label{fig:online-filter-event}
\end{marginfigure}
The filter first applies a noise cleaning algorithm based on a coincidence condition between hits on different DOMs, where hits in DOMs for which the HLC condition was met are always kept. The cleaned hit series is split between those hits that fall within the DeepCore fiducial volume and those outside of it. The veto algorithm then calculates the COG in space and time of the hits inside the fiducial volume and then the velocity that a signal would have to travel from each hit occurring outside the fiducial volume to coincide with the COG. If this velocity is close to the speed of light (between $0.25\;\mathrm{ns/s}$ and $0.4\;\mathrm{ns/s}$) for at least one hit, the event is rejected because it is consistent with a muon traveling through the veto region and entering DeepCore. Figure~\ref{fig:online-filter-event} shows an example of an event that would be rejected by the online filter. Only events passing the trigger and filter condition are sent North via satellite for further \emph{offline} filtering.

\subsection{Offline Filter}

The offline filter is separated into subsequently applied \emph{levels}, referred to as L3, L4, and L5, where each level reduces the amount of background (atmospheric muons and noise) by approximately an order of magnitude while keeping most of the DeepCore starting events that are the target of the selection.

\subsubsection{Level 3}
At the lowest offline filter level, L3, cuts are applied to simple variables that remove the most easily identifiable background events while using only few computational resources. The variables aimed at cutting noise consist mostly of different DOM hit counts within hit series to which noise cleaning algorithms have been applied. The cuts aimed at removing muons consist of conditions on the number of hits in the veto region as well as conditions on the vertical position of the first HLC hit. The distribution for one of the variables used in the L3 filter is shown in figure~\ref{fig:l3-var-cleaned-full-time-length}. It is apparent from the distributions that there is a significant population of events in data with large values of the plotted variable that does not exist in simulation. These events are discarded, improving the agreement between data and simulation for events passing the L3 filter.
\begin{figure}
    \centering
    \includegraphics[width=7 cm]{figures/icecube/selection/IC2018_LE_L3_Vars_CleanedFullTimeLength.pdf}
    \caption{Distribution of one of the variables used in the L3 offline filter, the time between the last hit and the first hit after noise cleaning. Histograms show the distributions in simulated data separated by particle and interaction type, data points with error bars show the distribution of real data. The bottom panel shows the ratio between data and simulation. Events falling on the "signal" side of the histogram are passed to the next filter level.}
    \label{fig:l3-var-cleaned-full-time-length}
\end{figure}

\subsubsection{Level 4}
In the next level, L4, more advanced selections based on the output of Boosted Decision Trees (BDTs) are applied, with a separately trained BDT for noise and muon rejection, respectively. The output of each BDT is a probability score between zero (background-like) and one (signal-like).  The inputs into the BDT aimed at noise rejection consist of hit counts in cleaned hit series and variables that characterize the geometric and temporal spread of the observed hits, such as the full width half maximum (FWHM) of the hit times. The BDT is trained using simulated pure noise and neutrino events. Events are passing the L4 noise cut if the BDT score is above 0.7, which reduces the number of pure noise events by two orders of magnitude from 36.6~mHz to approximately 0.3~mHz. The BDT that is used to reject atmospheric muons also takes simple variables as its input that consist mostly of different veto hit counts and variables that characterize the distribution of z-coordinates of the observed hits as well as their radial distance with respect to the center of the DeepCore fiducial volume. In contrast to the noise BDT, however, the muon BDT is trained using real data and simulated neutrino events, with the goal of rejecting data events. This is possible because the data sample consists to 99\% of atmospheric muons at this stage of the event selection. Events are passing the L4 muon cut if the output score from the muon BDT is above 0.65, removing 94\% of all muon events while keeping 87\% of all neutrinos. The distributions of the output scores of both BDTs are shown in figure~\ref{fig:l4-bdt-output}.
\begin{figure*}
    \centering
    \includegraphics[width=7 cm]{figures/icecube/selection/L4_noiseBDT_L4_NoiseClassifier_ProbNu.pdf}
    \includegraphics[width=7 cm]{figures/icecube/selection/L4_muon_L4_MuonClassifier_Data_ProbNu.pdf}
    \caption{Distribution scores for the noise (left) and muon (right) BDT. The distributions of the muon classifier are shown for events where the score of the noise BDT is greater than 0.7.}
    \label{fig:l4-bdt-output}
\end{figure*}

\subsubsection{Level 5}
The final offline filter level that is applied before the event reconstruction step is L5. This filter searches specifically for hits occurring in un-instrumented \emph{corridors} within the IceCube array through which an atmospheric muon can sneak into the DeepCore volume while evading previous veto cuts. In addition, events with more than seven hits in the outermost strings of the IceCube array or that have a down-going pattern of hits in the uppermost region of the detector are vetoed to remove events containing atmospheric muons entering the detector coincidentally with neutrinos. The distribution for one of the corridor variables and one of the muon rejection variables are shown in figure~\ref{fig:l5-vars}. Table~\ref{tab:l5_summary} shows the rates of each event type expected at each level of the selection up to L5 together with the efficiency of the filter at the final level.
\begin{figure*}
    \centering
    \includegraphics[width=7 cm]{figures/icecube/selection/L5_contained_L5_WideCorridorCutCount.pdf}
    \includegraphics[width=7 cm]{figures/icecube/selection/SRTTWOfflinePulsesDC_ContainmentVars.z_travel_top15.png}
    \caption{Distributions for one of the L5 corridor cut variables (left) and one of the variables used to reject coincident muon events (right). The distribution in the right panel is shown only for events which have at least four hits in the uppermost 15 DOMs combined over all IceCube strings.}
    \label{fig:l5-vars}
\end{figure*}

\begin{table}
\begin{tabular}{lrrrrr}
Event type  & DC filter   & L3   & L4   & L5   & Eff. (\%) \\
\toprule
Atm. $\mu$         & 7273 & 505  & 28.1 & 0.93 & 0.012          \\
Pure noise         & 6621 & 36.6 & 0.28 & 0.07 & 0.001          \\
Atm. $\nu_e$ CC    & 1.61 & 0.95 & 0.84 & 0.48 & 29.8           \\
Atm. $\nu_\mu$ CC  & 6.16 & 3.77 & 3.11 & 1.39 & 22.5           \\
Atm. $\nu_\tau$ CC & 0.19 & 0.13 & 0.12 & 0.07 & 36.8           \\
Atm. $\nu$ NC      & 0.86 & 0.53 & 0.46 & 0.23 & 26.7  \\
\end{tabular}
\caption{Summary of the rates (in mHz) obtained after each level of selection. Neutrinos are weighted to an atmospheric spectrum with oscillations included.}
\label{tab:l5_summary}
\end{table}

\subsection{Event Reconstruction}
\label{sec:event-reconstruction}

After the L5 selection, the rate of muons is reduced enough such that the majority of the total sample is expected to consist of atmospheric neutrinos, and it is at this point that the event reconstruction and signature classification is run. For the measurement presented in this thesis, three reconstructed quantities are required: The zenith angle, the energy, and a proxy score that determines the flavor of a neutrino. As described in section \ref{sec:particle-signatures}, all neutrino events in DeepCore can be effectively approximated as either a cascade ($\nu_e$ CC events, all NC events, and 83\% of $\nu_\tau$ CC events) or a combination of a cascade at the neutrino interaction point with an out-going muon track ($\nu_\mu$ CC events and 17\% of $\nu_\tau$ CC events). The zenith angle can be most accurately reconstructed for track-like events due to their elongated, highly directional signature. For cascades, the reconstruction of the direction is more difficult because of their most compact and diffuse light distribution. The energy of a neutrino event is reconstructed by comparing the expected light output of a combined track and cascade hypothesis to the observed hits. Finally, the flavor proxy is calculated using variables that characterize the elongation of the observed hit signature  and the goodness-of-fit of a combined track and cascade hypothesis compared to that of a cascade-only hypothesis. The resulting score allows the separation of muon neutrino interactions from other interactions, which is ideally suitable to observe the muon neutrino disappearance oscillation channel.

\subsubsection{Zenith angle reconstruction}
The zenith angle is reconstructed using the Single-string Antares-inspired Analysis (\textsc{santa})\sidecite{Garza2014Measurement}. It is an older algorithm aimed at reconstructing the direction of muon tracks that has been originally developed for use in the ANTARES neutrino telescope~\sidecite{Aguilar:2011zz}. It has since been refurbished and improved in IceCube as described in detail in~\sidecite{lowen-reco-paper}.

The reconstructed pulse series in every DOM is summarized by the time of the first pulse and the sum of charges of all pulses. This time and charge is the only information used by the reconstruction and is referred to as a \emph{hit} in the following. The first step of the  reconstruction algorithm is a cleaning routine that removes hits produced
by photons that have been scattered many times as they traveled
through the ice, leaving only hits from photons that have travelled in approximately straight lines based on the time difference between hits on the same string.
%The algorithm is a simplification from an earlier implementation described in \cite{Garza2014Measurement}.
It calculates the signal speed between hits on the same string, and removes a hit if this velocity is below the speed of light in ice. This is a simplification from the algorithm described in \cite{Garza2014Measurement}, where the effective signal velocity was updated during the selection process. The selection is run separately for each string, and if fewer than three hits remain on a string, all hits on the string are discarded. In total, it is necessary that at least five hits remain in an event in order to run the directional reconstruction. If only hits on one string remain after the selection, the event is referred to as a \emph{single-string} event, otherwise it is a \emph{multi-string} event. The reconstruction is generally more accurate for multi-string events, because the spacing between strings provides a long lever arm to constrain the direction of a track.

The directional reconstruction itself is a modified $\chi^2$-regression with respect to the observed and predicted observation time with an additional regularization term involving the observed charge, where the expected arrival time for unscattered Cherenkov photons is calculated geometrically under the assumption of an infinitely long track without stochastic energy losses.

% \subsubsection{Hit selection}

% \label{sub:hit-selection}The first step of the \textsc{santa} reconstruction
% is the selection of minimally scattered photons from all of the observed
% pulses by removing those pulses within the trigger window that are likely to have undergone a significant amount of scattering.

% We combine the pulse series recorded by each activated DOM to a \emph{hit} with the time of the first pulse and the total charge of all pulses. All subsequent cleaning and reconstruction steps are applied to these combined hits. To remove scattered light, we make use of the fact that the largest possible delay between hits that are produced by \emph{unscattered} light on two different DOMs, $i$ and $j$, on the same string, is the time it takes for a directly up- or down-going light front to travel from one DOM to the other, $\tau_{ij}=|\Delta z_{ij}|/c_{\mathrm{ice}}$, where $|\Delta z_{ij}|$ is the distance between DOMs $i$ and $j$ and $c_{\mathrm{ice}}$ is the speed of light in ice. If the time delay between two hits, $\Delta t_{ij}$, is larger than the maximum delay, $\tau_{ij}$, then we know that the light must have undergone some amount of scattering. As a starting point for the hit selection, we choose the hit with the highest charge on each string, $i=0$, and first remove any earlier hit, $j$, where $-\Delta t_{0j} > \tau_{0j}$. From there, the algorithm iterates through every hit, $i$, and removes any other hit, $j$, where $\Delta t_{ij} > \tau_{ij}$. If fewer than 3 hits remain on a string, the entire string is removed from the event. If less than 5 hits remain in the event, it cannot be reconstructed. This is a simplified version of the cleaning procedure described in Ref.~\cite{Garza2014Measurement} and leaves more scattered light in the events. This is compensated for by the addition of the robust loss function (Sec.~\ref{sec:robust-losses}). In this configuration, we can reconstruct about 10\% more \numucc events than with the original implementation~from Ref.~\cite{Garza2014Measurement} at a similar resolution. In the example event fits in Figs.~\ref{fig:santa-single-string-example} \& \ref{fig:santa-multi-string-example}, the hits that are removed by the hit selection are crossed out.


% \subsubsection{Single-String vs. Multi-String}

% After the hit-cleaning procedure, passing events fall
% into two basic categories that are reconstructed differently. The first category is \emph{multi-string}
% events that contain observed charges in modules on two or more strings
% of the detector. Since a string is removed entirely from an event
% if it has less than three hits left after hit cleaning, a multi-string
% event contains at least six modules with recorded charges: three on
% one string and three on another string. In these events, we reconstruct both the
% zenith and the azimuth angles of the direction of a track. The second category is \emph{single-string} events that contain only one string in which modules have observed charges. Since all modules on a single string share approximately the same $x$ and $y$ coordinates, the azimuth angle of a track cannot be reconstructed. Example events for single-string and multi-string event fits are shown in Figs.~\ref{fig:santa-single-string-example} and \ref{fig:santa-multi-string-example}, respectively.

% \begin{figure}[h]
%     \centering
%     \includegraphics[width=\linewidth]{figures/santa/single_string_example_with_cleaning_id_29130693.pdf}
%     \caption{Example of a \numucc event reconstructed with \textsc{santa} on a single string. Circles show each hit, where the z-coordinate is the position of the DOM and the time is the time of the first observed pulse in that DOM.}
%     \label{fig:santa-single-string-example}
% \end{figure}

% \begin{figure*}[ht]
%     \centering
%     \includegraphics[width=\textwidth]{figures/santa/multi_string_example_with_cleaning_id_12607962.pdf}
%     \caption{Example of a \numucc event reconstructed with \textsc{santa} with hits on several strings. Strings 84, 83 and 37 are spaced $\sim80\,\mathrm{m}$ apart from each other and form a highly obtuse triangle.}
%     \label{fig:santa-multi-string-example}
% \end{figure*}

\subsubsection{Geometry of Tracks in IceCube}
\label{subsec:geometric-time-derivation}

To perform the $\chi^{2}$-fit on the observed hit times for a track hypothesis, we first need to derive the expected photon arrival time for an optical module at position $\vec{r}=(x,y,z)$ given the parameters of the hypothesis.

We characterize a track by a normalized direction vector $\vec{u}=(u_{x},u_{y},u_{z})$,
an anchor point $\vec{q}=(q_{x},q_{y},q_{z})$ and a time $t_{0}$
at which the particle passes through $\vec{q}$. In this simplified
hypothesis, tracks are modeled as being infinite in both directions;
there are no parameters to fix the start and end position and the
velocity is fixed to the vacuum speed of light, $c$. Since the reconstruction ignores DOMs that have not recorded any pulses, the fact that the true track length is finite only makes a negligible  difference.
Without scattering, all Cherenkov photons lie on a cone with an opening
angle $\theta_{c}$ (see Fig.~\ref{fig:Detailed-track-geometry})
whose tip is at the position of the particle at the time $\vec{p}(t)$. The opening angle satisfies $\cos(\theta_c)=1/n_{\mathrm{ph}}$, where $n_{\mathrm{ph}}$ is the phase index of refraction of the ice.

\begin{figure}[h]
\begin{centering}
\tikzsetnextfilename{track_geometry_santa}%
\begin{tikzpicture}[scale=1,>=stealth]
	\path[name path=track] (0,3) -- (9,3);
	\node[shape=star,
	      star point height=1cm,
	      star point ratio=0.5,
	      draw, fill=black,
	      label=below:$\vec{p}(t_{\mathrm{em}})$] (emission) at (1,3) {};
	\draw[->, decorate,
	decoration={snake,amplitude=.4mm,segment length=2mm,post length=1mm}]
		(emission.center)
		-- node[sloped, above] {$d_{\gamma}$} +(40:4)
		node[label=above:DOM at $\vec{r}$] (dompos) {};
	\path[name path=cone] (dompos.center) -- +(-50:4);
	\draw[name intersections={of=track and cone, by=tip}]
		(dompos.center) -- node[sloped, above] {Cherenkov light cone} (tip)
		node[label=below:$\vec{p}(t_{\mathrm{geom}})$] (muonpos) {};
	\draw[fill=black, opacity=0.5] (dompos.center) circle (5pt);
	\draw[color=black, ->, style=very thick] (0,3) node[anchor=north]{muon} -- (muonpos.center);
	\draw (emission.center) +(1,0) node[anchor=south east]{$\theta_c$}  arc (0:40:1);
	\path (emission.center)
		-- node[shape=circle,
			fill=black,
			label=below:$\vec{q}$] (vertex) {}
		(tip);
	\draw[->] (vertex.center) -- node[sloped, below] {$\vec{r}-\vec{q}$} (dompos.center);
	%\draw (vertex.center) +(-1,0) arc (180:140:1);
	%\path (vertex.center) -- +(160:0.6) node {$\theta$};
	\draw[->] (vertex.center) ++(0.2, 0.2) -- node[above] {$\vec{u}$} +(1,0);
\end{tikzpicture}\par
\end{centering}
\caption{\label{fig:Detailed-track-geometry}Detailed geometry of a light cone
created by a track. $\vec{q}$ is the position of the anchor point
and $\vec{r}$ is the position of the optical module. $\vec{p}(t_{\mathrm{em}})$
and $\vec{p}(t_{\mathrm{geom}})$ are the positions of the muon at
the time the photon is emitted and when it is geometrically expected
to arrive, respectively.}
\end{figure}

% We solve the geometric equations analogously to~Ref.~\cite{Garza2014Measurement} assuming that a photon emitted by the moving particle travels in a straight line at a velocity of $c$ divided by the group index of refraction $n_{\mathrm{gr}}$, which gives the \emph{geometric time}, $t_{\mathrm{geom}}$, as a function of the track parameters

% \begin{equation}
% t_{\mathrm{geom}}=t_{0}+\frac{1}{c}\left(\left(\vec{r}-\vec{q}\right)\cdot\vec{u}+\frac{d_{\gamma}}{n_{\mathrm{ph}}}\left(n_{\mathrm{ph}}n_{\mathrm{gr}}-1\right)\right)\label{eq:t_geom-MS-track}
% \end{equation}
% where the distance traveled by the photon $d_\gamma$ is
% \begin{equation}
% d_{\gamma}=n_{\mathrm{ph}}\sqrt{\frac{1}{n_{\mathrm{ph}}^{2}-1}\left(\vec{u}\times\left(\vec{r}-\vec{q}\right)\right)^{2}}\,.\label{eq:photon-distance-3d}
% \end{equation}

% The group and phase indices of refraction depend on the wavelength, but for this reconstruction we use as value for the wavelength $\lambda=400\;\mathrm{nm}$
% \footnote{$400\;\mathrm{nm}$ is near the wavelength of the highest acceptance of the optical modules.\cite{Aartsen:2016nxy}},
% , where $n_{\mathrm{gr}}=1.356$ and $n_{\mathrm{ph}}=1.319$~from Ref.~\cite{PRICE200197}.

% \subsubsection{Fitting Procedure}
% \label{sec:santa-loss}

% For a given set of parameters $\vec{\theta}=(\vec{u},\vec{q},t_0)$, we minimize a modified chi-square loss function given by
% \begin{equation}
% L(\vec{\theta})=\sum_{i=1}^{N}r^2_i
% +
% \frac{1}{\bar{q}}\sum_{i=1}^{N}\tilde{q}_i \frac{d_{\gamma,i}}{d_0}\,.\label{eq:chi-square-mod-loss}
% \end{equation}
% where $\bar{q}$ is the average of $\tilde{q}_i$, and $r^2_i$ is the chi-square residual for each observed hit, $i$, between the observed time, $t_{\mathrm{obs}, i}$ and the geometric arrival time, $t_{\mathrm{geom},i}(\vec{\theta})$,

% \begin{equation}
% r_{i}^{2}=\left(\frac{t_{\mathrm{geom},i}(\vec{\theta})-t_{\mathrm{obs},i}}{\sigma_{t}}\right)^{2}\,.
% \end{equation}

% The uncertainty on the pulse time measurement is approximately $\sigma_{t}=3\,\mathrm{ns}$, corresponding to the readout rate of the modules~\cite{Abbasi:2008aa}.

% The second term in eq.~\ref{eq:chi-square-mod-loss} is a regularization term that multiplies the distance traveled by a photon to the optical module that recorded it, $d_{\gamma,i}$, by the measured charge, $\tilde{q}_i$, to penalize solutions where a large charge is observed far away from the hypothesized track position. Because the modules are most sensitive on the side facing towards the bedrock, we correct the observed total charge in each DOM, $q_i$, for the sensitivity with
% \begin{equation}
% \tilde{q}_i=q_i\frac{2}{1+\cos(\vartheta_i)}\,,
% \end{equation}
% where $\vartheta_i$ is the angle between the direction of the photon
% and a vector pointing up to the surface of the ice.
% The parameter $d_{0}$ determines the relative contribution of the regularization term. Its value has been optimized for best average performance of the reconstruction and is fixed to $7\,\mathrm{m}$. 

% \subsubsection{Robust loss functions }
% \label{sec:robust-losses}
% After the hit selection described in Sec.~\ref{subsec:hit-selection}, a small number of hits from photons that have undergone significant amounts of scattering will remain that could strongly bias the fit result. We improve the robustness of the regression against such outliers by wrapping the squared residuals for each pulse in eq. \ref{eq:chi-square-mod-loss}, $r^{2}_i$, 
% with the Cauchy robust loss function
% \begin{equation}
% r_i^2 \rightarrow \phi(r_{i}^{2})=\log\left(1+r_{i}^{2}\right)\,.
% \label{eq:cauchy-loss}
% \end{equation}
% It reproduces the original $r^{2}$ residual
% for small values of $r$, but grows more slowly than $r^2$ for large values of $r$, so that outliers are effectively given less weight.

% Additionally, we choose the point of a “soft cut-off”,
% denoted herein as $C$, at which the residual diverges from the regular $r_{i}^{2}$
% in units of standard deviations by setting 
% \begin{equation}
% \phi(r_{i}^{2})\rightarrow\phi\left(\nicefrac{r_{i}^{2}}{C^{2}}\right)C^{2}\,.
% \end{equation}

% Figure~\ref{fig:robust-losses-example} shows the Cauchy loss function
% for different values of $C$.  The choice of the value of $C$ is a trade-off. If it is too large, then the fit can be strongly influenced by single outliers. If it is too small, then the fit ignores too many hits and falls into degenerate solutions. We found the optimal value to be $C=3$. At this setting, the fit can effectively ignore hits that are far away from the Cherenkov cone.

% % \begin{figure}[h]
% % % \begin{centering}
% % \centering
% % \includegraphics[width=8cm]{figures/santa/cauchy_loss_c.pdf}
% % % \par\end{centering}
% % \caption{\label{fig:robust-losses-example}The Cauchy robust loss function for different values of the scaling parameter, $C$. In a $\chi^{2}$
% % fit, the residual is the difference between the model and the observation
% % in units of standard deviations, $r=\frac{x-\mu}{\sigma}$ .}
% % \end{figure}
% As a further constraint on the regression, we apply the robust loss function from eq.~\ref{eq:cauchy-loss} only to pulses where the observed time is later than the expected photon time, since we expect that scattering would only cause photons to arrive too late, never too early. 

% \subsubsection{Resolution}

% The performance of the \textsc{santa} algorithm will be presented, together with the next algorithm discussed, in Sec.~\ref{sec:performance}.




\setchapterpreamble[u]{\margintoc}
\setchapterstyle{kao}
\chapter{Simulation and Data Processing}
\labch{data-sample}


\section{Event Simulation}
\label{sec:event-simulation}
The method by which all of the measurements presented in this thesis are performed is that of \emph{Monte-Carlo (MC) forward folding}.
In a nutshell, this method involves producing a large set of simulated signal and background events that are then re-weighted in such a way that their distribution matches that of the observed data events as closely as possible.
To give reliable results, an accurate simulation of all particle interactions described in Section~\ref{sec:particle-interactions} as well as the detector electronics described in section~\ref{sec:dom-daq} is required.
The simulated and observed events are then passed through the same data processing chain described in section~\ref{sec:data-processing}.
The resulting MC simulated dataset and the observed dataset are then histogrammed in the same binning, and the weights of the MC events are adjusted to give the best match between the histograms according to a loss function as defined in section~\ref{sec:test-statistic}.

The simulation chain for neutrinos and atmospheric muons can generally be divided into three steps that are described in this chapter:
\begin{enumerate}
    \item Simulation of particle interactions
    \item Photon propagation in ice
    \item Response of detector DAQ systems
\end{enumerate}
A special case is the simulation of detector noise, for which no particle production or photon propagation is necessary.

% \subsection{Particle Interactions}

% The first step for the simulation of neutrinos and muons is to sample parameters for the primary particle, and to simulate the secondary charged particles that are produced when it interacts inside the detector. The charged components of the secondary particles are then passed on to the photon propagation step described in section~\ref{sec:photon-propagation}.

\subsection{Neutrino Interactions}

Because of the inherently low interaction rate of neutrinos, it would be impractical to simulate a constant flux of neutrinos from any particular direction, the vast majority of which would simply pass through the detector without producing any signal at all.
Instead, every simulated neutrino is forced to interact within a given volume, and the event is given a weight corresponding to the inverse of the simulated fluence,
\begin{equation}
    w = \frac{1}{F_{\mathrm{sim}}} \frac{1}{N_{\mathrm{sim}}}\;.
\end{equation}
Here, $N_{\mathrm{sim}}$ is the number of simulated events and $F_{\mathrm{sim}}$ is the number of neutrino events per area, solid angle, energy, and time in the simulation.
This weight, when multiplied with the flux of a given physics model and a live time, gives the expected number of events that this simulated event corresponds to.
The baseline neutrino flux model used in this work is that proposed by Honda~\emph{et.~al}\sidecite{Honda:2015fha} that is specifically computed for the South Pole\sidenote{Variations on this flux model and how they are propagated into the analysis are described in \refsec{flux_systs}.}.

%For this analysis, the simulated interaction volume is a cylinder centered in DeepCore, with a length and radius chosen such that all events that have a chance of producing a signal in DeepCore should be contained in it. The neutrino directions are sampled isotropically in azimuth and zenith, implying that the simulated flux per solid angle is $\phi_\Omega = \frac{1}{4\pi}$. The simulated neutrino flux is a power law with $\phi_e \propto E^{-2}$. After sampling the zenith and azimuth for an event, a random position is sampled
Under the assumption that neutrino absorption is negligible and that the material consists of isoscalar targets, the simulated fluence is given by the chosen probability density in the direction and energy, $\phi_\Omega \times \phi_E$,  the size of the interaction volume, $V$, the cross-section of the interaction, $\sigma$, and the density of the material, $\rho$, by
\begin{equation}
    F_{\mathrm{sim}}^{-1} = V \times \rho \times N_A \times 1\frac{\mathrm{mol}}{\mathrm{g}} \times \sigma \times \frac{1}{\phi_\Omega} \times \frac{1}{\phi_E}\;,
\end{equation}
where $N_A$ is Avogadro's number. The volume in which neutrino interactions are simulated is a cylinder centered in DeepCore, with a height and radius chosen such that all events that have a chance of producing a signal in DeepCore should be contained in it, depending on the neutrino flavor and energy (see also table~\ref{table:GENIE}). Neutrino directions are isotropically distributed in zenith and azimuth, implying $\phi_\Omega = \frac{1}{4\pi}$. The neutrino energies are sampled from a power law with $\phi_e \propto E^{-2}$. The simulated live time corresponding to a single simulated event is  $T_{\mathrm{sim}} =  F_{\mathrm{sim}} / \Phi$, where $\Phi$ is the expected neutrino flux including neutrino oscillations at global best-fit parameters.
The amount of simulation generated for each neutrino flavor is chosen such that the total simulated live time is $>70$~years over the entire energy range.
Neutrinos and anti-neutrinos are produced in ratios of 70\% and 30\%, respectively.
The simulated live time as a function of energy is shown in \reffig{sim-livetime}.
The livetime for electron neutrinos increases with energy because the simulated spectrum is harder than the real spectrum.
The livetime for tau neutrinos is much higher than that of other flavors because the contribution of tau neutrinos to the expected neutrino flux is very small.

\begin{table}
\caption{Table of generation volumes used for \textsc{Genie} neutrino simulation. The cylinder is centered in DeepCore in all cases. \label{table:GENIE}}
\begin{center}
\begin{tabular}{ ccccc }
\textbf{Flavor} & \textbf{Energy (GeV)} & \textbf{Radius (m)} & \textbf{Length (m)}\\
\toprule
\multirow{4}{*}{$\nu_e+\bar{\nu_e}$}  & 1-4 & 250 & 500 \\
 & 4-12 & 250 & 500   \\
 & 12-100 & 350 & 600  \\
 & 100-10000 & 550 & 1000  \\
 \midrule
\multirow{4}{*}{$\nu_{\mu}+\bar{\nu_{\mu}}$} & 1-5 & 250 & 500\\
 & 5-80 & 400 & 900\\
 & 80-1000 & 450 & 1500\\
 & 1000-10000 & 550 & 1500\\
 \midrule
\multirow{5}{*}{$\nu_{\tau}+\bar{\nu_{\tau}}$} & 1-4 & 250 & 500\\
 & 4-10 & 250 & 500\\
 & 10-50 & 350 & 600\\
 & 50-1000 & 450 & 800\\
 & 1000-10000 & 550 & 1500\\
 \bottomrule
\end{tabular}
\end{center}
\end{table}

\begin{figure}
    \centering
    
\tikzsetnextfilename{mc_livetime}%
\begin{tikzpicture}

\pgfplotstableread{figures/icecube/selection/livetime/livetime_hists.csv}\table

\begin{loglogaxis}[
    width=0.7\linewidth,
    height=0.5\linewidth,
    tick align=outside,
    tick pos=left,
    xmin=1, xmax=10000,
    xmajorgrids,
    ymajorgrids,
    xlabel=energy (GeV),
    ylabel=total MC livetime (years),
    ymin=20, ymax=80000,
    legend style={
      at={(0.95,0.95)},
      anchor=north east,
    },
]
% livetimes in the table are months per file
% number of files taken from the nominal MC only
\addplot[const plot, black, thick] table[x=energy, y expr=613 * \thisrow{genie_120000} / 12] from \table;
\addlegendentry{\(\nu_e\)}
\addplot[const plot, orange, thick] table[x=energy, y expr=1519 * \thisrow{genie_140000} / 12]  from \table;
\addlegendentry{\(\nu_\mu\)}
\addplot[const plot, skyblue, thick] table[x=energy, y expr=340 * \thisrow{genie_160000} / 12]  from \table;
\addlegendentry{\(\nu_\tau\)}
\end{loglogaxis}

\end{tikzpicture}

    \caption{Simulated MC livetime as a function of energy, calculated using the HKKM\cite{Honda:2015fha} model flux with \textsc{NuFit}~2.2\cite{nufit22} oscillation parameters.}
    \label{fig:sim-livetime}
\end{figure}

After sampling the parameters of the primary neutrino, the \textsc{Genie}\sidecite{Andreopoulos:2015wxa} software is used to simulate its interaction with the ice and the production of secondary particles and to calculate the cross-section of the interaction.
The propagation and Cherenkov light production of any muon that is produced in these interactions is simulated with \textsc{Proposal}\sidecite{proposal}.
The light output of secondary electrons, positrons, and gamma rays above 100~MeV, and that of hadronic showers above 30~GeV, are generated using analytic approximations from \cite{RADEL2013102} as described in sections \ref{sec:em-showers} and \ref{sec:had-showers}.
At lower energies, the full \textsc{Geant4} simulation of the shower development is run to produce Cherenkov photons.

\subsubsection{Cross-section uncertainties}
\label{sec:xsec_systs}
Two systematic parameters are included to account for uncertainties in the form factors of charged-current quasi-elastic ($M_{A}^{CCQE}$) events and charged-current resonant ($M_{A}^{CCRES}$) events. Both these form factors have a dependency on the momentum transfer, $Q^2$, of the form:\\

\begin{equation}
    F(Q^{2}) \propto \frac{1}{(1-(Q^{2}/M_{A}^{2})^{2}}
\end{equation}

Where $M_{A}$ is called the \textit{axial mass}, and can be measured experimentally.
The differential cross-section of each event is computed with \textsc{GENIE} at five discrete points, that is, the nominal mass and  -2$\sigma$,-1$\sigma$,1$\sigma$ and 2$\sigma$ away from the nominal mass, where $\sigma$ is a fractional uncertainty of 20\%.
This uncertainty approximates the recommendation of the GENIE collaboration, which suggests an asymmetric error of -15\% and +25\% for $M_A^{CCQE}$ and a symmetric error of $\pm20\%$ for $M_A^{CCRES}$\cite{Andreopoulos:2015wxa}.
In order to apply a continuous variation of that systematic parameter over the course of a minimization, a quadratic function is fit to interpolate between these discrete points.
\reffig{resonant_mass} shows the \textsc{GENIE} weights of a handful of $\nu_{e}$ CC events from resonance production, across the allowed range of axial masses, along with their fitted quadratic dependence.
The upper panel of \reffig{template_xsecsyst} illustrates an example of the varying $M_{A}^{RES}$ on the final level sample.

\begin{figure}
    \centering
    \tikzsetnextfilename{genie_sys_res}%
\begin{tikzpicture}

%% List of nue CC RES  events

\begin{axis}[
        xlabel=$\Delta M_{\mathrm{A}}^{\mathrm{CCRES}} / \sigma$,
        ylabel=\textsc{GENIE} weight,
        xmajorgrids, ymajorgrids,
        ymin=0.18, ymax=1.58,
        height=0.6\linewidth,
        width=0.8\linewidth,
        legend columns=2,
        legend style={mark=*, at={(0.05,0.95)}, anchor=north west}
    ]
\addplot[orange, domain=-2.2:2.2] {0.3504939715034138 * (1 + 0.11131609002187497 * x + -0.022932572545202236 * x^2};
\addlegendentry{event \#1}
\addplot[orange, only marks, forget plot] coordinates {
(-2, 0.2390467380668949)
(-1, 0.3066640473685763)
(0, 0.3504939715034138)
(1, 0.378941696047475)
(2, 0.397986006147438)
};

\addplot[skyblue, domain=-2.2:2.2] {0.7582274008087017 * (1 + 0.051334614671670824 * x + -0.013918323402408417 * x^2};
\addlegendentry{event \#2}
\addplot[skyblue, only marks, forget plot] coordinates {
(-2, 0.6361056338725656)
(-1, 0.716261703894783)
(0, 0.7582274008087017)
(1, 0.7824732034456762)
(2, 0.7976164413672049)
};

\addplot[bluishgreen, domain=-2.2:2.2] {0.8814816065644564 * (1 + 0.03273351968623338 * x + -0.009338762415529337 * x^2};
\addlegendentry{event \#3}
\addplot[bluishgreen, only marks, forget plot] coordinates {
(-2, 0.789189629008826)
(-1, 0.8507716696219192)
(0, 0.8814816065644564)
(1, 0.8987912601423985)
(2, 0.9094498113562377)
};

\addplot[yellow, domain=-2.2:2.2] {0.9483635232904346 * (1 + 0.26999687193847616 * x + 0.00565034846932953 * x^2};
\addlegendentry{event \#4}
\addplot[yellow, only marks, forget plot] coordinates {
(-2, 0.4643360012427789)
(-1, 0.6781015441978948)
(0, 0.9483635232904346)
(1, 1.2230743122646268)
(2, 1.4721255409542606)
};

\addplot[blue, domain=-2.2:2.2] {0.6341971566788007 * (1 + 0.22525771528791444 * x + -0.019807510312081195 * x^2};
\addlegendentry{event \#5}
\addplot[blue, only marks, forget plot] coordinates {
(-2, 0.29866260486349816)
(-1, 0.47226711305160485)
(0, 0.6341971566788007)
(1, 0.7653501328716419)
(2, 0.8664101077312704)
};

\end{axis}

\end{tikzpicture}

    \caption{\textsc{GENIE} interaction weights as a function of the pull of the axial mass term $M_{A}^{\mathrm{CCRES}}$, for five $\nu_{e}$ CC events produced via resonance interactions. Each dot represents a discrete point for which the event's cross section is computed in \textsc{GENIE}. The  line represents the quadratic fit made used to interpolate the weight value over the continuous range allowed for the systematic parameter.}
    \label{fig:resonant_mass}
\end{figure}

\begin{figure}[!t]
    \centering
    \begin{subfigure}[t]{0.9\textwidth}
        \centering
        \includegraphics[width=0.99\textwidth,trim={0 0 0 0.6cm},clip]{figures/measurement/systematics/xsec/Genie_Ma_RES.pdf}
        \caption{GENIE $M_{A}^\mathrm{RES}$}
    \end{subfigure}
    \begin{subfigure}[t]{0.9\textwidth}
        \centering
        \includegraphics[width=0.99\textwidth,trim={0 0 0 0.6cm},clip]{figures/measurement/systematics/xsec/dis_csms.pdf}
        \caption{DIS CSMS}
    \end{subfigure}
  \caption{Fractional difference in event rates between (top )$M_{A}^\mathrm{RES}$ (bottom) dis$\_$csms at 1$\sigma$ and at nominal value for both PID bins.
  \label{fig:template_xsecsyst}}
\end{figure}

The uncertainty on the DIS cross-section is primarily given by the disagreement in DIS calculation between CSMS\cite{csms-xsec} and GENIE\cite{Andreopoulos:2015wxa} cross-sections at energies above 100~GeV.
This analysis includes a parameter that interpolates between these two calculations with a linear extrapolation to energies below 100~GeV.
The bottom panel of \reffig{template_xsecsyst} illustrates an example of the varying this parameter, DIS, on the final level sample.
As expected, the impact of the parameter is largest in the highest energy bins.
There is an additional uncertainty of 20\% on the normalization of NC events to account for uncertainties of the hadronization process and the Weinberg angle in line with previous oscillation studies\cite{Aartsen_2018}.

\subsection{Atmospheric muons}
The offline filter steps described in section~\ref{sec:offline-filter} decrease the rate of atmospheric muons by several orders of magnitude as events pass through each of its stages.
This makes it challenging to produce a sufficiently large amount of simulated muon events to accurately estimate the expected background at the final level.
To overcome this challenge, two separate muon simulation sets are produced, one of which is used to tune the lower level (up to L4) offline filters and the other is used to estimate muon background at levels L5 and above.

For both sets, atmospheric muons are generated on the surface of a cylinder encompassing the entire IceCube detector with a radius of 800~m and a height of 1600~m.
Positions and directions of muons interacting in the detector are sampled using parametrized tables based on the approach described in \sidecite{BECHERINI20061}.
These tables are tuned to approximate the output of a detailed \textsc{CORSIKA}\sidecite{Heck1998CORSIKAAM} simulation of cosmic ray interactions and subsequent shower production using the cosmic ray flux model described in \sidecite{Gaisser:2011klf} and the \textsc{SIBYLL 2.1}\sidecite{sibyll} hadronic interaction model.
This flux is also used to weight simulated muon events and is distinct from the flux model used to weight neutrino events.
For the simulations used to tune the lower selection levels, the muon energy is sampled from a power law with a spectral index of -3 and all events are accepted to cover the entire IceCube array.
To produce the simulation that is used starting at the L5 trigger level, muons are only accepted if they intersect an inner cylinder centered in the DeepCore fiducial volume with a radius of 180~m and a height of 400~m.
Furthermore, muons are rejected based on a KDE estimate of the muon density in energy and zenith angle at the L5 filter level.
In this way, the sampling preferably produces such muon events that have a higher chance of passing the offline filtering up to L5, which greatly improves the efficiency of the simulation production.

After the position, direction and energy for a muon has been sampled, its propagation and photon production is simulated using \textsc{PROPOSAL} in just the same way as any muon that is produced in neutrino interaction would be.

\subsubsection{Muon Uncertainty}
\label{sec:atm-muons-systematic}
Because the muon background contamination is cut to only $\sim$2\% at the final level of the event selection (see \refsec{final-sample-binning}), the impact of muon systematic uncertainties is generally small.
Only the over-all scale is left as a free parameter in the analysis, its impact is shown in \reffig{weight-scale-syst}.
This scale also largely absorbs the effects of DOM efficiency uncertainties, since, to first order, an increase in DOM efficiency leads to a better muon rejection.
The spectral index of the muon flux has a very small effect far below the percent-level as shown in \reffig{delta-gamma-mu-syst} and is therefore not accounted for in the analysis.

\begin{figure}
    \centering
    \includegraphics[width=0.7\textwidth,trim={0 0 0 0.6cm},clip]{figures/measurement/systematics/muons/weight_scale.pdf}
    \caption{Impact on the final histograms when the muon normalization is increased by 50\%. The largest impact is seen above the horizon in the mixed PID channel with a change in bin count of 5\%.}
    \label{fig:weight-scale-syst}
\end{figure}

\begin{figure}
    \centering
    \includegraphics[width=0.7\textwidth,trim={0 0 0 0.6cm},clip]{figures/measurement/systematics/muons/delta_gamma_mu.pdf}
    \caption{Impact on the final histograms when the muon spectral index is increased by $1\sigma$.}
    \label{fig:delta-gamma-mu-syst}
\end{figure}

\subsection{Photon Propagation}
\label{sec:photon-propagation}

Photons are individually traced through the ice using the GPU-accelerated \textsc{clsim}\cite{clsim} package, which is an \textsc{OpenCL} re-implementation of the Photon-Propagation Code\sidecite{ppc}.
 The ice is modeled as 10~m thick layers with individual scattering and absorption lengths that are shown in \reffig{spice-model}.
The ice model used for the simulation in this work, also referred to as \emph{South Pole ICE (SPICE)}\sidecite{flasher_calibration}, incorporates the fact that the ice layers are slightly tilted with respect to the vertical axis, and that scattering and absorption strengths are not uniform as a function of azimuth.
For every photon, \textsc{clsim} first samples the absorption length from an exponential distribution whose expectation value is the absorption length of the current layer.
It then propagates all photons in parallel steps, where every step corresponds to one scattering event and the step length is sampled from an exponential distribution where the expectation value is the scattering length of the current layer.
The scattering angle is then sampled from a mixture of a Henyey-Greenstein distribution and a simplified Mie scattering distribution, where the shape parameters of these distributions have previously been calibrated using the in-situ LED calibration system\cite{flasher_calibration}.
Each photon stops when it has either reached its total absorption length or if it has intersected a DOM.
After all photons have either been absorbed or reached a sensor, the simulations stops and passes the photons that reached a sensor on to the next step simulating the detector response.

\begin{figure}
    \centering
    %\includegraphics[width=0.9\linewidth]{figures/icecube/ice/Spice3.2.1_layered_scatt_abs_withlength_annotated.png}
    \tikzsetnextfilename{spice_model}%
\begin{tikzpicture}
% let both axes use the same layers
\pgfplotsset{set layers}

\pgfplotstableread{figures/icecube/ice/spice_model/spice_3.2.1/icemodel.dat}\table

\begin{axis}[
    scale only axis,
    width=0.7\linewidth,
    height=0.5\linewidth,
    xmin=1100,xmax=2900,
    xticklabel style={/pgf/number format/.cd,1000 sep={}},
    axis y line*=left, % the '*' avoids arrow heads
    ymin=0,
    enlarge y limits=true,
    xlabel=depth (m),
    ylabel=Scattering length (m),
]
    \addplot[black, thick] table [x index=0, y expr=1 / \thisrowno{1}] \table;
    
    % dust layer
    \draw [name path=dust layer top, gray, thin] (2000, \pgfkeysvalueof{/pgfplots/ymin}) -- (2000, \pgfkeysvalueof{/pgfplots/ymax}); 
    \draw [name path=dust layer bottom, gray, thin] (2100, \pgfkeysvalueof{/pgfplots/ymin}) -- node[near end, sloped, above, black, font=\footnotesize\sffamily] {dust layer} (2100, \pgfkeysvalueof{/pgfplots/ymax});
    \addplot [gray, opacity=0.4] fill between [of=dust layer top and dust layer bottom];
    
    % IceCube
    \draw [name path=icecube top, gray, thin] (1450, \pgfkeysvalueof{/pgfplots/ymin}) -- (1450, \pgfkeysvalueof{/pgfplots/ymax}); 
    \draw [name path=icecube bottom, gray, thin] (2000, \pgfkeysvalueof{/pgfplots/ymin}) -- (2000, \pgfkeysvalueof{/pgfplots/ymax});
    \node[anchor=south, black, font=\footnotesize\sffamily] at (1750, 110) {IceCube\strut};
    \addplot [gray, opacity=0.2] fill between [of=icecube top and icecube bottom];
    
    % DeepCore
    \draw [name path=deepcore top, gray, thin] (2100, \pgfkeysvalueof{/pgfplots/ymin}) -- (2100, \pgfkeysvalueof{/pgfplots/ymax}); 
    \draw [name path=deepcore bottom, gray, thin] (2450, \pgfkeysvalueof{/pgfplots/ymin}) -- (2450, \pgfkeysvalueof{/pgfplots/ymax});
    \node[anchor=south, black, font=\footnotesize\sffamily] at (2270, 110) {DeepCore\strut};
    \addplot [gray, opacity=0.1] fill between [of=deepcore top and deepcore bottom];
    
\end{axis}

\begin{axis}[
    orange,
    scale only axis,
    width=0.7\linewidth,
    height=0.5\linewidth,
    xmin=1100,xmax=2900,
    axis y line*=right,
    axis x line=none,
    ymin=0,
    enlarge y limits=true,
    ylabel=Absorption length (m),
]
    \addplot[orange, thick] table [x index=0, y expr=1 / \thisrowno{2}] \table;

\end{axis}

\end{tikzpicture}

    \caption{Scattering and absorption lengths as a function of depth in the South Pole Ice (SPICE) model that is used to produce the simulation for this work.}
    \label{fig:spice-model}
\end{figure}


\subsection{Simulation of Detector Response}
\label{sec:sim-detector-response}
After the photons have reached the surface of the optical sensors, the simulation determines for each one if it is converted into a Monte-Carlo photo-electron (MCPE).
The probability that this occurs depends on the wavelength-dependent sensitivity of the DOM, as well as the angular acceptance.
The angular acceptance not only depends on the geometry of the DOM itself, but also incorporates the effect of the re-frozen column of ice at the center of each bore hole.
If a photon is accepted and converted into an MCPE, the next step is to simulate how much charge would be measured by the PMT inside the DOM as a response.
The charge is drawn from a combination of a normal distribution and two exponential distributions whose parameters have been calibrated \emph{in-situ} to match the observed charge distribution in each individual DOM\sidecite{ic_spe_20}.
This distribution, also referred to as the Single Photo-Electron (SPE) template, is shown in \reffig{spe-templates}.
The MCPEs with the samples charge are then converted into simulated waveforms for the ATWD and fADC readouts which are then passed into the data processing chain starting from the \emph{wavedeform} algorithm described in Section~\ref{sec:dom-daq}.
From there, the simulated events pass through all the same trigger and filter steps that are described in Section~\ref{sec:data-processing}.

\begin{figure}
    \centering
    \includegraphics[width=0.8\linewidth]{figures/icecube/detector_response/SPE_TA003_2.pdf}
    \caption{The green (yellow) regions show the 68\% (90\%) spread in the SPE charge templates for a given charge.  Superimposed are the average SPE charge templates for the variety of hardware configurations shown in the black dotted, dashed, and solid lines. The TA0003 distribution, shown in red, originates from laboratory measurements. Figure taken from \cite{ic_spe_20}.}
    \label{fig:spe-templates}
\end{figure}

\subsubsection{Detector Noise}

\begin{margintable}
\caption{\label{tab:vuvuzela_params} Parameters used in the noise simulation. Typical values taken from \cite{Michael_Larson_masters}, actual values are fit for each DOM individually.}
    \begin{tabular}{lc}\toprule
        \textbf{Parameter} & \textbf{Typical value} \\ \midrule
        Therm. rate &  $\lambda_\mathrm{th}\approx \SI{20}{\hertz}$ \\
        Decay rate &  $\lambda_\mathrm{dec}\approx \SI{250}{\hertz}$ \\
        Decay hits &  $\eta\approx 8$ \\
        Decay $\mu$ &  $\log_{10}(\frac{\mu}{\si{\nano\second}}) \approx -6$\\
        Decay $\sigma$ &  $\log_{10}(\frac{\sigma}{\si{\nano\second}}) \approx 2.7$ \\ \bottomrule
    \end{tabular}
\end{margintable}
In addition to Cherenkov photons induced by relativistic charged particles in the ice, IceCube detects photons from radioactive decays inside the glass housing of the DOMs and PMTs that are simulated using the \emph{Vuvuzela} module\sidecite{Michael_Larson_masters}\sidecite{Michael_Larson_phd}.
These "noise" MCPEs are simulated parametrically by sampling their times from distributions that take both thermal and non-thermal noise components into account. The thermal component comes from uncorrelated photons and PMT dark noise and is modeled as a Poisson process with a constant rate.
The non-thermal component comes from correlated bursts of photons that are produced by radioactive decays.
To simulate it, decay times are first drawn from a Poisson process with a constant rate, and the number of photons produced in each decay is sampled from a Poisson distribution.
The time differences between the non-thermal MCPEs produced by each decay are then sampled from a Log-Gaussian distribution.
This simulation method has five free parameters listed in \reftab{vuvuzela_params} that are calibrated \emph{in-situ} for every DOM.
All thermal and non-thermal MCPEs are injected into each simulated event together with the MCPEs from photons and passed into the rest of the simulation chain.
% \begin{margintable}
% \caption{\label{tab:vuvuzela_params} Parameters used in the noise simulation. }
%     \begin{tabular}{lcc}\toprule
%         \textbf{Parameter} & \textbf{Designation} & \textbf{Unit} \\ \midrule
%         Thermal rate & $\lambda_{Th}$ & $s^{-1}$ \\
%         Decay rate & $\lambda_{Decay}$ & $s^{-1}$ \\
%         Scintillation hits & $\eta_{Scint}$ & hits \\
%         Scintillation mean & $\mu_{Scint}$ & $\log_{10} (ns) $\\
%         Scintillation sigma & $\sigma_{Scint}$ &  $\log_{10} (ns) $ \\ \bottomrule
%     \end{tabular}
% \end{margintable}




\section{Data Processing}
\label{sec:data-processing}
\todo{break up differently. First simulation, then filtering, then uncertainties}
\subsection{Trigger}
As described in section \ref{sec:dom-daq}, the high-frequency ATWD waveform digitization in each DOM is triggered when it and its adjacent or next-to-adjacent neighbors on the same string record a voltage corresponding to at least 0.25 PE-equivalent within a $\pm$1~$\mu$s time window, which is referred to as the Hard Local Coincidence (HLC) condition.
Data acquisition for DeepCore is triggered when this condition is fulfilled for at least three DOMs inside the DeepCore fiducial volume within a $\pm$2.5~$\mu$s window.
If this condition is met, the waveforms for all DOMs that have observed voltages of at least 0.25~PE within a $\pm$10~$\mu$s time window centered around the trigger time are recorded.
This trigger is referred to as the "SMT3" trigger and is distinct from the so-called "SMT8" trigger that is used to activate the data acquisition of the larger IceCube array, which requires eight DOMs to fulfill the HLC condition in a $\pm$5~$\mu$s window.
A DOM that has recorded PEs within the readout window but for which the HLC condition has not been met is said to fulfill the \emph{Soft Local Coincidence} (SLC) condition.
The DeepCore SMT3 trigger rate is less than 10~Hz while accepting $\sim$70\% of $\nu_\mu$ events at 10~GeV and >90\% of $\nu_\mu$ events at 100~GeV\cite{DeepCore}.
The trigger efficiency for atmospheric muon neutrinos as a function of the primary neutrino energy is shown in \reffig{trigger-efficiency} for several different triggers that are used in IceCube.
\begin{figure}
    \centering
    \includegraphics[width=0.7\linewidth]{figures/icecube/selection/trigger/trigger_efficiency.jpg}
    \caption{Efficiency of the IceCube and DeepCore triggers as a function of the primary neutrino energy. Figure taken from \cite{DeepCore}.\label{fig:trigger-efficiency}}
\end{figure}

\subsection{Online Filter}
\begin{marginfigure}
    \includegraphics[width=\textwidth]{figures/icecube/eventviews/FilterDiagram.pdf}
    \caption{Example of an event that would be rejected by the online filter algorithm. DOMs that have observed light are highlighted in color depending on time from red (early hits) to blue (late hits). DOMs that have not observed any light are shown as black dots. Figure taken from \cite{DeepCore}.}
    \label{fig:online-filter-event}
\end{marginfigure}
Once the trigger condition is met, the recorded waveforms within the trigger window are converted into reconstructed pulses as described in \refsec{dom-daq} and are then passed into a set of \emph{online} filters (i.e. filters running on hardware at the Pole).
These filters are each designed to select events that are relevant to different physics measurements that are performed within the IceCube collaboration.
For the purposes of the analysis presented in this thesis, events are selected using the \emph{DeepCore filter}\sidecite{DeepCore}.
This filter is designed to select events that start inside the DeepCore fiducial volume and to reject those that are consistent with muons entering the detector from the outside.
The filter splits the observed series of hits between those hits that fall within the DeepCore fiducial volume and those outside of it.
It then estimates the "center of gravity" (COG) in space and time of the hits inside the fiducial volume and then calculates the velocity that a signal would have to travel from each hit occurring outside the fiducial volume to coincide with the COG.
If this velocity is close to the speed of light (between $0.25\;\mathrm{ns/s}$ and $0.4\;\mathrm{ns/s}$) for at least one hit, the event is rejected because it is consistent with a muon traveling through the veto region and entering DeepCore.
\reffig{online-filter-event} shows an example of an event that would be rejected by the online filter.
Only events that pass the trigger and filter conditions are sent north via satellite for further \emph{offline} filtering.

\subsection{Offline Filter}
\label{sec:offline-filter}
The offline filter is separated into subsequently applied \emph{levels}, referred to as L3, L4 and L5, where each level reduces the amount of background (atmospheric muons and noise) by approximately an order of magnitude while keeping most of the DeepCore starting events that are the target of the selection.

\subsubsection{Level 3}
At the lowest offline filter level, L3, cuts are applied to simple variables that remove the most easily identifiable background events while using only few computational resources.
The types of relevant types background events at this level of the event selection are pure noise, atmospheric muons and events with several coincident muons.

If a muon enters the detector after the data acquisition has already been triggered, it will create a series of pulses that extends much longer in time than what would be expected from a single particle interaction.
Because the MC simulation only simulates single particles, however, these events cause a significant disagreement between data and MC.
The time length of the observed hit series after noise cleaning is shown in the left panel of \reffig{l3-cut-vars}, where this disagreement at large times is apparent.
A cut at \SI{5000}{\nano\second}, also shown in the figure, removes such events.

To identify noise events, the observed series of hits is first passed into a cleaning algorithm that uses time window coincident conditions between DOMs to remove hits that are likely to originate from pure noise.
Only when at least six hits remain in the series after this cleaning procedure, the event is kept.
Another algorithm checks whether the observed hits show any sign of directionality and only accepts the event if that is the case.
Finally, an event should have more than two hits within a \SI{300}{\nano\second} sliding window.

The cuts aimed at removing muons consist of conditions on the number of hits in the veto region and conditions on the vertical position of the first HLC hit.
One of these variables is the z-position of the first hit DOM for which the HLC condition was fulfilled and its distribution is shown in the right panel of \reffig{l3-cut-vars}
A cut at \SI{-120}{\metre} from the origin of the IceCube coordinate system (corresponding to a depth of \SI{2068}{\metre} from the surface\sidenote{The origin of the IceCube coordinate system is at a depth of \SI{1948.07}{\metre} from the surface with the z-axis oriented upwards. The depth of a given z-coordinate is therefore $d = \SI{1948.07}{\metre} - z$\cite{icecube_detector_17}.}) removes events that are likely to originate from atmospheric muons since they begin above the fiducial volume of DeepCore.
The overall event rate after all L3 cuts have been applied is below \SI{1}{\hertz}.

% \begin{figure}
%     \centering
%     \begin{tikzpicture}

    \definecolor{black}{RGB}{0,0,0}
    \definecolor{orange}{RGB}{230, 159, 0}
    \definecolor{skyblue}{RGB}{86, 180, 233}
    \definecolor{bluishgreen}{RGB}{0, 158, 115}
    \definecolor{yellow}{RGB}{240, 228, 66}
    \definecolor{blue}{RGB}{0, 114, 178}
    \definecolor{vermilion}{RGB}{213, 94, 0}
    \definecolor{reddishpurple}{RGB}{204, 121, 167}

    \definecolor{lightgray204}{RGB}{204,204,204}


    \pgfplotstableread{figures/icecube/selection/Level3/DCFiducialHits_level3_data_mc_hists.csv}\table
    \begin{groupplot}[
        xmajorgrids, ymajorgrids,
        width=\linewidth,
        group/.cd,
        group size=1 by 2,
        xticklabels at=edge bottom,
        vertical sep=10pt
    ]
    \nextgroupplot[
        height=0.6\linewidth,
        legend cell align={left},
        legend columns=4,
        legend style={
          fill opacity=0.8,
          draw opacity=1,
          text opacity=1,
          at={(0.5,0.91)},
          anchor=north,
          draw=lightgray204
        },
        ymode=log,
        ymin=0.0001, ymax=1000.0,
        ylabel={rate (Hz)},
    ]
    \addplot[const plot, bluishgreen, thick] table [x=bin_edges, y=nuex100] from \table;
    \addlegendentry{$\nu_e + \bar{\nu}_e$ x100}
    \addplot[const plot, vermilion, thick] table [x=bin_edges, y=numux100] from \table;
    \addlegendentry{$\nu_\mu + \bar{\nu}_\mu$ x100}
    \addplot[const plot, yellow, thick] table [x=bin_edges, y=nutaux100] from \table;
    \addlegendentry{$\nu_\tau + \bar{\nu}_\tau$ x100}
    \addplot[const plot, orange, thick] table [x=bin_edges, y=muon] from \table;
    \addlegendentry{atm. muons}
    \addplot[const plot, skyblue, thick] table [x=bin_edges, y=noise] from \table;
    \addlegendentry{noise}
    \addplot[const plot, black, thick] table [x=bin_edges, y=total_mc] from \table;
    \addlegendentry{total MC}
    \addplot+[
        mark=*,
        mark options={scale=0.5, fill=black},
        black,
        only marks,
        error bars/.cd,
        x dir=none,
        y dir=both,
        y explicit
    ] table [x=bin_midpoints, y=data, y error=data__err]  from \table;
    \addlegendentry{data (2014)}


    \addplot[const plot, bluishgreen, thin, name path=nuex100_lo] table[x=bin_edges, y=nuex100__err_dn] from \table;
    \addplot[const plot, bluishgreen, thin, name path=nuex100_hi] table[x=bin_edges, y=nuex100__err_up] from \table;
    \addplot[bluishgreen,opacity=0.5] fill between[of = nuex100_lo and nuex100_hi];


    \addplot[const plot, vermilion, thin, name path=numux100_lo] table[x=bin_edges, y=numux100__err_dn] from \table;
    \addplot[const plot, vermilion, thin, name path=numux100_hi] table[x=bin_edges, y=numux100__err_up] from \table;
    \addplot[vermilion,opacity=0.5] fill between[of = numux100_lo and numux100_hi];


    \addplot[const plot, yellow, thin, name path=nutaux100_lo] table[x=bin_edges, y=nutaux100__err_dn] from \table;
    \addplot[const plot, yellow, thin, name path=nutaux100_hi] table[x=bin_edges, y=nutaux100__err_up] from \table;
    \addplot[yellow,opacity=0.5] fill between[of = nutaux100_lo and nutaux100_hi];


    \addplot[const plot, orange, thin, name path=muon_lo] table[x=bin_edges, y=muon__err_dn] from \table;
    \addplot[const plot, orange, thin, name path=muon_hi] table[x=bin_edges, y=muon__err_up] from \table;
    \addplot[orange,opacity=0.5] fill between[of = muon_lo and muon_hi];


    \addplot[const plot, skyblue, thin, name path=noise_lo] table[x=bin_edges, y=noise__err_dn] from \table;
    \addplot[const plot, skyblue, thin, name path=noise_hi] table[x=bin_edges, y=noise__err_up] from \table;
    \addplot[skyblue,opacity=0.5] fill between[of = noise_lo and noise_hi];


    \addplot[const plot, black, thin, name path=total_mc_lo] table[x=bin_edges, y=total_mc__err_dn] from \table;
    \addplot[const plot, black, thin, name path=total_mc_hi] table[x=bin_edges, y=total_mc__err_up] from \table;
    \addplot[black,opacity=0.5] fill between[of = total_mc_lo and total_mc_hi];


    \nextgroupplot[
        height=0.3\linewidth,
        ymin=0.5, ymax=2,
        ylabel={data/MC ratio},
    ]
    \addplot[const plot, black, thick] table[x=bin_edges, y=data_mc_ratio] from \table;


    \addplot[const plot, black, thin, name path=data_mc_ratio_lo] table[x=bin_edges, y=data_mc_ratio__err_dn] from \table;
    \addplot[const plot, black, thin, name path=data_mc_ratio_hi] table[x=bin_edges, y=data_mc_ratio__err_up] from \table;
    \addplot[black,opacity=0.5] fill between[of = data_mc_ratio_lo and data_mc_ratio_hi];
    \end{groupplot}
\end{tikzpicture}


%     \caption{DCFiducialHits}
%     \label{fig:l3-dc-fiducial-hits}
% \end{figure}
% \begin{figure}
%     \centering
%     %\includegraphics[width=7 cm]{figures/icecube/selection/IC2018_LE_L3_Vars_CleanedFullTimeLength.pdf}
%     \tikzsetnextfilename{l3_cleaned_full_time_length}%
\begin{tikzpicture}
    \pgfplotstableread{figures/icecube/selection/Level3/CleanedFullTimeLength_level3_data_mc_hists.csv}\table
    \begin{groupplot}[
        % xmin=0.0,xmax=25000.0,
        % this filter is applied to all x-coordinates in all plots!
        % Note that this filter would have to work on the *log* of the coordinates
        % if the x-axis was a log-axis (i.e. instead of scaling, you would shift).
        x filter/.expression={x*1e-3},
        xmin=0.0,xmax=25,
        xmajorgrids, ymajorgrids,
        width=0.45\linewidth,
        % The line below sets the position of the y-labels for all axes in
        % relative coordinates. Comment out to set positions automatically (which
        % may result in the "data/MC ratio" label being set a bit too far inside).
        ylabel style={at={(-0.15,0.5)}},
        group/.cd,
        group size=1 by 2,
        xticklabels at=edge bottom,
        vertical sep=10pt
        ]
    \nextgroupplot[
        height=0.35\linewidth,
        legend cell align={left},
        legend columns=4,
        legend to name=l3_vars_legend,
        legend style={
          at={(0.95,0.95)},
          anchor=north east,
        },
        ymode=log,
        % ymin=1e-05, ymax=1000.0,
        ymin=2e-5, ymax=5e2,
        ytick distance=1e1,
        ylabel={rate (Hz)},
        y filter/.expression={y < 1e-6 ? ln(1e-6) : ln(y)}
    ]
    % for some reason we called the column "muon", not "muons"...
    \ploterrorband[muon_color]{muon}{1}
    \addlegendentry{atm. muons}
    \ploterrorband[nue_color]{nue}{100}
    \addlegendentry{$\nu_e + \bar{\nu}_e$ x100}
    \ploterrorband[numu_color]{numu}{100}
    \addlegendentry{$\nu_\mu + \bar{\nu}_\mu$ x100}
    \ploterrorband[nutau_color]{nutau}{1000}
    \addlegendentry{$\nu_\tau + \bar{\nu}_\tau$ x1000}
    \ploterrorband[noise_color]{noise}{1}
    \addlegendentry{noise}
    \ploterrorband{total_mc}{1}
    \addlegendentry{total MC}

    \ploterrorbar{data}
    \addlegendentry{data (2014, 12 runs)}

    % draw cut
    % use "axis cs" to give coordinates in the data coordinate system!

    % \draw[thick,dashed] (axis cs:5000,1e-4) -- (axis cs:5000,100);
    % \draw[-stealth, very thick] (axis cs:5000,10)  -- node[anchor=south]{signal} (axis cs:1000,10);

    % since we scaled the coordnates to microseconds, we also need to use that for drawing
    \draw[thick,dashed] (axis cs:5,1e-6) -- (axis cs:5,100);
    \draw[-stealth, very thick] (axis cs:5,10)  -- node[anchor=south]{\footnotesize\sffamily signal} (axis cs:1,10);

    \nextgroupplot[
        height=0.2\linewidth,
        ymin=0.5, ymax=4,
        ylabel={data/MC ratio},
        xlabel={CleanedFullTimeLength ($\mu$s)}
    ]
    \ploterrorband{data_mc_ratio}{1}
    \end{groupplot}
\end{tikzpicture}

%     \caption{Distribution of one of the variables used in the L3 offline filter, the time between the last hit and the first hit after noise cleaning. Histograms show the distributions in simulated data separated by particle and interaction type, data points with error bars show the distribution of real data. The bottom panel shows the ratio between data and simulation. Events falling on the "signal" side of the histogram are passed to the next filter level.}
%     \label{fig:l3-var-cleaned-full-time-length}
% \end{figure}

\begin{figure*}
    \centering
    \ref{l3_vars_legend}\par
    \tikzsetnextfilename{l3_cleaned_full_time_length}%
\begin{tikzpicture}
    \pgfplotstableread{figures/icecube/selection/Level3/CleanedFullTimeLength_level3_data_mc_hists.csv}\table
    \begin{groupplot}[
        % xmin=0.0,xmax=25000.0,
        % this filter is applied to all x-coordinates in all plots!
        % Note that this filter would have to work on the *log* of the coordinates
        % if the x-axis was a log-axis (i.e. instead of scaling, you would shift).
        x filter/.expression={x*1e-3},
        xmin=0.0,xmax=25,
        xmajorgrids, ymajorgrids,
        width=0.45\linewidth,
        % The line below sets the position of the y-labels for all axes in
        % relative coordinates. Comment out to set positions automatically (which
        % may result in the "data/MC ratio" label being set a bit too far inside).
        ylabel style={at={(-0.15,0.5)}},
        group/.cd,
        group size=1 by 2,
        xticklabels at=edge bottom,
        vertical sep=10pt
        ]
    \nextgroupplot[
        height=0.35\linewidth,
        legend cell align={left},
        legend columns=4,
        legend to name=l3_vars_legend,
        legend style={
          at={(0.95,0.95)},
          anchor=north east,
        },
        ymode=log,
        % ymin=1e-05, ymax=1000.0,
        ymin=2e-5, ymax=5e2,
        ytick distance=1e1,
        ylabel={rate (Hz)},
        y filter/.expression={y < 1e-6 ? ln(1e-6) : ln(y)}
    ]
    % for some reason we called the column "muon", not "muons"...
    \ploterrorband[muon_color]{muon}{1}
    \addlegendentry{atm. muons}
    \ploterrorband[nue_color]{nue}{100}
    \addlegendentry{$\nu_e + \bar{\nu}_e$ x100}
    \ploterrorband[numu_color]{numu}{100}
    \addlegendentry{$\nu_\mu + \bar{\nu}_\mu$ x100}
    \ploterrorband[nutau_color]{nutau}{1000}
    \addlegendentry{$\nu_\tau + \bar{\nu}_\tau$ x1000}
    \ploterrorband[noise_color]{noise}{1}
    \addlegendentry{noise}
    \ploterrorband{total_mc}{1}
    \addlegendentry{total MC}

    \ploterrorbar{data}
    \addlegendentry{data (2014, 12 runs)}

    % draw cut
    % use "axis cs" to give coordinates in the data coordinate system!

    % \draw[thick,dashed] (axis cs:5000,1e-4) -- (axis cs:5000,100);
    % \draw[-stealth, very thick] (axis cs:5000,10)  -- node[anchor=south]{signal} (axis cs:1000,10);

    % since we scaled the coordnates to microseconds, we also need to use that for drawing
    \draw[thick,dashed] (axis cs:5,1e-6) -- (axis cs:5,100);
    \draw[-stealth, very thick] (axis cs:5,10)  -- node[anchor=south]{\footnotesize\sffamily signal} (axis cs:1,10);

    \nextgroupplot[
        height=0.2\linewidth,
        ymin=0.5, ymax=4,
        ylabel={data/MC ratio},
        xlabel={CleanedFullTimeLength ($\mu$s)}
    ]
    \ploterrorband{data_mc_ratio}{1}
    \end{groupplot}
\end{tikzpicture}

    \tikzsetnextfilename{l3_vertex_guess_z}%
\begin{tikzpicture}
    \pgfplotstableread{figures/icecube/selection/Level3/VertexGuessZ_level3_data_mc_hists.csv}\table
    \begin{groupplot}[
        xmin=-600,xmax=600,
        xmajorgrids, ymajorgrids,
        width=0.45\linewidth,
        % The line below sets the position of the y-labels for all axes in 
        % relative coordinates. Comment out to set positions automatically (which 
        % may result in the "data/MC ratio" label being set a bit too far inside).
        ylabel style={at={(-0.15,0.5)}},
        group/.cd,
        group size=1 by 2,
        xticklabels at=edge bottom,
        vertical sep=10pt
        ]
    \nextgroupplot[
        height=0.35\linewidth,
        legend cell align={left},
        legend columns=1,
        legend to name=l3_vars_legend_vertex_guess,
        ymode=log,
        ymin=2e-5, ymax=5e2,
        ytick distance=1e1,
        ylabel={rate (Hz)},
        y filter/.expression={y < 1e-6 ? ln(1e-6) : ln(y)}
    ]
    % for some reason we called the column "muon", not "muons"...
    \ploterrorband[muon_color]{muon}{1}
    \addlegendentry{atm. muons}
    \ploterrorband[nue_color]{nue}{100}
    \addlegendentry{$\nu_e + \bar{\nu}_e$ x100}
    \ploterrorband[numu_color]{numu}{100}
    \addlegendentry{$\nu_\mu + \bar{\nu}_\mu$ x100}
    \ploterrorband[nutau_color]{nutau}{1000}
    \addlegendentry{$\nu_\tau + \bar{\nu}_\tau$ x1000}
    \ploterrorband[noise_color]{noise}{1}
    \addlegendentry{noise}
    \ploterrorband{total_mc}{1}
    \addlegendentry{total MC}

    \ploterrorbar{data}
    \addlegendentry{data (2014, 12 runs)}

    \draw[thick,dashed] (axis cs:-120,1e-6) -- (axis cs:-120,100);
    \draw[-stealth, very thick] (axis cs:-120,5)  -- node[anchor=south]{\footnotesize\sffamily signal} (axis cs:-400,5);

    \nextgroupplot[
        height=0.2\linewidth,
        ymin=0.5, ymax=4,
        ylabel={data/MC ratio},
        xlabel={first HLC z-position}
    ]
    \ploterrorband{data_mc_ratio}{1}
    \end{groupplot}
\end{tikzpicture}


    \caption{Distribution of one of the variables used in the L3 offline filter, the time between the last hit and the first hit after noise cleaning (left) and the z-position of the first HLC hit (right). Histograms show the distributions in simulated data separated by event type, data points with error bars show the distribution of real data. The bottom panel shows the ratio between data and simulation. Events falling on the "signal" side of the histogram are passed to the next filter level.}
    \label{fig:l3-cut-vars}
\end{figure*}

\subsubsection{Level 4}
\label{sec:level4-selection}
In the next level, L4, more advanced selections based on the output of Boosted Decision Trees (BDTs) are applied, with a separately trained BDT for noise and muon rejection, respectively.
The output of each BDT is a probability score between zero (background-like) and one (signal-like) and is shown in \reffig{l4-bdt-output}.
%\begin{table}
%\caption{Input variables into the L3 noise BDT.\label{tab:l4-noise-variables}}
%\begin{tabular}{cp{4cm}}\toprule
%    Variable & Description  \\ \midrule
%    cleaned $N_\mathrm{ch}$ & Number of hits in the noise cleaned hit series that was also used at L3. \\
%    \texttt{STW m3500p4000 DTW200} & Slide a \SI{200}{\nano\second} time window over all pulses occurring from \SI{-3.5}{\micro\second} to \SI{4}{\micro\second} around the trigger time and take the largest number of pulses to fall within the window.  \\ \bottomrule
%\end{tabular}
%\end{table}
The first BDT to be evaluated is the one used to reject pure noise events.
Its inputs consist of five variables:
\begin{itemize}
    \item Cleaned $N_\mathrm{ch}$: Number of hits in the noise cleaned hit series that was also used at L3.
    \item \texttt{STW m3500p4000 DTW200}: Slide a \SI{200}{\nano\second} time window over all pulses occurring from \SI{-3.5}{\micro\second} to \SI{4}{\micro\second} around the trigger time and take the largest number of pulses to fall within the window.
    \item The speed that is returned by the \emph{LineFit} algorithm\cite{linefit} on the observed hits
    \item The "fill ratio", that is, the fraction of DOMs that have recorded any hits inside a sphere centered around the first HLC hit.
    %and with a radius of $R = 1.6\times\frac{1}{N}\sum_i^N \abs{\vec{r}_{\mathrm{DOM},i} - \vec{r}_\mathrm{vertex}}$, where $N$ is the number of DOMs with hits, $\vec{r}_{\mathrm{DOM},i}$ is the position of DOM $i$, and $\vec{r}_\mathrm{vertex}$ is the position of the first HLC hit.
    This variable effectively measures how compactly the hits are distributed around the starting point of the event and has been used in the past to identify cascades in IceCube\cite{icecube_2011}.
    \item The ratio between the total duration of the cleaned hit series and the uncleaned hit series.
\end{itemize}
The BDT is trained using simulated pure noise and neutrino events.
An event passes the noise filter if the BDT score is above 0.7, which reduces the number of pure noise events by two orders of magnitude from 36.6~mHz to approximately 0.3~mHz.
Passing events are passed into the second BDT that is used to reject atmospheric muons.
This BDT takes in a larger number of variables that can be summarized as belonging to three different categories:
\begin{itemize}
    \item Several $N_\mathrm{ch}$-like variables counting the number of hits in different veto regions of the detector.
    \item The time to reach 75\% of an event's charge in the cleaned pulse series.
    \item Several variables that characterize the spacial distribution of hits in z-coordinate and the radius with respect to string 36 (roughly the center of DeepCore).
\end{itemize}
In contrast to the noise BDT, however, the muon BDT is trained using real data and simulated neutrino events, with the goal of rejecting data events.
This is possible because the data sample consists to 99\% of atmospheric muons at this stage of the event selection.
Events pass the L4 muon cut if the output score of the muon BDT is greater than 0.65, removing 94\% of all muon events while keeping 87\% of all neutrinos.
These thresholds are shown alongside the distribution of the BDT outputs in \reffig{l4-bdt-output}.
\begin{figure*}
    \centering
    %\includegraphics[width=7 cm]{figures/icecube/selection/L4_noiseBDT_L4_NoiseClassifier_ProbNu.pdf}
    %\includegraphics[width=7 cm]{figures/icecube/selection/L4_muon_L4_MuonClassifier_Data_ProbNu.pdf}
    \ref{l4_bdt_hist_legend}\hfill
    \tikzsetnextfilename{l4_noise_classifier_probnu}%
\begin{tikzpicture}
    \pgfplotstableread{figures/icecube/selection/Level4/L4_NoiseClassifier_ProbNu_level4_data_mc_hists.csv}\table
    \begin{groupplot}[
        xmin=0.0,xmax=1.0,
        xmajorgrids, ymajorgrids,
        width=0.45\linewidth,
        ylabel style={at={(-0.15,0.5)}},
        group/.cd,
        group size=1 by 2,
        xticklabels at=edge bottom,
        vertical sep=10pt,
        %every axis y label style={xshift=0.5 cm}
        ]
    \nextgroupplot[
        height=0.35\linewidth,
        legend cell align={left},
        % legend columns=-1,
        legend to name=l4_bdt_hist_legend,
        legend columns=-1,
        ymode=log,
        ymin=1e-07, ymax=1.0,
        ylabel={rate (Hz)},
        ytick distance=1e1
    ]
    \ploterrorband[muon_color]{muons}{1}
    \addlegendentry{atm. muons}
    \ploterrorband[nu_color]{nu}{1}
    \addlegendentry{atm. $\nu$}
    % \ploterrorband[nue_color]{nue}{1}
    % \addlegendentry{$\nu_e + \bar{\nu}_e$}
    % \ploterrorband[numu_color]{numu}{1}
    % \addlegendentry{$\nu_\mu + \bar{\nu}_\mu$}
    % \ploterrorband[nutau_color]{nutau}{1}
    % \addlegendentry{$\nu_\tau + \bar{\nu}_\tau$}
    \ploterrorband[noise_color]{noise}{1}
    \addlegendentry{noise}
    \ploterrorband{total_mc}{1}
    \addlegendentry{total MC}
    \ploterrorbar{data}
    \addlegendentry{season 2014
    (12 runs)}

    % draw cut
    \draw[thick,dashed] (axis cs:0.7,1e-7) -- (axis cs:0.7,10);
    \draw[-stealth, very thick] (axis cs:0.7,2e-2)  -- node[anchor=south]{\footnotesize\sffamily signal} (axis cs:0.9,2e-2);

    \nextgroupplot[
        height=0.2\linewidth,
        ymin=-0.1, ymax=1.8,
        ylabel=rate/total MC,
        xlabel={L4 NoiseClassifier ProbNu}
    ]

    \ploterrorbar{data_mc_ratio}
    \plotratioerrorband[muon_color]{muons}{total_mc}
    \plotratioerrorband[nu_color]{nu}{total_mc}
    % \plotratioerrorband[nue_color]{nue}{total_mc}
    % \plotratioerrorband[numu_color]{numu}{total_mc}
    % \plotratioerrorband[nutau_color]{nutau}{total_mc}
    \plotratioerrorband[noise_color]{noise}{total_mc}
    \end{groupplot}
\end{tikzpicture}

    \tikzsetnextfilename{l4_muon_classifier_probnu}%
\begin{tikzpicture}
    \pgfplotstableread{figures/icecube/selection/Level4/L4_MuonClassifier_Data_ProbNu_level4_data_mc_hists.csv}\table
    \begin{groupplot}[
        xmin=0.0,xmax=1.0,
        xmajorgrids, ymajorgrids,
        width=0.45\linewidth,
        ylabel style={at={(-0.15,0.5)}},
        group/.cd,
        group size=1 by 2,
        xticklabels at=edge bottom,
        vertical sep=10pt
        ]
    \nextgroupplot[
        height=0.35\linewidth,
        legend cell align={left},
        legend columns=2,
        ymode=log,
        ymin=1e-07, ymax=1.0,
        ylabel={rate (Hz)},
        ytick distance=1e1
    ]
    \ploterrorband[muon_color]{muons}{1}
    % \addlegendentry{atm. muons}
    \ploterrorband[nu_color]{nu}{1}
    % \addlegendentry{atm. $\nu$}
    % \ploterrorband[nue_color]{nue}{1}
    % % \addlegendentry{$\nu_e + \bar{\nu}_e$ x10}
    % \ploterrorband[numu_color]{numu}{1}
    % % \addlegendentry{$\nu_\mu + \bar{\nu}_\mu$ x10}
    % \ploterrorband[nutau_color]{nutau}{1}
    % % \addlegendentry{$\nu_\tau + \bar{\nu}_\tau$ x100}
    \ploterrorband[noise_color]{noise}{1}
    % \addlegendentry{noise}
    \ploterrorband{total_mc}{1}
    % \addlegendentry{total MC}

    \addplot[
        mark=*,
        mark options={scale=0.5, fill=black},
        black,
        only marks,
        error bars/.cd,
        x dir=none,
        y dir=both,
        y explicit,
    ] table [x=bin_midpoints, y=data, y error=data__err]  from \table;
    % \addlegendentry{season 2014 (12 runs)}

    % draw cut
    \draw[thick,dashed] (axis cs:0.65,1e-7) -- (axis cs:0.65,10);
    \draw[-stealth, very thick] (axis cs:0.65,2e-2)  -- node[anchor=south]{\footnotesize\sffamily signal} (axis cs:0.85,2e-2);

    \nextgroupplot[
        height=0.2\linewidth,
        ymin=-0.1, ymax=1.8,
        ylabel=rate/total MC,
        xlabel={L4 MuonClassifier Data ProbNu}
    ]

    \ploterrorbar{data_mc_ratio}
    \plotratioerrorband[muon_color]{muons}{total_mc}
    \plotratioerrorband[nu_color]{nu}{total_mc}
    % \plotratioerrorband[nue_color]{nue}{total_mc}
    % \plotratioerrorband[numu_color]{numu}{total_mc}
    % \plotratioerrorband[nutau_color]{nutau}{total_mc}
    \plotratioerrorband[noise_color]{noise}{total_mc}

    \end{groupplot}
\end{tikzpicture}


    \caption{Distribution scores for the noise (left) and muon (right) BDT. The distributions of the muon classifier are shown for events where the score of the noise BDT is greater than 0.7. Histograms show the distributions in simulated data separated by event type, data points with error bars show the distribution of real data. The bottom panel shows the ratio between data and simulation. Events falling on the "signal" side of the histogram are passed to the next filter level.}
    \label{fig:l4-bdt-output}
\end{figure*}

\subsubsection{Level 5}
The final filter that is applied before the event reconstruction step is L5.
This filter searches specifically for hits occurring in un-instrumented \emph{corridors} within the IceCube array through which an atmospheric muon can sneak into the DeepCore volume while evading previous veto cuts.
In addition, events with more than seven hits in the outermost strings of the IceCube array or that have a down-going pattern of hits in the uppermost region of the detector are vetoed to remove events containing atmospheric muons entering the detector coincidentally with neutrinos.
The distribution for one of the corridor variables and one of the muon rejection variables are shown in \reffig{l5-vars}.
Table~\ref{tab:l5_summary} shows the rates of each event type expected at each level of the selection up to L5 together with the efficiency of the filter at the final level.
\begin{figure*}
    \centering
    % \includegraphics[width=7 cm]{figures/icecube/selection/L5_contained_L5_WideCorridorCutCount.pdf}
    % \includegraphics[width=7 cm]{figures/icecube/selection/SRTTWOfflinePulsesDC_ContainmentVars.z_travel_top15.png}
    \ref{l5_corridor_vars_legend}
    \tikzsetnextfilename{l5_corridor_cut_count}%
\begin{tikzpicture}
    \pgfplotstableread{figures/icecube/selection/Level5/L5_WideCorridorCutCount_level5_data_mc_hists.csv}\table
    \begin{groupplot}[
        xmin=0,xmax=15,
        xmajorgrids, ymajorgrids,
        width=0.45\linewidth,
        ylabel style={at={(-0.15,0.5)}},
        %scale only axis=true,
        group/.cd,
        group size=1 by 2,
        xticklabels at=edge bottom,
        vertical sep=10pt
        ]
    \nextgroupplot[
        height=0.3\linewidth,
        legend cell align={left},
        legend columns=4,
        legend to name=l5_corridor_vars_legend,
        ymode=log,
        ymin=5e-06, ymax=0.2,
        ytick distance=1e1,
        ylabel={rate (Hz)},
        y filter/.expression={y < 1e-7 ? ln(1e-8) : ln(y)}
    ]

    \ploterrorband[muon_color]{muons}{1}
    \addlegendentry{atm. muons}
    \ploterrorband[nue_color]{nue}{1}
    \addlegendentry{$\nu_e + \bar{\nu}_e$}
    \ploterrorband[numu_color]{numu}{1}
    \addlegendentry{$\nu_\mu + \bar{\nu}_\mu$}
    \ploterrorband[nutau_color]{nutau}{1}
    \addlegendentry{$\nu_\tau + \bar{\nu}_\tau$}
    \ploterrorband[noise_color]{noise}{1}
    \addlegendentry{noise}
    \ploterrorband{total_mc}{1}
    \addlegendentry{total MC}
    \ploterrorbar{data}
    \addlegendentry{season 2014
    (12 runs)}

    % draw cut
    \draw[thick,dashed] (axis cs:2,1e-6) -- (axis cs:2,1e-1);
    \draw[stealth-, very thick] (axis cs:2,2e-2)  -- node[anchor=south]{\footnotesize\sffamily cut} (axis cs:5,2e-2);

    \nextgroupplot[
        height=0.2\linewidth,
        ymin=-0.1, ymax=2.8,
        ylabel=rate/total MC,
        xlabel={Number of hits in corridor\strut }
    ]
   
    \ploterrorbar{data_mc_ratio}
    \plotratioerrorband[muon_color]{muons}{total_mc}
    \plotratioerrorband[nue_color]{nue}{total_mc}
    \plotratioerrorband[numu_color]{numu}{total_mc}
    \plotratioerrorband[nutau_color]{nutau}{total_mc}
    \plotratioerrorband[noise_color]{noise}{total_mc}
    
    \end{groupplot}
\end{tikzpicture}

    \tikzsetnextfilename{l5_corridor_angle_total_cos_diff}%
\begin{tikzpicture}
    \pgfplotstableread{figures/icecube/selection/Level5/L5_WideCorridorCutTrack_L5_SPEFit11_angles_total_cos_diff_level5_data_mc_hists.csv}\table
    \begin{groupplot}[
        xmin=-1.1,xmax=1.0,
        xmajorgrids, ymajorgrids,
        width=0.45\linewidth,
        ylabel style={at={(-0.15,0.5)}},
        %scale only axis=true,
        group/.cd,
        group size=1 by 2,
        xticklabels at=edge bottom,
        vertical sep=10pt
        ]
    \nextgroupplot[
        height=0.3\linewidth,
        legend cell align={left},
        legend columns=2,
        legend to name=l5_corridor_anglediff_legend,
        ymode=log,
        ymin=5e-06, ymax=2e-2,
        ytick distance=1e1,
        ylabel={rate (Hz)},
        y filter/.expression={y < 1e-7 ? ln(1e-7) : ln(y)}
    ]
    \ploterrorband[muon_color]{muons}{1}
    \addlegendentry{atm. muons}
    \ploterrorband[nue_color]{nue}{1}
    \addlegendentry{$\nu_e + \bar{\nu}_e$}
    \ploterrorband[numu_color]{numu}{1}
    \addlegendentry{$\nu_\mu + \bar{\nu}_\mu$}
    \ploterrorband[nutau_color]{nutau}{1}
    \addlegendentry{$\nu_\tau + \bar{\nu}_\tau$}
    \ploterrorband[noise_color]{noise}{1}
    \addlegendentry{noise}
    \ploterrorband{total_mc}{1}
    \addlegendentry{total MC}
    
    \ploterrorbar{data}
    \addlegendentry{season 2014
    (12 runs)}

    % draw cut
    \draw[thick,dashed] (axis cs:0.7,1e-6) -- (axis cs:0.7,1e-2);
    \draw[-stealth, very thick] (axis cs:0.7,2e-3)  -- node[anchor=south]{\footnotesize\sffamily signal} (axis cs:0.2,2e-3);
    
    \nextgroupplot[
        height=0.2\linewidth,
        ymin=0, ymax=2,
        ylabel={data/MC ratio},
        xlabel={Cosine of angle difference to corridor\strut}
    ]
    \ploterrorband{data_mc_ratio}{1}

    \end{groupplot}
\end{tikzpicture}

    \caption{Distributions for two of the L5 corridor cut variables. Histograms show the distributions in simulated data separated by event type, data points with error bars show the distribution of real data. The bottom panel shows the ratio between data and simulation. Events falling on the "signal" side of the histogram (or, equivalently, opposite to the "cut" side of the histogram) are passed to the next filter level.}
    \label{fig:l5-vars}
\end{figure*}

\begin{table}
\caption{Summary of the rates obtained after each level of selection. Neutrinos are weighted to an atmospheric spectrum with oscillations included.}
\label{tab:l5_summary}
\begin{tabular}{lrrrrr}\toprule
& \multicolumn{4}{c}{rate ($\mu$Hz)} & \\ \cmidrule{2-5}
Event type  & DeepCore filter   & L3   & L4   & L5   & Eff. (\%) \\
\toprule
Atm. $\mu$         & 7273 & 505  & 28.1 & 0.93 & 0.012          \\
Pure noise         & 6621 & 36.6 & 0.28 & 0.07 & 0.001          \\
Atm. $\nu_e$ CC    & 1.61 & 0.95 & 0.84 & 0.48 & 29.8           \\
Atm. $\nu_\mu$ CC  & 6.16 & 3.77 & 3.11 & 1.39 & 22.5           \\
Atm. $\nu_\tau$ CC & 0.19 & 0.13 & 0.12 & 0.07 & 36.8           \\
Atm. $\nu$ NC      & 0.86 & 0.53 & 0.46 & 0.23 & 26.7  \\
\bottomrule
\end{tabular}
\end{table}

\subsection{Event Reconstruction}
\label{sec:event-reconstruction}

After the L5 selection, the rate of muons is reduced enough so that the majority of the total sample is expected to consist of atmospheric neutrinos, and it is at this point that the event reconstruction and signature classification are run.
For the measurement presented in this thesis, three reconstructed quantities are required: the zenith angle, the energy, and a proxy score determining the flavor of a neutrino.
As described in Section \ref{sec:particle-signatures}, all neutrino events in DeepCore can be effectively approximated as a cascade ($\nu_e$ CC events, all NC events and 83\% of $\nu_\tau$ CC events) or a combination of a cascade at the neutrino interaction point with an outgoing muon track ($\nu_\mu$ CC events and 17\% of $\nu_\tau$ CC events).
The zenith angle can be most accurately reconstructed for track-like events due to their elongated, highly directional signature.
For cascades, the reconstruction of the direction is more difficult because of their more compact and diffuse light distribution.
The energy of a neutrino event is reconstructed by comparing the expected light output of a combined track and cascade hypothesis with the observed hits.
Finally, the flavor proxy is calculated using variables that characterize the elongation of the observed hit signature and the goodness of fit of a combined track and cascade hypothesis compared to that of a cascade-only hypothesis.
The resulting score allows the separation of muon neutrino interactions from other interactions, which is ideally suitable to observe the muon neutrino disappearance oscillation channel.

\subsubsection{Zenith angle reconstruction}
\label{sec:santa}
The zenith angle is reconstructed using the Single-string Antares-inspired Analysis (\textsc{santa})\sidecite{Garza2014Measurement}.
It is an older algorithm aimed at reconstructing the direction of muon tracks that was originally developed for use in the ANTARES neutrino telescope~\sidecite{Aguilar:2011zz}.
It has since been refurbished and improved in IceCube, as described in detail in~\sidecite{lowen-reco-paper}.

The reconstructed pulse series in every DOM is summarized by the time of the first pulse and the sum of charges of all pulses.
This time and charge are the only information used by the reconstruction and are referred to as a \emph{hit} in the following.
The first step of the reconstruction algorithm is a cleaning routine that removes hits produced
by photons that have been scattered many times as they traveled
through the ice, leaving only hits from photons that have traveled in approximately straight lines based on the time difference between hits on the same string.
%The algorithm is a simplification from an earlier implementation described in \cite{Garza2014Measurement}.
It calculates the signal speed between hits on the same string, and removes a hit if this velocity is below the speed of light in ice.
This is a simplification of the algorithm described in \cite{Garza2014Measurement}, where the effective signal velocity was updated during the selection process.
The selection is run separately for each string, and if fewer than three hits remain on a string, all hits on the string are discarded.
In total, it is necessary for at least five hits to remain in an event in order to run the directional reconstruction.
If only hits on one string remain after the selection, the event is referred to as a \emph{single-string} event, otherwise it is a \emph{multi-string} event.
The reconstruction is generally more accurate for multi-string events, because the spacing between strings provides a long lever arm to constrain the direction of a track.
In addition, the azimuth angle of the track can only be reconstructed for \emph{multi-string} events due to the rotational symmetry of a single string.

\begin{figure*}
    \centering
    \includegraphics[width=\linewidth]{figures/icecube/reconstruction/santa/multi_string_example_with_cleaning_id_12607962.pdf}
    \caption{Example of a \numucc event reconstructed with \textsc{santa} with hits on several strings. Strings 84, 83 and 37 are spaced $\sim80\,\mathrm{m}$ apart from each other and form a highly obtuse triangle.}
    \label{fig:santa-multi-string-example}
\end{figure*}

The directional reconstruction itself is a regression that minimizes a modified $\chi^2$ loss that is defined as
\begin{equation}
L(\vec{\theta})=\sum_{i=1}^{N}\phi(r^2_i(\vec{\theta}))
+
\frac{1}{\bar{q}}\sum_{i=1}^{N}\tilde{q}_i \frac{d_{\gamma,i}}{d_0}\,.\label{eq:chi-square-mod-loss}
\end{equation}
The first term in \refeq{chi-square-mod-loss} is a $\chi^2$ loss that is modified to be robust against outliers. The second term is a regularization term that penalizes solutions where large charges are observed at large distances. Here, $r^2_i$ is the chi-square residual for each observed hit, $i$, between the observed time, $t_{\mathrm{obs}, i}$ and the geometric arrival time, $t_{\mathrm{geom},i}(\vec{\theta})$,
\begin{equation}
r_{i}^{2}(\vec{\theta})=\left(\frac{t_{\mathrm{geom},i}(\vec{\theta})-t_{\mathrm{obs},i}}{\sigma_{t}}\right)^{2}\,.
\end{equation}
The residual is wrapped in a \emph{robust loss function}, $\phi(r_{i}^{2})=\log\left(1+r_{i}^{2}/C^2\right)C^2$, which grows much more slowly than $r_{i}^{2}$ for values of $r_i$ greater than the "soft cutoff", $C$, while behaves very similarly to $r_i^2$ for values of $r_i$ smaller than $C$.
Effectively, this robust loss reduces the influence of hits that pass the hit selection procedure despite having undergone a significant amount of scattering.
The uncertainty in the pulse-time measurement is approximately $\sigma_{t}=3\,\mathrm{ns}$, corresponding to the readout rate of the modules~\cite{icecube_daq}.

In the second term of \refeq{chi-square-mod-loss}, $\tilde{q}_i$ is the total observed charge in DOM $i$ divided by the effective area of the DOM at the angle of incidence of the photon, and $\bar{q}$ is the average over all $\tilde{q}_i$.
For every DOM, the charge per effective area is multiplied by the distance traveled by the photon, $d_{\gamma,i}$, and divided by a typical scaling distance, $d_0$.
The distance $d_0$ determines the strength of the regularization term and has been optimized empirically to achieve the optimal resolution of the reconstruction to a value of \SI{7}{\meter}.

The expected arrival time for unscattered Cherenkov photons is calculated geometrically under the assumption of an infinitely long track characterized by a normalized direction vector $\vec{u}=(u_{x},u_{y},u_{z})$,
an anchor point $\vec{q}=(q_{x},q_{y},q_{z})$ and a time $t_{0}$
at which the particle passes through $\vec{q}$.
The
velocity is fixed to the vacuum speed of light, $c$.
Since the reconstruction ignores DOMs that have not recorded any pulses, the fact that the true track length is finite only makes a negligible  difference.
The arrangement of these vectors is shown in \reffig{Detailed-track-geometry}.
Without scattering, all Cherenkov photons lie on a cone with an opening
angle $\theta_{c}$
\begin{figure}[h]
\begin{centering}
\tikzsetnextfilename{track_geometry_santa}%
\begin{tikzpicture}[scale=1,>=stealth]
	\path[name path=track] (0,3) -- (9,3);
	\node[shape=star,
	      star point height=1cm,
	      star point ratio=0.5,
	      draw, fill=black,
	      label=below:$\vec{p}(t_{\mathrm{em}})$] (emission) at (1,3) {};
	\draw[->, decorate,
	decoration={snake,amplitude=.4mm,segment length=2mm,post length=1mm}]
		(emission.center)
		-- node[sloped, above] {$d_{\gamma}$} +(40:4)
		node[label=above:DOM at $\vec{r}$] (dompos) {};
	\path[name path=cone] (dompos.center) -- +(-50:4);
	\draw[name intersections={of=track and cone, by=tip}]
		(dompos.center) -- node[sloped, above] {Cherenkov light cone} (tip)
		node[label=below:$\vec{p}(t_{\mathrm{geom}})$] (muonpos) {};
	\draw[fill=black, opacity=0.5] (dompos.center) circle (5pt);
	\draw[color=black, ->, style=very thick] (0,3) node[anchor=north]{muon} -- (muonpos.center);
	\draw (emission.center) +(1,0) node[anchor=south east]{$\theta_c$}  arc (0:40:1);
	\path (emission.center)
		-- node[shape=circle,
			fill=black,
			label=below:$\vec{q}$] (vertex) {}
		(tip);
	\draw[->] (vertex.center) -- node[sloped, below] {$\vec{r}-\vec{q}$} (dompos.center);
	%\draw (vertex.center) +(-1,0) arc (180:140:1);
	%\path (vertex.center) -- +(160:0.6) node {$\theta$};
	\draw[->] (vertex.center) ++(0.2, 0.2) -- node[above] {$\vec{u}$} +(1,0);
\end{tikzpicture}\par
\end{centering}
\caption{\label{fig:Detailed-track-geometry}Detailed geometry of a light cone
created by a track.
$\vec{q}$ is the position of the anchor point
and $\vec{r}$ is the position of the optical module. $\vec{p}(t_{\mathrm{em}})$
and $\vec{p}(t_{\mathrm{geom}})$ are the positions of the muon at
the time the photon is emitted and when it is geometrically expected
to arrive, respectively.}
\end{figure}
whose tip is in the position of the particle at the time $\vec{p}(t)$. The opening angle satisfies $\cos(\theta_c)=1/n_{\mathrm{ph}}$, where $n_{\mathrm{ph}}$ is the phase index of refraction of the ice.
Assuming that a photon has traveled in a straight line at the group velocity in ice, the geometric arrival time, $t_{\mathrm{geom}}$, for a DOM at position $\vec{r}$ is

\begin{equation}
t_{\mathrm{geom}}=t_{0}+\frac{1}{c}\left(\left(\vec{r}-\vec{q}\right)\cdot\vec{u}+\frac{d_{\gamma}}{n_{\mathrm{ph}}}\left(n_{\mathrm{ph}}n_{\mathrm{gr}}-1\right)\right)\label{eq:t_geom-MS-track}
\end{equation}
where the distance traveled by the photon $d_\gamma$ is
\begin{equation}
d_{\gamma}=n_{\mathrm{ph}}\sqrt{\frac{1}{n_{\mathrm{ph}}^{2}-1}\left(\vec{u}\times\left(\vec{r}-\vec{q}\right)\right)^{2}}\,.\label{eq:photon-distance-3d}
\end{equation}

\begin{figure}
    \centering
    \includegraphics[width=0.8\linewidth]{figures/icecube/reconstruction/santa/santa_absolute_error_final.pdf}
    \caption{Median error on the reconstructed zenith angle at the final level of the sample selection as a function of the true simulated neutrino energy.}
    \label{fig:santa-resolution}
\end{figure}

The group and phase indices of refraction depend on the wavelength, but for this reconstruction the value for a wavelength of $\lambda=400\;\mathrm{nm}$
\footnote{$400\;\mathrm{nm}$ is near the wavelength of the highest acceptance of the optical modules.\cite{icecube_detector_17}} is used, where $n_{\mathrm{gr}}=1.356$ and $n_{\mathrm{ph}}=1.319$~from~\sidecite{PRICE200197}.
An example of a simulated event reconstructed with \textsc{santa} is shown in \reffig{santa-multi-string-example}.
The solid and dashed lines show the geometric arrival time calculated according to equation~\ref{eq:t_geom-MS-track} using reconstructed and true track parameters, respectively.
The circles indicate hits in DOMs, and those hits that have been removed by the hit cleaning procedure are crossed out.
The median error on the zenith angle reconstructed using \textsc{SANTA} is shown in \reffig{santa-resolution}, split by neutrino interaction type.
As expected, the error is the smallest for $\nu_\mu$-CC interactions, since those produce track signatures that most closely resemble the infinite track hypothesis underlying the \textsc{SANTA} reconstruction algorithm.
The worst resolution is achieved for interactions that only produce electronic or hadronic showers, since they produce cascade signatures hardly resembling the infinite track assumption.
It is also apparent that the median resolution for $\nu_\tau$-CC events lies between that of $\nu_\mu$-CC events and pure cascade events.
This is readily explained by the fact that 17\% of these interactions also produce a muon in their final state.

In addition to the zenith angle reconstruction, SANTA can also be used to fit a simplified cascade hypothesis to the observed hits.
For this purpose, it is assumed that light is emitted uniformly in all directions originating from the interaction vertex as shown in the left panel in \reffig{idealized_signatures}.
With this assumption of perfect rotational symmetry, it is not possible to reconstruct a direction, and the cascade is fully characterized by the position of the vertex and the interaction time.
The ratio of the $\chi^2$ of the infinite-track regression and the $\chi^2$ of this \emph{cascade-only} regression is used as a proxy for the neutrino flavor in this analysis.
If it is smaller than one, the infinite-track hypothesis achieves a better fit to the data than the cascade-only hypothesis.

\subsubsection{Energy reconstruction}
\label{sec:leera}
The energy reconstruction runs as a separate step after the zenith angle reconstruction.
In contrast to \textsc{SANTA}, the Low-Energy Energy Reconstruction Algorithm (LEERA)\cite{Terliuk2018Measurement} fits a combined hypothesis consisting of a cascade and a finite-length track originating at the same point of the cascade.
Both the cascade and the track are constrained to move only along the infinite track that has been fit in the zenith reconstruction.
This means that the model fit in the energy reconstruction is fully characterized by the shift of the vertex along the infinite track, the length of the finite track (which is linearly related to the track energy), and the energy of the cascade.
Given these parameters, the expected light yield for all DOMs is calculated using the so-called \emph{photonics tables}.
The tables consist of B-spline coefficients that have been fit to simulated photon propagation for cascades and 3~m long tracks segments at different depths and directions inside the IceCube array to give a (time-dependent) expectation value for the photon count at arbitrary positions inside the detector.
The expectation of an arbitrarily long track is calculated by chaining the 3~m segments together that fully cover the desired track length, and scaling the amplitude of the last segment by the remainder of the division of the desired length by the length of the segments.
Given these expectation values as a function of event parameters, $\lambda_i(\theta)$, for every DOM, $i$, a simple Poisson ''hit vs.
no-hit'' log-likelihood is calculated as
\begin{equation}
    \log(\mathcal{L}) = \sum_{i\in\mathrm{DOMs\;without\;hits}} e^{-\lambda_i(\theta)} + \sum_{i\in\mathrm{DOMs\;with\;hits}} (1 - e^{-\lambda_i(\theta)})\;.
    \label{eq:leera-llh}
\end{equation}
This likelihood is maximized under the hypothesis that the shift, track length, and cascade energy are all free parameters, and under the alternative hypothesis where the track length is fixed to zero, the latter of which corresponds to a cascade-only hypothesis. The difference between these two log-likelihoods provides a measure of the degree to which the combined track and cascade hypothesis fits the observed data better than a track-only hypothesis. It is one of the inputs that is used in a BDT to calculate an overall score of how track-like an observed event signature is. The median relative error in the reconstructed total energy (that is, the sum of the track energy and the cascade energy) is shown in \reffig{leera-resolution}. As with the zenith angle reconstruction, the relative error is smallest for \numucc-events. This is expected, since these events fit the hypothesis of an initial cascade combined with a finite track the best. The second-best resolution is achieved for $\nu_{e,\mathrm{CC}}$-events, while it is poorest for $\nu_{\tau,\mathrm{CC}}$ and neutral-current events. This is explained by the fact that the expected light yield that is put in Equation~\ref{eq:leera-llh} is based on the assumption that all particles that are produced in the interaction are visible to the detector. While this assumption is a good approximation for $\nu_{e,\mathrm{CC}}$-events, it does not hold for hadronic cascades that contain some neutral components as discussed in chapter~\ref{sec:had-showers}.
The true energy of the primary particles that produce hadronic cascades is therefore systematically under-estimated and has a larger uncertainty.
This additional uncertainty is fundamentally irreducible, because it is not possible to distinguish the signatures of hadronic and electromagnetic showers.

\begin{figure}
    \centering
    \includegraphics[width=0.8\linewidth]{figures/icecube/reconstruction/leera/leera_absolute_error_final.pdf}
    \caption{Median fractional error on the reconstructed energy at the final level of the sample selection as a function of neutrino energy.}
    \label{fig:leera-resolution}
\end{figure}

\subsection{Signature Classification}
\label{sec:pid}
In addition to the energy and zenith angle, the measurement presented in this thesis requires a score that separates track-like \numucc-events from other types of interaction.
While previous analyses used only single variables such as the reconstructed track length to differentiate between tracks and cascades~\cite{deepcore_sterile_2017, Aartsen_2015,IceCube:2019dqi}, the analysis presented in this thesis uses several variables as input into a Boosted Decision-Tree (BDT) to compute a score for how track-like the observed signature is.
The BDT classifier is taken from the \texttt{scikit-learn}\cite{scikit-learn} package and trained to classify between tracks and cascades using the following input variables:
\begin{itemize}
    \item SANTA $\chi^2\textrm{-ratio}$, defined as  $\frac{(\chi^{2}/\mathrm{d.o.f.})_{\mathrm{track}}}{(\chi^{2}/\mathrm{d.o.f.})_{\mathrm{cascade}}}$, i.e.
the ratio of goodness-of-fit metrics from each fit hypothesis in the directional reconstruction (see section \ref{sec:santa})
    \item $\Delta$LLH from energy reconstruction, defined as LLH$_\mathrm{track}-$LLH$_\mathrm{cascade}$, i.e.
the best-fit LLH value from each hypothesis
    \item Reconstructed muon track length, $L_{\mu}$
    \item Radial distance of the reconstructed interaction vertex from string 36\footnote{String 36 is approximately at the center of the array, and near to the densest region of DeepCore (see \reffig{icecube-schematic}).}, $\rho^{36}_{vertex}$
    \item Radial distance of the end-point from string 36, $\rho^{36}_{stop}$
    \item Depth of the interaction vertex, $z_{vertex}$
    \item Depth of the end point, $z_{stop}$
\end{itemize}
\begin{figure*}
    \centering
    \ref{leera_pid_legend}


    \tikzsetnextfilename{santa_pid_prefit}%
\begin{tikzpicture}
    \pgfplotstableread{figures/icecube/classification/variables/SANTA_PID.csv}\table
    \begin{groupplot}[
        xmin=-0.25,xmax=5.25,
        xmode=normal,
        xmajorgrids, ymajorgrids,
        width=0.45\linewidth,
        ylabel style={at={(-0.15,0.5)}},
        group/.cd,
        group size=1 by 2,
        xticklabels at=edge bottom,
        vertical sep=10pt
        ]
    \nextgroupplot[
        height=0.3\linewidth,
        legend cell align={left},
        legend columns=-1,
        legend to name=santa_pid_legend,
        ymode=log,
        ymin=5.937489735229232e-11, ymax=0.05052335190265824,
        ylabel=rate (Hz),
        % add magic filter to correctly handle empty bins in logarithmic y-axes:
        % If a bin-count is too low or zero, it would cause the line to be
        % interrupted, which creates artefacts and ugliness. Instead, we replace
        % these bin-counts with values that are just below the axis limit.
        % Because of the way pgfplots works, the input is the raw number but the
        % output has to be the log. Weird, I know.
        y filter/.expression={y < 5.937489735229232e-11 ? ln(5.937489735229232e-12) : ln(y)}
    ]

    \ploterrorband[muon_color]{muon}{1}
    \addlegendentry{atm. muons}

    % \ploterrorband[nue_color]{nuenuebar}{1}
    % \addlegendentry{$\nu_e + \bar{\nu}_e$}

    % \ploterrorband[numu_color]{numunumubar}{1}
    % \addlegendentry{$\nu_\mu + \bar{\nu}_\mu$}

    % \ploterrorband[nutau_color]{nutaunutaubar}{1}
    % \addlegendentry{$\nu_\tau + \bar{\nu}_\tau$}


    % alternative event breakdown by interaction
    \ploterrorband[nue_color]{nue_ccnuebar_cc}{1}
    \addlegendentry{$\nu_e + \bar{\nu}_e$, CC}
    
    \ploterrorband[numu_color]{numu_ccnumubar_cc}{1}
    \addlegendentry{$\nu_\mu + \bar{\nu}_\mu$, CC}
    
    \ploterrorband[nutau_color]{nutau_ccnutaubar_cc}{1}
    \addlegendentry{$\nu_\tau + \bar{\nu}_\tau$, CC}
    
    \ploterrorband[nc_color]{nuall_ncnuallbar_nc}{1}
    \addlegendentry{all $\nu$, NC}

    \ploterrorband{total_mc}{1}
    \addlegendentry{total MC}

    \ploterrorbar{data}
    \addlegendentry{data}

    \nextgroupplot[
        height=0.2\linewidth,
        ymin=-0.1, ymax=1.3,
        ylabel=rate/total MC,
        xlabel=SANTA $\chi^2\textrm{-ratio}$
    ]
   
    \ploterrorbar{data_mc_ratio}
    \plotratioerrorband[muon_color]{muon}{total_mc}
    \plotratioerrorband[nue_color]{nue_ccnuebar_cc}{total_mc}
    \plotratioerrorband[numu_color]{numu_ccnumubar_cc}{total_mc}
    \plotratioerrorband[nutau_color]{nutau_ccnutaubar_cc}{total_mc}
    \plotratioerrorband[nc_color]{nuall_ncnuallbar_nc}{total_mc}
    \end{groupplot}
\end{tikzpicture}

    \tikzsetnextfilename{leera_pid_prefit}%
\begin{tikzpicture}
    \pgfplotstableread{figures/icecube/classification/variables/LEERA_PID.csv}\table
    \begin{groupplot}[
        xmin=-22.0,xmax=22.0,
        xmode=normal,
        xmajorgrids, ymajorgrids,
        width=0.45\linewidth,
        ylabel style={at={(-0.15,0.5)}},
        group/.cd,
        group size=1 by 2,
        xticklabels at=edge bottom,
        vertical sep=10pt
        ]
    \nextgroupplot[
        height=0.3\linewidth,
        legend cell align={left},
        legend columns=-1,
        legend to name=leera_pid_legend,
        ymode=log,
        ymin=4.429678646618006e-09, ymax=5e-3,
        ylabel=rate (Hz),
        % add magic filter to correctly handle empty bins in logarithmic y-axes:
        % If a bin-count is too low or zero, it would cause the line to be
        % interrupted, which creates artefacts and ugliness. Instead, we replace
        % these bin-counts with values that are just below the axis limit.
        % Because of the way pgfplots works, the input is the raw number but the
        % output has to be the log. Weird, I know.
        y filter/.expression={y < 4.429678646618006e-09 ? ln(4.429678646618006e-10) : ln(y)}
    ]

    \ploterrorband[muon_color]{muon}{1}
    \addlegendentry{atm. muons}

    % \ploterrorband[nue_color]{nuenuebar}{1}
    % \addlegendentry{$\nu_e + \bar{\nu}_e$}

    % \ploterrorband[numu_color]{numunumubar}{1}
    % \addlegendentry{$\nu_\mu + \bar{\nu}_\mu$}

    % \ploterrorband[nutau_color]{nutaunutaubar}{1}
    % \addlegendentry{$\nu_\tau + \bar{\nu}_\tau$}


    % alternative event breakdown by interaction
    \ploterrorband[nue_color]{nue_ccnuebar_cc}{1}
    \addlegendentry{$\nu_e + \bar{\nu}_e$, CC}

    \ploterrorband[numu_color]{numu_ccnumubar_cc}{1}
    \addlegendentry{$\nu_\mu + \bar{\nu}_\mu$, CC}

    \ploterrorband[nutau_color]{nutau_ccnutaubar_cc}{1}
    \addlegendentry{$\nu_\tau + \bar{\nu}_\tau$, CC}

    \ploterrorband[nc_color]{nuall_ncnuallbar_nc}{1}
    \addlegendentry{all $\nu$, NC}

    \ploterrorband{total_mc}{1}
    \addlegendentry{total MC}

    \ploterrorbar{data}
    \addlegendentry{data}

    \nextgroupplot[
        height=0.2\linewidth,
        ymin=-0.1, ymax=1.3,
        ylabel=rate/total MC,
        xlabel=LLH$_\mathrm{track}-$LLH$_\mathrm{cascade}$
    ]

    \ploterrorbar{data_mc_ratio}

    \plotratioerrorband[muon_color]{muon}{total_mc}
    \plotratioerrorband[nue_color]{nue_ccnuebar_cc}{total_mc}
    \plotratioerrorband[numu_color]{numu_ccnumubar_cc}{total_mc}
    \plotratioerrorband[nutau_color]{nutau_ccnutaubar_cc}{total_mc}
    \plotratioerrorband[nc_color]{nuall_ncnuallbar_nc}{total_mc}

    \end{groupplot}
\end{tikzpicture}

    \caption{Distribution and data/MC comparison for the two most important input variables into the classification BDT.}
    \label{fig:bdt-input-vars}
\end{figure*}
Of these variables, the SANTA $\chi^2\textrm{-ratio}$ and $\Delta$LLH contribute the most to the final score.
Their distributions and comparison between data and simulation can be seen in \reffig{bdt-input-vars} at the L5 selection level, where neutrinos are weighted with \textsc{NuFit}~4.0\cite{nufit40} global fit parameters.
The training data consists of simulated $\nu_e$-CC interactions and neutral-current interactions representing cascades, and $\nu_\mu$-CC interactions representing tracks.
Tau neutrino interactions are not included in the training data in order to avoid confusion due to the 17\% of $\nu_\tau$-CC interactions that produce track-like signatures.
The training samples are weighted to approximate the neutrino flux expected from the HKKM model~\sidecite{Honda:2015fha} \emph{without} oscillations.
This is done to avoid imprinting the event distributions at certain values of the oscillation parameters into the trained classifier.
The distributions for these variables for tracks and cascades as they were used in training can be found in the Appendix~\ref{sec:apx-pidvars}.
Only half of the available simulation is used for training, while the other half is held out to validate that the classifier generalized to events that it has not seen during training.
The output score of the classifier is referred to as \emph{particle-ID} (PID) and ranges from zero (very cascade-like) to one (very track-like).
The distribution of the PID score for simulated neutrino interactions is shown in \reffig{pid-score}, broken down by flavor and interaction type.
The distributions are individually normalized to help visualize the shape differences between the different neutrino interactions.
The distributions for all interaction types show a large peak around a probability score of 0.5, suggesting that the event signature cannot be clearly classified for the majority of events.
A second peak exists only in the distribution of $\nu_\mu$-CC events close to a score of one, meaning that there exists a population of these events that can be very clearly classified as being track-like.
There also exists some excess of high PID values in the distribution of $\nu_\tau$-CC events corresponding to those events where the decay of the tauon produces a muon.
Notably, there is no population of events that can be cleanly classified as a cascade event, i.e., there are no PID scores close to zero.
The reason for this is that the two classes are nested hypotheses, one containing only a cascade and the other containing a combination of a cascade and a track, and it is never possible to prove that a cascade-like signature does not contain at least a short track segment.
\begin{figure}
    \centering
    %\includegraphics[width=0.8\linewidth]{figures/icecube/classification/bdt_score_normalized.pdf}
    \tikzsetnextfilename{bdt_score_normalized}%
\begin{tikzpicture}

    \pgfplotstableread{figures/icecube/classification/gbm_normalized_nu_hists_nufit22.csv}\table
    \begin{axis}[
        xmin=0,xmax=1.0,
        xmajorgrids, ymajorgrids,
        width=0.7\linewidth,
        height=0.6\linewidth,
        ymode=log,
        ymin=1e-02, ymax=100,
        ylabel=distribution density,
        xlabel=BDT score,
        legend cell align={left},
        legend columns=2,
    ]

    \addplot[const plot, nue_color, thick] table [x=bin_edges, y=nue_cc] from \table;
    \addlegendentry{$\nu_e + \bar{\nu}_e$, CC}
    \addplot[const plot, numu_color, thick] table [x=bin_edges, y=numu_cc] from \table;
    \addlegendentry{$\nu_\mu + \bar{\nu}_\mu$, CC}
    \addplot[const plot, nutau_color, thick] table [x=bin_edges, y=nutau_cc] from \table;
    \addlegendentry{$\nu_\tau + \bar{\nu}_\tau$, CC}
    \addplot[const plot, nc_color, thick] table [x=bin_edges, y=nu_nc] from \table;
    \addlegendentry{all $\nu$, NC}
    
    \end{axis}
\end{tikzpicture}

    \caption{PID score distribution for simulated neutrino events at the final level of the event selection, weighted according to the HKKM flux model~\cite{Honda:2015fha} and neutrino oscillations with \textsc{NuFit}~4.0\cite{nufit40} global fit parameters, normalized to unity. The BDT score ranges from the most cascade-like at 0 to the most track-like event signature at 1.}
    \label{fig:pid-score}
\end{figure}

\subsection{Final Sample Selection and Binning}
\label{sec:final-sample-binning}
After the reconstruction and classification step, several final cut variables are applied to reduce the background of atmospheric muons to only a few percent, and to remove a small number of events from data containing coincident muons.
These cuts are:
\begin{itemize}
    \item The reconstruction of energy and zenith angle has to be successful.
This requires, in particular, that at least five hits remain after the hit cleaning procedure described in \refsec{santa}.
    \item The reconstructed energy should be in the range between \SI{6}{\giga\electronvolt} and \SI{156}{\giga\electronvolt}.
    \item Require a minimum PID score (see \refsec{pid}) of 0.55 to only include at least somewhat track-like events.
    \item Reconstructed $\cos(\theta_z) < 0.1$ to remove events that enter the detector from above the horizon.
    \item Require a minimum goodness-of-fit of the zenith reconstruction with $\chi^2_{\mathrm{mod}}/\mathrm{d.o.f.} < 50$.
    \item A tighter cut on the L4 muon BDT score (see \refsec{level4-selection}) of $P_\nu > 0.97$.
    \item Fewer than eight hits in the outermost strings of the IceCube array, and a positive "z-travel" value for hits in the uppermost 15 layers of DOMs in the (non-DeepCore) IceCube array.
The "z-travel" value for a given sequence of hits is calculated by subtracting the mean value of the z-coordinate of the first quartile of hits from the mean z-coordinate of all hits.
\end{itemize}

\begin{figure}
    \centering
    \tikzsetnextfilename{z_travel_top15}%
\begin{tikzpicture}
    \pgfplotstableread{figures/icecube/selection/Level7/z_travel_top15.csv}\table

    \begin{axis}[
    	xmin=-150,xmax=150,
        xmode=normal,
        xmajorgrids, ymajorgrids,
        width=0.75\linewidth,
        % ylabel style={at={(-0.15,0.5)}},
        height=0.5\linewidth,
        legend cell align={left},
        %legend to name=z_travel_legend,
        legend columns=-1,
        ymode=log,
        ymin=2e-9, ymax=5e-6,
        y filter/.expression={y < 1e-10 ? ln(1e-11) : ln(y)},
        ylabel=rate (Hz),
        xlabel=z-travel in upper 15 IceCube DOMs
    ]

    % \ploterrorband[muon_color]{muon}{1}
    % \addlegendentry{atm. muons}

    % \ploterrorband[nue_color]{nuenuebar}{1}
    % \addlegendentry{$\nu_e + \bar{\nu}_e$}
    %
    % \ploterrorband[numu_color]{numunumubar}{1}
    % \addlegendentry{$\nu_\mu + \bar{\nu}_\mu$}
    %
    % \ploterrorband[nutau_color]{nutaunutaubar}{1}
    % \addlegendentry{$\nu_\tau + \bar{\nu}_\tau$}


    \ploterrorband{total_mc}{1}
    \addlegendentry{total MC}


	% draw cut
    % use "axis cs" to give coordinates in the data coordinate system!
    \draw[thick,dashed] (axis cs:0,1e-9) -- (axis cs:0,5e-5);
    \draw[-stealth, very thick] (axis cs:0,3e-9)  -- node[anchor=south]{\footnotesize\sffamily signal} (axis cs:50,3e-9);

    \ploterrorbar{data}
    \addlegendentry{data}

	\end{axis}
\end{tikzpicture}

    \caption{Distribution of the "z-travel" variable calculated for the uppermost 15 layers of IceCube DOMs. Only events with at least 4 hits in the uppermost 15 layers of DOMs are included in the histogram.}
    \label{fig:z_travel_distribution}
\end{figure}

% \begin{figure}
%     \centering
%     \tikzsetnextfilename{santa_chi2dof_l7}%
\begin{tikzpicture}
    \pgfplotstableread{figures/icecube/selection/Level7/santa_chi2dof.csv}\table
    \begin{groupplot}[
        xmin=3,xmax=300,
        xmode=log,
        xmajorgrids, ymajorgrids,
        width=0.45\linewidth,
        ylabel style={at={(-0.15,0.5)}},
        group/.cd,
        group size=1 by 2,
        xticklabels at=edge bottom,
        vertical sep=10pt
        ]
    \nextgroupplot[
        height=0.3\linewidth,
        legend cell align={left},
        legend columns=3,
        legend to name=santa_chi2dof_legend,
        ymode=log,
        ymin=2e-8, ymax=5e-3,
        ylabel=rate (Hz),
    ]

    \ploterrorband[muon_color]{muon}{1}
    \addlegendentry{atm. muons}

    \ploterrorband[nue_color]{nuenuebar}{1}
    \addlegendentry{$\nu_e + \bar{\nu}_e$}

    \ploterrorband[numu_color]{numunumubar}{1}
    \addlegendentry{$\nu_\mu + \bar{\nu}_\mu$}

    \ploterrorband[nutau_color]{nutaunutaubar}{1}
    \addlegendentry{$\nu_\tau + \bar{\nu}_\tau$}

    % alternative  event breakdown by interaction

    % \ploterrorband[nue_color]{nue_ccnuebar_cc}{1}
    % \addlegendentry{$\nu_e + \bar{\nu}_e$, CC}

    % \ploterrorband[numu_color]{numu_ccnumubar_cc}{1}
    % \addlegendentry{$\nu_\mu + \bar{\nu}_\mu$, CC}

    % \ploterrorband[nutau_color]{nutau_ccnutaubar_cc}{1}
    % \addlegendentry{$\nu_\tau + \bar{\nu}_\tau$, CC}

    % \ploterrorband[nc_color]{nuall_ncnuallbar_nc}{1}
    % \addlegendentry{all $\nu$, NC}

    \ploterrorband{total_mc}{1}
    \addlegendentry{total MC}

    % draw cut
    % use "axis cs" to give coordinates in the data coordinate system!
    \draw[thick,dashed] (axis cs:50,1e-8) -- (axis cs:50,5e-4);
    \draw[-stealth, very thick] (axis cs:50,1e-7)  -- node[anchor=south]{\footnotesize\sffamily signal} (axis cs:20,1e-7);

    \ploterrorbar{data}
    \addlegendentry{data}

    \nextgroupplot[
        height=0.2\linewidth,
        ymin=-0.1, ymax=1.8,
        ylabel=rate/total MC,
        xlabel=SANTA $\chi^2 / \mathrm{d.o.f.}$\strut
    ]

    \ploterrorbar{data_mc_ratio}
    \plotratioerrorband[muon_color]{muon}{total_mc}
    \plotratioerrorband[nue_color]{nuenuebar}{total_mc}
    \plotratioerrorband[numu_color]{numunumubar}{total_mc}
    \plotratioerrorband[nutau_color]{nutaunutaubar}{total_mc}
    %\plotratioerrorband[nc_color]{nuall_ncnuallbar_nc}{total_mc}

    \end{groupplot}
\end{tikzpicture}

%     \caption{Distribution of the SANTA goodness-of-fit variable.}
%     \label{fig:santa_chi2dof_distribution}
% \end{figure}


\begin{figure*}
    \centering
    \ref{reco_coszen_legend}\par
    \tikzsetnextfilename{santa_chi2dof_l7}%
\begin{tikzpicture}
    \pgfplotstableread{figures/icecube/selection/Level7/santa_chi2dof.csv}\table
    \begin{groupplot}[
        xmin=3,xmax=300,
        xmode=log,
        xmajorgrids, ymajorgrids,
        width=0.45\linewidth,
        ylabel style={at={(-0.15,0.5)}},
        group/.cd,
        group size=1 by 2,
        xticklabels at=edge bottom,
        vertical sep=10pt
        ]
    \nextgroupplot[
        height=0.3\linewidth,
        legend cell align={left},
        legend columns=3,
        legend to name=santa_chi2dof_legend,
        ymode=log,
        ymin=2e-8, ymax=5e-3,
        ylabel=rate (Hz),
    ]

    \ploterrorband[muon_color]{muon}{1}
    \addlegendentry{atm. muons}

    \ploterrorband[nue_color]{nuenuebar}{1}
    \addlegendentry{$\nu_e + \bar{\nu}_e$}

    \ploterrorband[numu_color]{numunumubar}{1}
    \addlegendentry{$\nu_\mu + \bar{\nu}_\mu$}

    \ploterrorband[nutau_color]{nutaunutaubar}{1}
    \addlegendentry{$\nu_\tau + \bar{\nu}_\tau$}

    % alternative  event breakdown by interaction

    % \ploterrorband[nue_color]{nue_ccnuebar_cc}{1}
    % \addlegendentry{$\nu_e + \bar{\nu}_e$, CC}

    % \ploterrorband[numu_color]{numu_ccnumubar_cc}{1}
    % \addlegendentry{$\nu_\mu + \bar{\nu}_\mu$, CC}

    % \ploterrorband[nutau_color]{nutau_ccnutaubar_cc}{1}
    % \addlegendentry{$\nu_\tau + \bar{\nu}_\tau$, CC}

    % \ploterrorband[nc_color]{nuall_ncnuallbar_nc}{1}
    % \addlegendentry{all $\nu$, NC}

    \ploterrorband{total_mc}{1}
    \addlegendentry{total MC}

    % draw cut
    % use "axis cs" to give coordinates in the data coordinate system!
    \draw[thick,dashed] (axis cs:50,1e-8) -- (axis cs:50,5e-4);
    \draw[-stealth, very thick] (axis cs:50,1e-7)  -- node[anchor=south]{\footnotesize\sffamily signal} (axis cs:20,1e-7);

    \ploterrorbar{data}
    \addlegendentry{data}

    \nextgroupplot[
        height=0.2\linewidth,
        ymin=-0.1, ymax=1.8,
        ylabel=rate/total MC,
        xlabel=SANTA $\chi^2 / \mathrm{d.o.f.}$\strut
    ]

    \ploterrorbar{data_mc_ratio}
    \plotratioerrorband[muon_color]{muon}{total_mc}
    \plotratioerrorband[nue_color]{nuenuebar}{total_mc}
    \plotratioerrorband[numu_color]{numunumubar}{total_mc}
    \plotratioerrorband[nutau_color]{nutaunutaubar}{total_mc}
    %\plotratioerrorband[nc_color]{nuall_ncnuallbar_nc}{total_mc}

    \end{groupplot}
\end{tikzpicture}

    \tikzsetnextfilename{reco_coszen_l7}%
\begin{tikzpicture}
    \pgfplotstableread{figures/icecube/selection/Level7/reco_coszen.csv}\table
    \begin{groupplot}[
        xmin=-1.1,xmax=1.1,
        xmode=normal,
        xmajorgrids, ymajorgrids,
        width=0.45\linewidth,
        ylabel style={at={(-0.15,0.5)}},
        group/.cd,
        group size=1 by 2,
        xticklabels at=edge bottom,
        vertical sep=10pt
        ]
    \nextgroupplot[
        height=0.3\linewidth,
        legend cell align={left},
        legend columns=-1,
        legend to name=reco_coszen_legend,
        ymode=log,
        ymin=2e-7, ymax=5e-4,
        ylabel=rate (Hz),
        % add magic filter to correctly handle empty bins in logarithmic y-axes:
        % If a bin-count is too low or zero, it would cause the line to be
        % interrupted, which creates artefacts and ugliness. Instead, we replace
        % these bin-counts with values that are just below the axis limit.
        % Because of the way pgfplots works, the input is the raw number but the
        % output has to be the log. Weird, I know.
        y filter/.expression={y < 8.7587858920556e-09 ? ln(8.758785892055601e-10) : ln(y)}
    ]

    \ploterrorband[muon_color]{muon}{1}
    \addlegendentry{atm. muons}

    \ploterrorband[nue_color]{nuenuebar}{1}
    \addlegendentry{$\nu_e + \bar{\nu}_e$}

    \ploterrorband[numu_color]{numunumubar}{1}
    \addlegendentry{$\nu_\mu + \bar{\nu}_\mu$}

    \ploterrorband[nutau_color]{nutaunutaubar}{1}
    \addlegendentry{$\nu_\tau + \bar{\nu}_\tau$}


    % alternative event breakdown by interaction
    % \ploterrorband[nue_color]{nue_ccnuebar_cc}{1}
    % \addlegendentry{$\nu_e + \bar{\nu}_e$, CC}
    %
    % \ploterrorband[numu_color]{numu_ccnumubar_cc}{1}
    % \addlegendentry{$\nu_\mu + \bar{\nu}_\mu$, CC}
    %
    % \ploterrorband[nutau_color]{nutau_ccnutaubar_cc}{1}
    % \addlegendentry{$\nu_\tau + \bar{\nu}_\tau$, CC}
    %
    % \ploterrorband[nc_color]{nuall_ncnuallbar_nc}{1}
    % \addlegendentry{all $\nu$, NC}

    \ploterrorband{total_mc}{1}
    \addlegendentry{total MC}

    \ploterrorbar{data}
    \addlegendentry{data}


    % draw cut
    % use "axis cs" to give coordinates in the data coordinate system!
    \draw[thick,dashed] (axis cs:0.1,1e-9) -- (axis cs:0.1,5e-4);
    \draw[-stealth, very thick] (axis cs:0.1,1e-4)  -- node[anchor=south]{\footnotesize\sffamily signal} (axis cs:-0.4,1e-4);

    \nextgroupplot[
        height=0.2\linewidth,
        ymin=-0.1, ymax=1.3,
        ylabel=rate/total MC,
        xlabel=reconstructed $\cos(\theta_{\mathrm{zenith}})$\strut
    ]

    \ploterrorbar{data_mc_ratio}
    \plotratioerrorband[muon_color]{muon}{total_mc}
    \plotratioerrorband[nue_color]{nuenuebar}{total_mc}
    \plotratioerrorband[numu_color]{numunumubar}{total_mc}
    \plotratioerrorband[nutau_color]{nutaunutaubar}{total_mc}
    \end{groupplot}
\end{tikzpicture}

    \caption{Distribution of the SANTA goodness-of-fit variable and the reconstructed zenith angle at L5 of the event selection process.}
    \label{fig:final_cut_vars_l5}
\end{figure*}

The cuts on energy, zenith angle, and PID define the range of the binning that will be used in the analysis.
The cut on the zenith angle in particular is applied not only to reduce the background of atmospheric muons, but also to remove the phase space of neutrino events where muons that are produced in the same air shower that also produced the neutrino cause it to be vetoed by the muon filter cuts.
This effect is referred to as the "self-veto" effect and would lead to a disagreement between data and simulation since coincident muons are never simulated.
The distribution of the cosine of the reconstructed zenith angle is shown in the right panel of \reffig{final_cut_vars_l5}, and it is apparent from the distributions that atmospheric muons dominate in the region of down-going events.

The requirement on the SANTA goodness-of-fit not only ensures that the included events are well-reconstructed, but also reduces the fraction of muons in the sample, as can be seen from the distributions shown in the left panel of \reffig{final_cut_vars_l5}.
The number of hits on the outermost strings and the "z-travel" variable calculated for hits in the uppermost 15 layers of IceCube DOMs are indicators of muons that hit the detector within the trigger window of a neutrino event.
Such coincidences are entirely absent in simulation, which becomes especially apparent in the distribution of the "z-travel" variable shown in \reffig{z_travel_distribution}, where a negative value indicates a down-going signal.
After the application of all these cuts, the data sample consists of 21,914 well-reconstructed, track-like events with an expected background from atmospheric muons of only $\sim 2\%$ as shown in \reftab{muon-rejection-cut-rates}.
For the purpose of the oscillation measurements presented in this work, both data and simulation sets are binned in reconstructed energy ($E_{\rm reco}$), cosine of the reconstructed zenith angle ($\cos(\theta_z)$), and PID as follows:

\begin{itemize}
    \item $E_{\rm reco}$: 11 bins spanning the range from 6.31~GeV to 158.49~GeV, the two bins with the highest energy are merged.
    \item $\cos(\theta_z)$: 10 bins spanning the range from -1 to 0.1
    \item PID: One bin between 0.55 and 0.75, and one bin between 0.75 and 1.0.
\end{itemize}

The lower PID bin between 0.55 and 0.75 consists to 69\%  (pre-fit MC estimate) of charged-current $\nu_\mu + \bar{\nu}_\mu$ events and is referred to as the \emph{mixed} channel, while the higher PID channel between 0.75 and 1.0 consists to 94\% of charged-current $\nu_\mu + \bar{\nu}_\mu$ events and is referred to as the \emph{tracks} channel.
The expectation values of the histogram in both PID channels is shown in \reffig{nominal-hist-null-hypo} at current global best-fit parameters for standard three-flavor oscillations.
The expectation values are calculated from Monte-Carlo (MC) simulation that is described in detail in \refsec{event-simulation}.
The detailed breakdown of event counts in the final data sample by particle type and PID channel is given in \reftab{event-rate}.

\begin{figure}
    \centering
    \includegraphics[width=0.95\textwidth]{figures/measurement/simulation_and_data/binning/plot_maps_total.pdf}
    \caption{Expected event counts in 7.5 years of live time assuming no sterile mixing and NuFit~4.0~\cite{nufit40} global best fit parameters at Normal Ordering.}
    \label{fig:nominal-hist-null-hypo}
\end{figure}

\begin{table}
\caption{Successively applied cuts on the data sample. The bottom row corresponds to the final rates in the sample after all cuts have been applied. The total rate of the data and simulation does not match, which is expected since there is a large amount of uncertainty in the total normalization. Muon contamination is the muon rate divided by the total event rate. Numbers calculated at the NuFit 4.0 global best-fit point.}
\centering
\begin{tabular}{@{}lrrrr@{}}\toprule
& \multicolumn{3}{c}{rate ($\mu$Hz)} & \\ \cmidrule{2-4}
condition                              & {$\nu$ (sim)} & {$\mu$ (sim)} & {data} & {$\mu$ fraction} \\ \midrule
has SANTA reconstruction         & 957  & 314  & 1183 & 24.7 \%  \\
$\SI{6}{\giga\electronvolt} < E_\mathrm{reco}  < \SI{156}{\giga\electronvolt}$         & 862  & 311  & 1095 & 26.5 \%  \\
BDT score $>0.55$                      & 232  & 117  &  336 & 33.5 \%  \\
$\cos(\theta_{\mathrm{reco}}) < 0.1$   & 175  &  17  &  177 & 8.8 \%   \\
SANTA $\chi^2/\mathrm{d.o.f.} < 50$    & 164  &  12  &  161 & 6.6 \%   \\
L4 muon $\nu$ prob $> 0.97$            & 101  &   2  &   93 & 2.1 \% \\
\midrule\addlinespace
coinc. $\mu$ cuts (final rate) & \textbf{101} & \textbf{2} & \textbf{93} & \textbf{2.1 \%} \\ \bottomrule
\end{tabular}
\label{tab:muon-rejection-cut-rates}
\end{table}

%\begin{figure*}
%    \centering
%    \ref{reco_coszen_prefit_legend}\par
%    \begin{tikzpicture}
    \pgfplotstableread{figures/icecube/selection/final_sample_prefit/reco_coszen.csv}\table
    \begin{groupplot}[
        xmin=-1.055,xmax=0.15500000000000003,
        xmode=normal,
        xmajorgrids, ymajorgrids,
        width=0.45\linewidth,
        ylabel style={at={(-0.15,0.5)}},
        group/.cd,
        group size=1 by 2,
        xticklabels at=edge bottom,
        vertical sep=10pt
        ]
    \nextgroupplot[
        height=0.3\linewidth,
        legend cell align={left},
        legend columns=-1,
        legend to name=reco_coszen_prefit_legend,
        ymode=log,
        ymin=2.670138976548827e-09, ymax=3e-5,
        ylabel=rate (Hz),
        ytick distance=1e1,
        % add magic filter to correctly handle empty bins in logarithmic y-axes:
        % If a bin-count is too low or zero, it would cause the line to be
        % interrupted, which creates artefacts and ugliness. Instead, we replace
        % these bin-counts with values that are just below the axis limit.
        % Because of the way pgfplots works, the input is the raw number but the
        % output has to be the log. Weird, I know.
        y filter/.expression={y < 2.670138976548827e-09 ? ln(2.6701389765488273e-10) : ln(y)}
    ]

    \ploterrorband[muon_color]{muon}{1}
    \addlegendentry{atm. muons}

    \ploterrorband[nue_color]{nuenuebar}{1}
    \addlegendentry{$\nu_e + \bar{\nu}_e$}

    \ploterrorband[numu_color]{numunumubar}{1}
    \addlegendentry{$\nu_\mu + \bar{\nu}_\mu$}

    \ploterrorband[nutau_color]{nutaunutaubar}{1}
    \addlegendentry{$\nu_\tau + \bar{\nu}_\tau$}


    % alternative event breakdown by interaction
    % \ploterrorband[nue_color]{nue_ccnuebar_cc}{1}
    % \addlegendentry{$\nu_e + \bar{\nu}_e$, CC}
    %
    % \ploterrorband[numu_color]{numu_ccnumubar_cc}{1}
    % \addlegendentry{$\nu_\mu + \bar{\nu}_\mu$, CC}
    %
    % \ploterrorband[nutau_color]{nutau_ccnutaubar_cc}{1}
    % \addlegendentry{$\nu_\tau + \bar{\nu}_\tau$, CC}
    %
    % \ploterrorband[nc_color]{nuall_ncnuallbar_nc}{1}
    % \addlegendentry{all $\nu$, NC}

    \ploterrorband{total_mc}{1}
    \addlegendentry{total MC}

    \ploterrorbar{data}
    \addlegendentry{data}


    \nextgroupplot[
        height=0.2\linewidth,
        ymin=0.5, ymax=1.5,
        ylabel=data/MC ratio,
        xlabel=reconstructed $\cos(\theta_{\mathrm{zenith}})$
    ]

    \ploterrorband{data_mc_ratio}{1}
    \end{groupplot}
\end{tikzpicture}

%    \tikzsetnextfilename{final_level_postfit_three_flav_reco_energy}%
\begin{tikzpicture}
    \pgfplotstableread{figures/measurement/three_flavor/results/data_mc_post_fit/reco_energy.csv}\table
    \begin{groupplot}[
        xmin=5.362591587787688,xmax=185.61920737481475,
        xmode=log,
        xmajorgrids, ymajorgrids,
        width=0.45\linewidth,
        ylabel style={at={(-0.15,0.5)}},
        group/.cd,
        group size=1 by 2,
        xticklabels at=edge bottom,
        vertical sep=10pt
        ]
    \nextgroupplot[
        height=0.3\linewidth,
        legend cell align={left},
        legend columns=3,
        legend to name=reco_energy_postfit_threeflav_legend,
        ymode=log,
        ymin=3e-8, ymax=3e-5,
        ylabel=rate (Hz),
        % add magic filter to correctly handle empty bins in logarithmic y-axes:
        % If a bin-count is too low or zero, it would cause the line to be
        % interrupted, which creates artefacts and ugliness. Instead, we replace
        % these bin-counts with values that are just below the axis limit.
        % Because of the way pgfplots works, the input is the raw number but the
        % output has to be the log. Weird, I know.
        % y filter/.expression={y < 1.129498430227963e-09 ? ln(1.1294984302279631e-10) : ln(y)}
        y filter/.expression={y < \pgfkeysvalueof{/pgfplots/ymin} ? ln(\pgfkeysvalueof{/pgfplots/ymin}) - 1 : ln(y)}
    ]

    \ploterrorband[muon_color]{muon}{1}
    \addlegendentry{atm. muons}

    \ploterrorband[nue_color]{nuenuebar}{1}
    \addlegendentry{$\nu_e + \bar{\nu}_e$}

    \ploterrorband[numu_color]{numunumubar}{1}
    \addlegendentry{$\nu_\mu + \bar{\nu}_\mu$}

    \ploterrorband[nutau_color]{nutaunutaubar}{1}
    \addlegendentry{$\nu_\tau + \bar{\nu}_\tau$}


    % alternative event breakdown by interaction
    % \ploterrorband[nue_color]{nue_ccnuebar_cc}{1}
    % \addlegendentry{$\nu_e + \bar{\nu}_e$, CC}
    %
    % \ploterrorband[numu_color]{numu_ccnumubar_cc}{1}
    % \addlegendentry{$\nu_\mu + \bar{\nu}_\mu$, CC}
    %
    % \ploterrorband[nutau_color]{nutau_ccnutaubar_cc}{1}
    % \addlegendentry{$\nu_\tau + \bar{\nu}_\tau$, CC}
    %
    % \ploterrorband[nc_color]{nuall_ncnuallbar_nc}{1}
    % \addlegendentry{all $\nu$, NC}

    \ploterrorband{total_mc}{1}
    \addlegendentry{total MC}

    \ploterrorbar{data}
    \addlegendentry{data}


    \nextgroupplot[
        height=0.2\linewidth,
        ymin=0.7, ymax=1.3,
        ylabel=rate/total MC,
        xlabel=reconstructed energy (GeV)
    ]

    \ploterrorbar{data_mc_ratio}
    % \plotratioerrorband[muon_color]{muon}{total_mc}
    % \plotratioerrorband[nue_color]{nuenuebar}{total_mc}
    % \plotratioerrorband[numu_color]{numunumubar}{total_mc}
    % \plotratioerrorband[nutau_color]{nutaunutaubar}{total_mc}

    \node[anchor=south, font=\footnotesize] at (axis description cs:0.5, 0.01) {$\chi^2_{\mathrm{mod}}/\mathrm{dof} = 1.19$};
    \end{groupplot}
\end{tikzpicture}

%    \caption{Distributions of reconstructed energy and zenith angle at the final level of the event selection, calculated assuming \textsc{NuFit}~4.0\cite{nufit40} global best fit oscillation parameters.}
%    \label{fig:pre-fit-energy-coszen}
%\end{figure*}


\begin{table}[htb]
\centering
\caption{Expected event rate with 8 years livetime broken down in event types and PID bins, calculated at NuFit~4.0 global best fit parameters.}
\label{tab:event-rate}
\begin{tabular}{lcrS} \toprule
Type  & PID & Event Count & {Rate ($\mathrm{\mu Hz}$)} \\ \midrule
All MC & mixed  &   11428 &   48.3\\
All MC & tracks &   12238 &   51.7\\ \midrule
${\nu_{\rm all}} + {\bar\nu_{\rm all}} \, {\rm NC} $ & mixed  &     943 &    4.0 \\
${\nu_e} + {\bar\nu_e} \, {\rm CC}                 $ & mixed  &    1704 &    7.2 \\
${\nu_\mu} + {\bar\nu_\mu} \, {\rm CC}             $ & mixed  &    7901 &   33.4 \\
${\nu_\tau} + {\bar\nu_\tau} \, {\rm CC}           $ & mixed  &     470 &    2.0 \\
muons                                                & mixed  &     410 &    1.7 \\
\midrule
${\nu_{\rm all}} + {\bar\nu_{\rm all}} \, {\rm NC} $ & tracks &     171 &    0.7 \\
${\nu_e} + {\bar\nu_e} \, {\rm CC}                 $ & tracks &     294 &    1.2 \\
${\nu_\mu} + {\bar\nu_\mu} \, {\rm CC}             $ & tracks &   11517 &   48.7 \\
${\nu_\tau} + {\bar\nu_\tau} \, {\rm CC}           $ & tracks &     162 &    0.7 \\
muons                                                & tracks &      93 &    0.4 \\
\bottomrule
\end{tabular}
\end{table}


\subsubsection{Muon Smearing}
\label{section:muon_kde}

After all the filtering steps described in section~\ref{sec:data-processing}, the muon contamination of the data sample is reduced to $\sim 2\%$ of the sample.
This reduces the statistics of muon simulation so much, that the resulting histograms become very sparse as shown in figure~\ref{fig:muon-template-no-kde}.
Such sparse histograms, in which single MC events have to serve as a stand-in for several real data events, are a poor template for what can be expected in data.
To produce a more realistic expectation of the bin counts, the muon histograms are smeared using KDEs as shown in figure \ref{fig:muon-template-with-kde}.
Since the KDE operates on events on the entire zenith and energy range, including events that fall outside the analysis binning, some events bleed into the highest $\cos(\theta_z)$ bin from further above the horizon.
The KDE kernel is mirrored at $\cos(\theta_z) = -1$ to avoid spurious disappearance of events at the edge.
The smeared muon histogram is added to the expectation values from the neutrino MC simulation to estimate the total expectation value in every bin shown in \reffig{nominal-hist-null-hypo}.

\begin{figure}[H]
    \centering
    \begin{subfigure}{0.8\textwidth}
        \centering
        \includegraphics[width=\textwidth,trim={0 0 0 0.6cm},clip]{figures/measurement/systematics/muons/muon_hist_no_kde.pdf}
        \caption{Without KDE smoothing}
        \label{fig:muon-template-no-kde}
    \end{subfigure}
    \begin{subfigure}{0.8\textwidth}
        \centering
        \includegraphics[width=\textwidth,trim={0 0 0 0.6cm},clip]{figures/measurement/systematics/muons/plot_maps_muon.pdf}
        \caption{With KDE smoothing}
        \label{fig:muon-template-with-kde}
    \end{subfigure}

    \caption{Muon template before (top) and after (bottom) the application of KDE smoothing. The shown values are the average of 20 KDE evaluations on different bootstrap samples.}
    \label{fig:muon-kde-smoothing}
\end{figure}

\subsection{Seasonal Stability}
\label{sec:sample-stability}
As the operating conditions of the DOMs are very stable after their deployment, the calibration of the DOM response described in \refsec{sim-detector-response} is performed only once per year. Such a re-calibration usually also coincides with the release of a new IceCube software package and a new \emph{season} of data taking. To ensure that the data sample is stable under these re-calibrations and software updates, the distributions of the reconstructed energy, zenith angle and PID as well as some control variables are compared between seasons using a Kolmogorov-Smirnov (KS) test\sidecite{ks-test}. The test calculates the p-value of the largest difference between the cumulative distributions of two samples under the null hypothesis that the samples are drawn from the same distribution. The results for every pair of seasons included in the data sample is shown in heat maps in \reffig{data_stability_2D_control_KS} in Appendix \refsec{ks-test-appendix}. The results show good agreement between the distributions of the different seasons in the sample.

\section{Implementation of systematic uncertainties}
\subsection{Variation of Detector Properties}
\label{sec:detector-unc}
Systematic uncertainties on the detector properties that need to be taken into account are the overall optical efficiency of the DOMs as well as the properties of the surrounding ice.
The parametrization and priors of each of these properties are informed by IceCube calibration studies.
\begin{itemize}
    \item DOM efficiency: A factor that scales the probability that a photon hitting the PMT of a DOM will produce a photo-electron that is measured by the electronics.
Nominal value is 1, prior standard deviation is 10\%.
    \item Hole ice: Two parameters describe the effect of the optical properties of the column of re-frozen ice within the bore holes in which the strings have been deployed.
The details of this parametrization is described below.
    \item Bulk ice: The over-all absorption and scattering coefficients of all ice layers are multiplied by a scaling factor.
The nominal value for ice absorption is 1.0 with a prior standard deviation of 5\%.
The nominal value for ice scattering is 1.05 with a prior standard deviation of 10\%.
\end{itemize}

In total, the uncertainties on the detector properties are modeled by five  parameters, one for the DOM efficiency, two for the hole ice model and two for the bulk ice uncertainty.
To model the effect of these parameters on the analysis histogram, several MC sets at different variations of DOM efficiency, hole ice, and bulk ice parameters are produced.
These MC sets are used to find a parametrization that will model how the distribution of events in energy, zenith and PID will change as a function of these parameters.

\subsubsection{DOM efficiency calibration}
\label{sec:domeff-calibration}
As described in \refsec{dom-daq}, the DOMs contain LEDs that are used to calibrate the detector \emph{in-situ}. However, these LEDs are not calibrated with respect to their absolute brightness and therefore are not suited for the calibration of the optical efficiency of the DOMs. Instead, minimally ionizing muons that are produced in air showers are used as a light source with a known brightness. The calibration is performed using a sample of events that pass the \emph{Minimum Bias Trigger}\cite{icecube_detector_17} and in which the reconstructed muon track stops within the instrumented volume of IceCube. The DOM efficiency is estimated by comparing the observed charges in the DOMs and the light expectation from the reconstructed muons. Multiple such calibration studies have been run \sidecite{domeff_nick}\sidecite{domeff_jake} and found variations in the optical efficiency of approximately 10\%, which is used as a prior for the measurements presented in this work.

\subsubsection{Hole Ice Parametrization}
\label{sec:hole-ice-parametrization}

The bore holes in which IceCube's strings have been deployed were drilled using hot water to melt a column of ice into which the strings with their attached optical sensors could be lowered.
This water column re-froze after deployment to form what is referred to as \emph{hole ice}\sidecite{Fiedlschuster:2019unl}.
Camera observations of this re-freezing process suggest that the hole ice is transparent near the edges of be hole and contains a bubble column in its center\sidecite{rongen2016measuring}.
The bubble column has a much shorter scattering length than the surrounding bulk ice and therefore decreases the probability of a photon entering a DOM directly from below.
The effect of the re-frozen ice column surrounding the strings can be modeled as a modification to the optical efficiency of the DOMs as a function of the incident angle of incoming photons.
In the past, many different angular acceptance curves have been produced from \emph{in-situ} calibration measurements\cite{flasher_calibration}, the best fit results of previous DeepCore oscillation analyses, in addition to the laboratory measurements that have been made in water tanks before the deployment of IceCube.
For the analysis presented in this work, a two-dimensional parametrization was developed that can approximate any of these hole ice models such that it can be used as a \emph{unified} hole ice model.
To do this, all previous angular acceptance curves are evaluated as a function of the cosine of the photon incidence angle, $\cos(\eta)$, at 100 points over the entire valid domain between -1 and 1, where 1 represents a photon entering a DOM directly from below.
The curves are furthermore normalized to an area of 1 to avoid affecting the total observed charge.
Using Principal Component Analysis\sidecite{pca}, the variations between the different models are decomposed into a mean and the most important components that explain the variance between models.
It was found that the two most important components, $p_0$ and $p_1$, describe all known hole ice models adequately.
Their effect is shown in \reffig{hole-ice-parametrization} as variations with the acceptance curve that is used as the baseline in this analysis.
The right panel of \reffig{hole-ice-parametrization} also shows where the older hole ice models are located in the space spanned by $p_0$ and $p_1$.
The laboratory measurement, which did not include any hole ice effects, notably lies far outside of the region of all other hole ice models that are all produced \emph{in-situ}.

\begin{figure*}
    \centering
    \tikzsetnextfilename{hole_ice_p0_p1}%
\begin{tikzpicture}

\pgfplotstableread{figures/measurement/systematics/detector/hole_ice/angsens_example_fluct.csv}\table
\pgfplotstableread[col sep=comma]{figures/measurement/systematics/detector/hole_ice/all_acceptance_curves.csv}\acceptancecurves

\begin{axis}[
        width=0.45\linewidth, height=0.4\linewidth,
		xmajorgrids, ymajorgrids,
		xlabel=$\cos(\eta)$, ylabel=relative optical efficiency,
		legend style={at={(0.02,0.95)}, anchor=north west},
        ytick distance=0.2,
	]
    \addplot[black, thick] table [x=cos_theta, y=baseline] \acceptancecurves;
    \addlegendentry{baseline}
    
    \addplot[vermilion, thick] table [x=cos_theta, y=nominal] \acceptancecurves;
    \addlegendentry{laboratory}
    
    % \addplot[bluishgreen, thick] table [x=cos_theta, y=bfp_three_flav] \acceptancecurves;
    % \addlegendentry{best fit}
    
    \addplot[blue, thin, name path=fluct_p0_dn, forget plot] table [x=cos_theta, y=fluct_p0_dn] \table;
    \addplot[blue, thin, name path=fluct_p0_up, forget plot] table [x=cos_theta, y=fluct_p0_up] \table;
    \addplot[blue, opacity=0.5] fill between[of = fluct_p0_dn and fluct_p0_up];
    \addlegendentry{example variation in $p_0$}
    
    \addplot[orange, thin, name path=fluct_p1_dn, forget plot] table [x=cos_theta, y=fluct_p1_dn] \table;
    \addplot[orange, thin, name path=fluct_p1_up, forget plot] table [x=cos_theta, y=fluct_p1_up] \table;
    \addplot[orange, opacity=0.5] fill between[of = fluct_p1_dn and fluct_p1_up];
    \addlegendentry{example variation in $p_1$}
	
\end{axis}
\end{tikzpicture}
    \tikzsetnextfilename{hole_ice_models_scatter}%
\begin{tikzpicture}
\begin{axis}[
        width=0.45\linewidth, height=0.4\linewidth,
		xmajorgrids, ymajorgrids,
		xlabel=$p_0$,
		ylabel=$p_1$,
		legend columns=1,
		legend style={
			/tikz/every even column/.append style={column sep=0.2cm},
			at={(0.95, 0.8)},
			anchor=north east
		}
        % ytick={-0.1, -0.05, 0, 0.05, 0.1},
        % yticklabels={-0.1, -0.05, 0, 0.05, 0.1}
	]
    \addplot[blue, only marks, mark=+, thick] table [x=p0, y=p1] {
       	name	p0	p1
        %nominal	0.567771	0.168269
        h1-100cm	-0.123027	0.131104
        h2-50cm	-0.405128	0.075841
        h3-30cm	-0.595997	0.017468
        dima	0.232258	-0.042754
        dima+	0.265798	-0.070837
        dima-	0.198792	-0.014733
        dragon	0.072961	-0.043175
        greco	0.167150	-0.060809
        %baseline	0.101569	-0.049344
        msu2	0.357705	-0.036428
        martin-0.6-14	-0.028113	-0.035254
        martin-0.8-40	-0.246929	-0.010401
        martin-1.8-125	-0.569511	-0.034839
        all	-0.427318	-0.067581
        tilted	-0.501349	-0.087664
        horizontal	-0.378935	-0.038564
    };
    \addlegendentry{all models}
    \addplot[vermilion, only marks, mark=asterisk, very thick] coordinates {
        (0.567771,	0.168269)
    };
    \addlegendentry{laboratory}
    \addplot[black, only marks, mark=asterisk, very thick] coordinates {
        (0.101569,	-0.049344)
    };
    \addlegendentry{baseline}
    

	
\end{axis}
\end{tikzpicture}
    \caption{Two parameter model used to parametrize the optical efficiency in this analysis (left) and the positions in this two-dimensional space where older hole ice models are located (right). The relative optical efficiency curves are normalized to have the same area, which can lead to acceptance values greater than 1.}
    \label{fig:hole-ice-parametrization}
\end{figure*}

\subsubsection{Depth-dependent ice properties}
\label{sec:depth-dependent-ice-properties}

In the parametrization of the uncertainties of the detector properties described in \refsec{detector-unc}, variations of the scattering and absorption coefficients are only described by global, depth-independent scaling factors.
In principle, the error on the properties of the ice can also change as a function of depth.
Such variations are expected because regions of higher absorption and scattering coefficients will also absorb and scatter the light from the LED flashers that is used to do the calibration.
Higher uncertainties are also expected near the edges of the detector since there are no more calibration light sources outside of the instrumented volume.
Of particular interest for the analysis presented in this work are variations of the ice properties at length scales of the DeepCore fiducial volume located within DeepCore.
Variations at much longer scales would be indistinguishable from uniform variations given the size of the event signatures observed below 100~GeV, while variations at much shorter scales are expected to average out.
To test how significantly such a variation would impact the final level histograms, two MC sets are produced in which the scattering and absorption coefficients vary following a sigmoid function centered in DeepCore with an amplitude of $\pm 2\%$ in opposing directions as shown in \reffig{step-function-ice-model}.
\begin{figure}
    \centering
    \tikzsetnextfilename{ice_step_func_perturbations}%
\begin{tikzpicture}
\begin{axis}[
    height=0.5\linewidth,
    width=0.8\linewidth,
    xmin=1100,xmax=2900,
    xticklabel style={/pgf/number format/.cd,1000 sep={}},
    ymin=0.97, ymax=1.03,
    enlarge y limits=true,
    xlabel=depth (m),
    legend columns=2,
    ylabel=perturbation factor,
    xmajorgrids, ymajorgrids
]
    \addplot[black, thick] table [x=depth, y=perturbation1, col sep=comma] {figures/measurement/systematics/detector/ice_perturbation/ice_perturbations.csv};
    \addlegendentry{perturbation +2\%}
    
    \addplot[orange, thick] table [x=depth, y=perturbation2, col sep=comma] {figures/measurement/systematics/detector/ice_perturbation/ice_perturbations.csv};
    \addlegendentry{perturbation -2\%}
    % dust layer
    \draw [name path=dust layer top, gray, thin] (2000, \pgfkeysvalueof{/pgfplots/ymin}) -- (2000, \pgfkeysvalueof{/pgfplots/ymax}); 
    \draw [name path=dust layer bottom, gray, thin] (2100, \pgfkeysvalueof{/pgfplots/ymin}) -- node[sloped, above, black, font=\footnotesize\sffamily] {dust layer} (2100, \pgfkeysvalueof{/pgfplots/ymax});
    \addplot [gray, opacity=0.4] fill between [of=dust layer top and dust layer bottom];
    
    % IceCube
    \draw [name path=icecube top, gray, thin] (1450, \pgfkeysvalueof{/pgfplots/ymin}) -- (1450, \pgfkeysvalueof{/pgfplots/ymax}); 
    \draw [name path=icecube bottom, gray, thin] (2000, \pgfkeysvalueof{/pgfplots/ymin}) -- (2000, \pgfkeysvalueof{/pgfplots/ymax});
    \node[anchor=south, black, font=\footnotesize\sffamily] at (1750, 0.97) {IceCube\strut};
    \addplot [gray, opacity=0.2] fill between [of=icecube top and icecube bottom];
    
    % DeepCore
    \draw [name path=deepcore top, gray, thin] (2100, \pgfkeysvalueof{/pgfplots/ymin}) -- (2100, \pgfkeysvalueof{/pgfplots/ymax}); 
    \draw [name path=deepcore bottom, gray, thin] (2450, \pgfkeysvalueof{/pgfplots/ymin}) -- (2450, \pgfkeysvalueof{/pgfplots/ymax});
    \node[anchor=south, black, font=\footnotesize\sffamily] at (2270, 0.97) {DeepCore\strut};
    \addplot [gray, opacity=0.1] fill between [of=deepcore top and deepcore bottom];
    
\end{axis}

\end{tikzpicture}

    \caption{Perturbation of the scattering and absorption coefficients with respect to the nominal ice model applied in additional MC sets.}
    \label{fig:step-function-ice-model}
\end{figure}
The size of this variation corresponds approximately a $1\sigma$-allowed variation according to flasher calibration data.
For every bin in the final analysis histogram, a linear regression is fit to the bin counts of the nominal MC set and the two variations.
By comparing the $\chi^2$ test statistic resulting from the regression with the free fit and a regression where the slope is fixed to zero, a p-value can be calculated for every bin, where the null hypothesis is that the step-function variation has no effect.
The p-values for all analysis bins are shown in \reffig{steppiness-pvals} and are consistent with random fluctuations.
Therefore, it was concluded that the effect of a depth-dependent ice model variation is well within the statistical uncertainty of the simulation and need not be included in the measurement.
\begin{figure}
    \centering
    \includegraphics[width=0.9\linewidth]{figures/measurement/systematics/detector/ice_perturbation/steppiness_slope_pvals_verification_sample.pdf}
    \caption{Bin-wise p-value of the fitted slopes as a function of the step-function ice model variation.}
    \label{fig:steppiness-pvals}
\end{figure}

\subsection{Variation of the Atmospheric Neutrino Flux}
\label{sec:flux_systs}

The atmospheric neutrino flux can vary depending on the choice of primary cosmic ray (CR) model, assumed meson yield, hadronic interaction (HI) model and atmospheric density model that are used in the calculation.
The nominal flux, $\Phi_\mathrm{nom}$, is modified to a systematic flux, $\Phi_\mathrm{sys}$, so that

\begin{equation}
    \Phi_{\mathrm{sys}}(E) = \Phi_{\mathrm{nom}} \cdot \left( \frac{E}{E_\mathrm{pivot}}\right)^{\Delta \gamma}
    +
    \sum_{i=1}^{N_\mathrm{Barr}} B_i \cdot \frac{\mathrm{d} \Phi_{\mathrm{nom}}}{\mathrm{d}B_i}\label{eq:flux-variation}
\end{equation}

The $\Delta \gamma$ in \refeq{flux-variation} is due to the CR flux uncertainty and corresponds to shifting the spectral index of the neutrino flux, with a pivot point at $E_\mathrm{pivot}=24\;\mathrm{GeV}$. The second term describes the uncertainty of the Pion and Kaon production yield, where each $B_i$ corresponds to the variation in one \emph{Barr block} (further described below). The gradients with respect to these variations, $\frac{\mathrm{d} \Phi_{\mathrm{nom}}}{\mathrm{d}B_i}$, are calculated using the \textsc{MCEq}\cite{mceq, fedynitch2012influence,fedynitch2015calculation} flux calculator.

\subsubsection{Uncertainty on Meson Production}
\labsec{barr-scheme}
The Barr scheme\sidecite{Barr2006} entails dividing the phase space of incident parent particle $E_\mathrm{i}$ and the outgoing secondary particle $E_\mathrm{s}$ (or, equivalently, $x_{\mathrm{LAB}}=E_\mathrm{s}/E_\mathrm{i}$) into regions that are each denoted by a Barr variable.
There are eight regions/variables that define the uncertainty on $ \pi^+ $ production, and four regions that define the $K^+$ production, as shown in \reffig{barr-blocks}.
For every region, a different relative uncertainty is assigned based on the experimental constraints in that region, as shown in \reffig{barr-blocks-uncertainty}.
For primary particle energies $>500\;\mathrm{GeV}$, an additional energy-dependent term is added to the uncertainty to account for the fact that no accelerator measurements are available at these energies to constrain the meson yield.
As the pion ratio is well-measured, the uncertainty on $ \pi^- $ is defined by the uncertainty on $ \pi^+ $ combined with the uncertainty on the pion ratio.
The uncertainty on $ K^- $ production is parametrized separately from the $K^+$ production.
The only modification to the original Barr scheme used in this analysis is that the low-energy $ \pi^+ $ Barr variables A-F are summarized to a single variable with a relative uncertainty of 63\%, because their impact was found to be highly correlated.
Thus the uncertainty from meson production is described by $N_\mathrm{Barr}=17$ Barr variables that enter \refeq{flux-variation}.

\begin{figure}
    \centering
    \input{figures/measurement/systematics/flux/barr_blocks_annotated.tex}
    %\includegraphics[width=0.8\linewidth]{figures/measurement/systematics/flux/barr_blocks.pdf}
    \caption{Fully correlated regions of uncertainties in the hadronic interaction model. Figure taken from \cite{Barr2006}.}
    \labfig{barr-blocks}
\end{figure}
\begin{figure}
    \centering
    \tikzsetnextfilename{barr_blocks_uncertainty_annotated}%
\begin{tikzpicture}
    \node[above right, inner sep=0] (image) at (0,0) {
        \includegraphics[width=0.8\linewidth]{figures/measurement/systematics/flux/barr_blocks_uncertainty.pdf}
    };
    % Create scope where axes are matching the Pion grid
    \begin{scope}[
        x={($0.42*(image.south east)$)},
        y={($0.67*(image.north west)$)},
        shift={($0.107*(image.south east) + 0.2*(image.north west)$)}
    ]
        % Grid
        %\draw[darkgray,step=0.2] (0,0) grid (1,1);
        \draw[thick, orange, fill=orange, fill opacity=0.8] (0, 0.4) rectangle (1, 1);
        \node[anchor=south, fill=white, draw=black] at (0.5, 0.55) {63\%};
    \end{scope}
\end{tikzpicture}
    %\includegraphics[width=0.8\linewidth]{figures/measurement/systematics/flux/barr_blocks_uncertainty.pdf}
    \caption{Relative uncertainty assigned to each region of hadron phase space in percent. Figure taken from \cite{Barr2006}.}
    \labfig{barr-blocks-uncertainty}
\end{figure}

%Only in the sterile analysis, the prior on the variables with an energy-dependent uncertainty, \texttt{barr\_i\_Pi}, \texttt{barr\_z\_K}, and \texttt{barr\_z\_antiK} by a factor of 5 to 0.61, because it was found that the original priors used in the standard three-flavor analysis greatly under-estimated the impact of these parameters compared to the original Barr 2006 paper.

\subsubsection{Atmospheric density}

The development of particle showers in the atmosphere is governed by competing processes of decay and interactions with the surrounding air.
The density of the atmosphere can therefore influence the rate of neutrino production and could potentially contribute a systematic uncertainty to oscillation measurements.
The size of the effect of atmospheric density uncertainty on the analysis presented in this work is estimated using the same procedure as described in \cite{MEOWS}.
This is done by obtaining a variation of atmospheric density profile by perturbing the Earth’s atmospheric temperature within a prior range given by the NASA Atmospheric InfraRed Sounder (AIRS) satellite~\cite{AIRS} temperature data.
The resulting atmospheric density profile are injected into \textsc{MCEq} to calculate new fluxes.
This is performed for a variety of CR models and hadronic interaction models available in \textsc{MCEq}.
The resulting fluctuations of the neutrino flux observed at the detector were found to be consistently below 1\% for the energy ranges most relevant to DeepCore measurements and is therefore not included as a systematic uncertainty in this work.



% Measurement of neutrino oscillation parameters
% \setchapterstyle{kao}
% \chapter{Measurement of Neutrino Oscillation Parameters}
% \setchapterpreamble[u]{\margintoc}
% \labch{measurement}

% \section{Simulation and Data Sample}

The method by which all of the measurements presented in this thesis are performed is that of \emph{Monte-Carlo (MC) forward folding}. In a nutshell, this method involves producing a large set of simulated signal and background events that are then re-weighted in such a way that their distribution matches that of the observed data events as closely as possible. To give reliable results, an accurate simulation of all particle interactions described in Section~\ref{sec:particle-interactions} as well as the detector electronics described in section~\ref{sec:dom-daq} is required. The simulated and observed events are then passed through the same data processing chain described in section~\ref{sec:data-processing}. The resulting MC simulated dataset and the observed dataset are then histogrammed in the same binning, and the weights of the MC events are adjusted to give the best match between the histograms according to a loss function as defined in section~\ref{sec:test-statistic}.

The simulation chain for neutrinos and atmospheric muons can generally be divided into three steps that are described in this chapter:
\begin{enumerate}
    \item Simulation of particle interactions
    \item Photon propagation in ice
    \item Response of detector DAQ systems
\end{enumerate}
A special case is the simulation of detector noise, for which no particle production or photon propagation is necessary.

\subsection{Particle Interactions}

The first step for the simulation of neutrinos and muons is to sample parameters for the primary particle, and to simulate the secondary charged particles that are produced when it interacts inside the detector. The charged components of the secondary particles are then passed on to the photon propagation step described in section~\ref{sec:photon-propagation}.

\subsubsection{Neutrinos}

Because of the inherently low interaction rate of neutrinos, it would be impractical to simulate a constant flux of neutrinos from any particular direction, the vast majority of which would simply pass through the detector without producing any signal at all. Instead, every simulated neutrino is forced to interact within a given volume, and the event is given a weight corresponding to the inverse of the simulated fluence,
\begin{equation}
    w = \frac{1}{F_{\mathrm{sim}}} \frac{1}{N_{\mathrm{sim}}}\;,
\end{equation}
where $N_{\mathrm{sim}}$ is the number of simulated events and $F_{\mathrm{sim}}$ is the fluence per area, solid angle, energy, and time. This weight, when multiplied with the flux of a given physics model and a live time, gives the expected number of events that this simulated event corresponds to.
%% TODO: I'm honestly not sure if it is necessary to explicitly calculate the start- and endpoints and calculate the length of the path through the cylinder. In my mind, if the position is randomly chosen inside the volume, then the weight should also only depend on the volume. The path taken through the volume should only matter if absorption plays a role
%For this analysis, the simulated interaction volume is a cylinder centered in DeepCore, with a length and radius chosen such that all events that have a chance of producing a signal in DeepCore should be contained in it. The neutrino directions are sampled isotropically in azimuth and zenith, implying that the simulated flux per solid angle is $\phi_\Omega = \frac{1}{4\pi}$. The simulated neutrino flux is a power law with $\phi_e \propto E^{-2}$. After sampling the zenith and azimuth for an event, a random position is sampled  
Under the assumption that neutrino absorption is negligible and that the material consists of isoscalar targets, the simulated fluence is given by the chosen probability density in the direction and energy, $\phi_\Omega \times \phi_E$,  the size of the interaction volume, $V$, the cross-section of the interaction, $\sigma$, and the density of the material, $\rho$, by
\begin{equation}
    F_{\mathrm{sim}}^{-1} = V \times \rho \times N_A \times 1\frac{\mathrm{mol}}{\mathrm{g}} \times \sigma \times \frac{1}{\phi_\Omega} \times \frac{1}{\phi_E}\;,
\end{equation}
where $N_A$ is Avogadro's number. The volume in which neutrino interactions are simulated is a cylinder centered in DeepCore, with a length and radius chosen such that all events that have a chance of producing a signal in DeepCore should be contained in it, depending on the neutrino flavor and energy (see also table~\ref{table:GENIE}). Neutrino directions are isotropically distributed in zenith and azimuth, implying $\phi_\Omega = \frac{1}{4\pi}$. The neutrino energies are sampled from a power law with $\phi_e \propto E^{-2}$. The simulated live time corresponding to a single simulated event is  $T_{\mathrm{sim}} =  F_{\mathrm{sim}} / \Phi$, where $\Phi$ is the expected neutrino flux including neutrino oscillations at global best-fit parameters. The amount of simulation generated for each neutrino flavor is chosen such that the total simulated live time is $\sim 70$~years. Neutrinos and anti-neutrinos are produced in ratios of 70\% and 30\%, respectively. The simulated live time as a function of energy is shown in figure~\ref{fig:sim-livetime}. The livetime for electron neutrinos is increasing with energy because the simulated spectrum is harder than the real spectrum. The livetime for tau neutrinos is much higher than that of other flavors because the contribution of tau neutrinos to the expected neutrino flux is very small.

\begin{table}
\caption{Table of generation volumes used for \textsc{Genie} neutrino simulation. The cylinder is centered in DeepCore in all cases. \label{table:GENIE}}
\begin{center}
\begin{tabular}{ ccccc } 
\textbf{Flavor} & \textbf{Energy (GeV)} & \textbf{Radius (m)} & \textbf{Length (m)}\\
\toprule
\multirow{4}{*}{$\nu_e+\bar{\nu_e}$}  & 1-4 & 250 & 500 \\
 & 4-12 & 250 & 500   \\ 
 & 12-100 & 350 & 600  \\
 & 100-10000 & 550 & 1000  \\
 \midrule
\multirow{4}{*}{$\nu_{\mu}+\bar{\nu_{\mu}}$} & 1-5 & 250 & 500\\
 & 5-80 & 400 & 900\\
 & 80-1000 & 450 & 1500\\
 & 1000-10000 & 550 & 1500\\
 \midrule
\multirow{5}{*}{$\nu_{\tau}+\bar{\nu_{\tau}}$} & 1-4 & 250 & 500\\
 & 4-10 & 250 & 500\\
 & 10-50 & 350 & 600\\
 & 50-1000 & 450 & 800\\
 & 1000-10000 & 550 & 1500\\
 \bottomrule
\end{tabular}
\end{center}
\end{table}

\begin{figure}
    \centering
    
\tikzsetnextfilename{mc_livetime}%
\begin{tikzpicture}

\pgfplotstableread{figures/icecube/selection/livetime/livetime_hists.csv}\table

\begin{loglogaxis}[
    width=0.7\linewidth,
    height=0.5\linewidth,
    tick align=outside,
    tick pos=left,
    xmin=1, xmax=10000,
    xmajorgrids,
    ymajorgrids,
    xlabel=energy (GeV),
    ylabel=total MC livetime (years),
    ymin=20, ymax=80000,
    legend style={
      at={(0.95,0.95)},
      anchor=north east,
    },
]
% livetimes in the table are months per file
% number of files taken from the nominal MC only
\addplot[const plot, black, thick] table[x=energy, y expr=613 * \thisrow{genie_120000} / 12] from \table;
\addlegendentry{\(\nu_e\)}
\addplot[const plot, orange, thick] table[x=energy, y expr=1519 * \thisrow{genie_140000} / 12]  from \table;
\addlegendentry{\(\nu_\mu\)}
\addplot[const plot, skyblue, thick] table[x=energy, y expr=340 * \thisrow{genie_160000} / 12]  from \table;
\addlegendentry{\(\nu_\tau\)}
\end{loglogaxis}

\end{tikzpicture}

    \caption{Simulated livetime for per file, calculated using the HKKM model flux with \textsc{NuFit}~2.2\cite{nufit22} oscillation parameters.}
    \label{fig:sim-livetime}
\end{figure}

After sampling the parameters of the primary neutrino, the \textsc{Genie}~\sidecite{Andreopoulos:2015wxa} software is used to simulate its interaction with the ice and the production of secondary particles and to calculate the cross-section of the interaction. The propagation and Cherenkov light production of any muon that is produced in these interactions is simulated with \textsc{Proposal}\sidecite{proposal}. The light output of secondary electrons, positrons, and gamma rays above 100~MeV, and that of hadronic showers above 30~GeV, is generated using analytic approximations from \cite{RADEL2013102} as described in sections \ref{sec:em-showers} and \ref{sec:had-showers}. At lower energies, the full \textsc{Geant4} simulation of the shower development is run to produce the Cherenkov photon yield.

\subsubsection{Atmospheric muons}
As events propagate through the offline filter steps described in section~\ref{sec:offline-filter}, the rate of atmospheric muons decreases by several orders of magnitude. This makes it challenging to produce a sufficiently large amount of simulated muon events to accurately estimate the expected background at the final level. To overcome this challenge, two separate muon simulation sets are produced, one of which is used to tune the lower level (up to L4) offline filters and the other is used to estimate muon background at levels L5 and above. 

For both sets, atmospheric muons are generated on the surface of a cylinder encompassing the entire IceCube detector with a radius of 800~m and a height of 1600~m. Positions and directions are sampled according to a flux expectation derived from cosmic ray flux model described in \sidecite{Gaisser:2011klf} and the \textsc{SIBYLL 2.1}\sidecite{sibyll} hadronic interaction model. The same flux model is also used to weight the simulated events. For the simulations used to tune the lower selection levels, the muon energy is sampled from a power law with a spectral index of -3 and all events are accepted to cover the entire IceCube array. To produce the simulation that is used starting at the L5 trigger level, muons are only accepted if they intersect an inner cylinder centered in the DeepCore fiducial volume with a radius of 180~m and a height of 400~m. Furthermore, muons are (problematically) rejected based on a KDE estimate of the muon density in energy and zenith angle at the L5 filter level. In this way, the sampling preferably produces such muon events that have a higher chance of passing the offline filtering up to L5, which greatly improves the efficiency of the simulation production.

After the position, direction and energy for a muon has been sampled, its propagation and photon production is simulated using \textsc{PROPOSAL} in just the same way as any muon that is produced in neutrino interaction would be.


\subsection{Photon Propagation}
\label{sec:photon-propagation}

Photons are individually traced through the ice using the GPU-accelerated \textsc{clsim}\cite{clsim} package, which is an \textsc{OpenCL} re-implementation of the Photon-Propagation Code\sidecite{ppc}.  The ice is modeled as 10~m thick layers with individual scattering and absorption coefficients that are shown in Figure~\ref{fig:spice-model}. The ice model used for the simulation in this work also incorporates the fact that the ice layers are slightly tilted with respect to the vertical axis, and that scattering and absorption distances are not uniform as a function of azimuth. For every photon, \textsc{clsim} first samples the absorption length from an exponential distribution where the expectation value is the absorption length of the current layer. It then propagates all photons in parallel steps, where every step corresponds to one scattering event and the step length is sampled from an exponential distribution where the expectation value is the scattering length of the current layer. The scattering angle is then sampled from a mixture of a Henyey-Greenstein distribution and a simplified Mie scattering distribution, where the shape parameters of these distributions have previously been calibrated using the in-situ LED calibration system\sidecite{flasher_calibration}. The algorithm determines after every step if the photon has either reached its total absorption length or if it has intersected a DOM and stops its propagation if that is the case. After all photons have either been absorbed or reached a sensor, the simulations stops and passes the photons that reached a sensor on to the next step simulating the detector response.

\begin{figure*}
    \centering
    \includegraphics[width=0.9\linewidth]{figures/icecube/ice/Spice3.2.1_layered_scatt_abs_withlength_annotated.png}
    \caption{Scattering and absorption coefficients as a function of depth in the South Pole Ice (SPICE) model that is used to produce the simulation for this work.}
    \label{fig:spice-model}
\end{figure*}


\subsection{Simulation of Detector Response}

After the photons have reached the surface of the optical sensors, the simulation determines for each one if it is converted into a Monte-Carlo photo-electron (MCPE). The probability that this occurs depends on the wavelength-dependent sensitivity of the DOM, as well as the angular acceptance. The angular acceptance not only depends on the geometry of the DOM itself, but also incorporates the effect of the re-frozen column of ice surrounding each string. If a photon is accepted and converted into an MCPE, the next step is to simulate how much charge would be measured by the PMT inside the DOM as a response. The charge is drawn from a combination of a normal distribution and two exponential distributions whose parameters have been calibrated \emph{in-situ} to match the observed charge distribution in each individual DOM\sidecite{ic_spe_20}. This distribution, also referred to as the Single Photo-Electron (SPE) template, is shown in Figure~\ref{fig:spe-templates}. The MCPEs with the samples charge are then converted into simulated waveforms for the ATWD and fADC readouts which are then passed into the data processing chain starting from the \emph{wavedeform} algorithm described in Section~\ref{sec:dom-daq}. From there, the simulated events pass through all the same trigger and filter steps that are described in Section~\ref{sec:data-processing}.

\begin{figure}
    \centering
    \includegraphics[width=0.8\linewidth]{figures/icecube/detector_response/SPE_TA003_2.pdf}
    \caption{The green (yellow) regions show the 68\% (90\%) spread in the SPE charge templates for a given charge.  Superimposed are the average SPE charge templates for the variety of hardware configurations shown in the black dotted, dashed, and solid lines. The TA0003 distribution, shown in red, originates from laboratory measurements. Figure taken from \cite{ic_spe_20}.}
    \label{fig:spe-templates}
\end{figure}

\subsubsection{Detector Noise}

\begin{margintable}
\caption{\label{tab:vuvuzela_params} Parameters used in the noise simulation. }
    \begin{tabular}{lc}\toprule
        \textbf{Parameter} & \textbf{Unit} \\ \midrule
        Thermal rate &  $s^{-1}$ \\ 
        Decay rate &  $s^{-1}$ \\
        Decay hits &  hits \\ 
        Decay hits mean &  $\log_{10} (ns) $\\ 
        Decay hits sigma &  $\log_{10} (ns) $ \\ \bottomrule
    \end{tabular}
\end{margintable}
Detector noise in IceCube consists mostly of photons produced in radioactive decay inside the glass housing of the DOMs and the PMTs. However, the simulation does not simulate the propagation of these photons. Instead, noise MCPEs are directly sampled from distributions that take both thermal and non-thermal noise components into account.The thermal component comes from uncorrelated photons and PMT dark noise and is modeled as a Poisson process with a constant rate. The non-thermal component comes from correlated bursts of photons that are produced by radioactive decays. To simulate it, decay times are first drawn from a Poisson process with a constant rate, and the number of photons produced in each event is sampled from a Poisson distribution. The time differences between the non-thermal MCPEs produced by each decay are then sampled from a Log-Gaussian distribution. This simulation method has five free parameters listed in Table~\ref{tab:vuvuzela_params} that are calibrated \emph{in-situ} for every DOM. All thermal and non-thermal MCPEs are injected into each simulated event together with the MCPEs from photons and passed into the rest of the simulation chain.
% \begin{margintable}
% \caption{\label{tab:vuvuzela_params} Parameters used in the noise simulation. }
%     \begin{tabular}{lcc}\toprule
%         \textbf{Parameter} & \textbf{Designation} & \textbf{Unit} \\ \midrule
%         Thermal rate & $\lambda_{Th}$ & $s^{-1}$ \\ 
%         Decay rate & $\lambda_{Decay}$ & $s^{-1}$ \\
%         Scintillation hits & $\eta_{Scint}$ & hits \\ 
%         Scintillation mean & $\mu_{Scint}$ & $\log_{10} (ns) $\\ 
%         Scintillation sigma & $\sigma_{Scint}$ &  $\log_{10} (ns) $ \\ \bottomrule
%     \end{tabular}
% \end{margintable}


\subsection{Final Sample and Binning}
\label{sec:sample-binning}
The starting point for this analysis is a DeepCore data sample consisting of 7.5 years of good live time and 21,914 events that pass through all selection steps described in section~\ref{sec:data-processing}. For every event in the sample, the energy and zenith angle is reconstructed as discussed in section~\ref{sec:event-reconstruction}. Both data and simulation sets are binned in reconstructed energy ($E_{\rm reco}$), cosine of the reconstructed zenith angle ($\cos(\theta_z)$), and PID as follows:

\begin{itemize}
    \item $E_{\rm reco}$: 11 bins spanning the range from 6.31~GeV to 158.49~GeV, the two bins with the highest energy are merged.
    \item $\cos(\theta_z)$: 10 bins spanning the range from -1 to 0.1
    \item PID: One bin between 0.55 and 0.75, and one bin between 0.75 and 1.0.
\end{itemize}

The lower PID bin between 0.55 and 0.75 consists to 69\%  (pre-fit MC estimate) of charged-current $\nu_\mu + \bar{\nu}_\mu$ events and is referred to as the \emph{mixed} channel, while the higher PID channel between 0.75 and 1.0 consists to 94\% of charged-current $\nu_\mu + \bar{\nu}_\mu$ events and is referred to as the \emph{tracks} channel. The histogram of both channels at the null hypothesis (i.e. no sterile signal) is shown in figure~\ref{fig:nominal-hist-null-hypo}.

\begin{figure}
    \centering
    \includegraphics[width=0.95\textwidth]{figures/measurement/simulation_and_data/binning/plot_maps_total.pdf}
    \caption{Expected event counts in 7.5 years of live time assuming no sterile mixing and NuFit~4.0~\cite{nufit40} global best fit parameters at Normal Ordering.}
    \label{fig:nominal-hist-null-hypo}
\end{figure}

\begin{table}[htb]
\centering
\caption{Expected event rate with 8 years livetime broken down in event types and PID bins, calculated at NuFit~4.0 global best fit parameters.}
\label{tab:event-rate}
\begin{tabular}{lccc} \toprule
Type  & PID & Counts [8 years] & Rate [$\mathrm{\mu Hz}$] \\ \midrule
All MC & mixed  &   11428 &   48.3\\
All MC & tracks &   12238 &   51.7\\ \midrule
${\nu_{\rm all}} + {\bar\nu_{\rm all}} \, {\rm NC} $ & mixed  &     943 &    4.0 \\
${\nu_e} + {\bar\nu_e} \, {\rm CC}                 $ & mixed  &    1704 &    7.2 \\
${\nu_\mu} + {\bar\nu_\mu} \, {\rm CC}             $ & mixed  &    7901 &   33.4 \\
${\nu_\tau} + {\bar\nu_\tau} \, {\rm CC}           $ & mixed  &     470 &    2.0 \\
muons                                                & mixed  &     410 &    1.7 \\
\midrule
${\nu_{\rm all}} + {\bar\nu_{\rm all}} \, {\rm NC} $ & tracks &     171 &    0.7 \\
${\nu_e} + {\bar\nu_e} \, {\rm CC}                 $ & tracks &     294 &    1.2 \\
${\nu_\mu} + {\bar\nu_\mu} \, {\rm CC}             $ & tracks &   11517 &   48.7 \\
${\nu_\tau} + {\bar\nu_\tau} \, {\rm CC}           $ & tracks &     162 &    0.7 \\
muons                                                & tracks &      93 &    0.4 \\
\bottomrule
\end{tabular}
\end{table}

% \begin{figure*}
%     \centering
%     \ref{reco_coszen_prefit_legend}\par
%     \begin{tikzpicture}
    \pgfplotstableread{figures/icecube/selection/final_sample_prefit/reco_coszen.csv}\table
    \begin{groupplot}[
        xmin=-1.055,xmax=0.15500000000000003,
        xmode=normal,
        xmajorgrids, ymajorgrids,
        width=0.45\linewidth,
        ylabel style={at={(-0.15,0.5)}},
        group/.cd,
        group size=1 by 2,
        xticklabels at=edge bottom,
        vertical sep=10pt
        ]
    \nextgroupplot[
        height=0.3\linewidth,
        legend cell align={left},
        legend columns=-1,
        legend to name=reco_coszen_prefit_legend,
        ymode=log,
        ymin=2.670138976548827e-09, ymax=3e-5,
        ylabel=rate (Hz),
        ytick distance=1e1,
        % add magic filter to correctly handle empty bins in logarithmic y-axes:
        % If a bin-count is too low or zero, it would cause the line to be
        % interrupted, which creates artefacts and ugliness. Instead, we replace
        % these bin-counts with values that are just below the axis limit.
        % Because of the way pgfplots works, the input is the raw number but the
        % output has to be the log. Weird, I know.
        y filter/.expression={y < 2.670138976548827e-09 ? ln(2.6701389765488273e-10) : ln(y)}
    ]

    \ploterrorband[muon_color]{muon}{1}
    \addlegendentry{atm. muons}

    \ploterrorband[nue_color]{nuenuebar}{1}
    \addlegendentry{$\nu_e + \bar{\nu}_e$}

    \ploterrorband[numu_color]{numunumubar}{1}
    \addlegendentry{$\nu_\mu + \bar{\nu}_\mu$}

    \ploterrorband[nutau_color]{nutaunutaubar}{1}
    \addlegendentry{$\nu_\tau + \bar{\nu}_\tau$}


    % alternative event breakdown by interaction
    % \ploterrorband[nue_color]{nue_ccnuebar_cc}{1}
    % \addlegendentry{$\nu_e + \bar{\nu}_e$, CC}
    %
    % \ploterrorband[numu_color]{numu_ccnumubar_cc}{1}
    % \addlegendentry{$\nu_\mu + \bar{\nu}_\mu$, CC}
    %
    % \ploterrorband[nutau_color]{nutau_ccnutaubar_cc}{1}
    % \addlegendentry{$\nu_\tau + \bar{\nu}_\tau$, CC}
    %
    % \ploterrorband[nc_color]{nuall_ncnuallbar_nc}{1}
    % \addlegendentry{all $\nu$, NC}

    \ploterrorband{total_mc}{1}
    \addlegendentry{total MC}

    \ploterrorbar{data}
    \addlegendentry{data}


    \nextgroupplot[
        height=0.2\linewidth,
        ymin=0.5, ymax=1.5,
        ylabel=data/MC ratio,
        xlabel=reconstructed $\cos(\theta_{\mathrm{zenith}})$
    ]

    \ploterrorband{data_mc_ratio}{1}
    \end{groupplot}
\end{tikzpicture}

%     \tikzsetnextfilename{final_level_postfit_three_flav_reco_energy}%
\begin{tikzpicture}
    \pgfplotstableread{figures/measurement/three_flavor/results/data_mc_post_fit/reco_energy.csv}\table
    \begin{groupplot}[
        xmin=5.362591587787688,xmax=185.61920737481475,
        xmode=log,
        xmajorgrids, ymajorgrids,
        width=0.45\linewidth,
        ylabel style={at={(-0.15,0.5)}},
        group/.cd,
        group size=1 by 2,
        xticklabels at=edge bottom,
        vertical sep=10pt
        ]
    \nextgroupplot[
        height=0.3\linewidth,
        legend cell align={left},
        legend columns=3,
        legend to name=reco_energy_postfit_threeflav_legend,
        ymode=log,
        ymin=3e-8, ymax=3e-5,
        ylabel=rate (Hz),
        % add magic filter to correctly handle empty bins in logarithmic y-axes:
        % If a bin-count is too low or zero, it would cause the line to be
        % interrupted, which creates artefacts and ugliness. Instead, we replace
        % these bin-counts with values that are just below the axis limit.
        % Because of the way pgfplots works, the input is the raw number but the
        % output has to be the log. Weird, I know.
        % y filter/.expression={y < 1.129498430227963e-09 ? ln(1.1294984302279631e-10) : ln(y)}
        y filter/.expression={y < \pgfkeysvalueof{/pgfplots/ymin} ? ln(\pgfkeysvalueof{/pgfplots/ymin}) - 1 : ln(y)}
    ]

    \ploterrorband[muon_color]{muon}{1}
    \addlegendentry{atm. muons}

    \ploterrorband[nue_color]{nuenuebar}{1}
    \addlegendentry{$\nu_e + \bar{\nu}_e$}

    \ploterrorband[numu_color]{numunumubar}{1}
    \addlegendentry{$\nu_\mu + \bar{\nu}_\mu$}

    \ploterrorband[nutau_color]{nutaunutaubar}{1}
    \addlegendentry{$\nu_\tau + \bar{\nu}_\tau$}


    % alternative event breakdown by interaction
    % \ploterrorband[nue_color]{nue_ccnuebar_cc}{1}
    % \addlegendentry{$\nu_e + \bar{\nu}_e$, CC}
    %
    % \ploterrorband[numu_color]{numu_ccnumubar_cc}{1}
    % \addlegendentry{$\nu_\mu + \bar{\nu}_\mu$, CC}
    %
    % \ploterrorband[nutau_color]{nutau_ccnutaubar_cc}{1}
    % \addlegendentry{$\nu_\tau + \bar{\nu}_\tau$, CC}
    %
    % \ploterrorband[nc_color]{nuall_ncnuallbar_nc}{1}
    % \addlegendentry{all $\nu$, NC}

    \ploterrorband{total_mc}{1}
    \addlegendentry{total MC}

    \ploterrorbar{data}
    \addlegendentry{data}


    \nextgroupplot[
        height=0.2\linewidth,
        ymin=0.7, ymax=1.3,
        ylabel=rate/total MC,
        xlabel=reconstructed energy (GeV)
    ]

    \ploterrorbar{data_mc_ratio}
    % \plotratioerrorband[muon_color]{muon}{total_mc}
    % \plotratioerrorband[nue_color]{nuenuebar}{total_mc}
    % \plotratioerrorband[numu_color]{numunumubar}{total_mc}
    % \plotratioerrorband[nutau_color]{nutaunutaubar}{total_mc}

    \node[anchor=south, font=\footnotesize] at (axis description cs:0.5, 0.01) {$\chi^2_{\mathrm{mod}}/\mathrm{dof} = 1.19$};
    \end{groupplot}
\end{tikzpicture}

%     \caption{Distributions of reconstructed energy and zenith angle at the final level of the event selection, calculated assuming \textsc{NuFit}~4.0 global best fit oscillation parameters.}
%     \label{fig:pre-fit-energy-coszen}
% \end{figure*}


%\section{Statistical Analysis}

\subsection{Definition of test statistic}
\label{sec:test-statistic}

To make a measurement, the discrepancy between the histograms from re-weighted MC events and observed data events has to be measured by an appropriate test statistic.
This measurement uses a modified $\chi^2$ test statistic defined as
\begin{equation}
\chi^2_{\mathrm{mod}} = \sum_{i \in \mathrm{bins}}^{}\frac{(N^{\mathrm{exp}}_i - N^{\mathrm{obs}}_i)^2}{N^{\mathrm{exp}}_i + (\sigma^{\mathrm{exp}}_i)^2} + \sum_{j \in \mathrm{syst}}^{}\frac{(s_j - \hat{s_j})^2}{\sigma^2_{s_j}},
\label{eq:mod-chi2}
\end{equation}

\noindent where the expectation within a bin is calculated as the sum of the MC event weights $N^{\mathrm{exp}}_i = \sum_{i}^{\mathrm{evts}} w_i$. The error term due to Poisson fluctuations of the data is calculated with the MC expectation, $N^{\mathrm{exp}}_i$. The statistical uncertainty due to finite simulation statistics is also included as $(\sigma^{\mathrm{exp}}_i)^2 = \sum_{i}^{\mathrm{evts}} w_i^2$.
The second term in equation~\ref{eq:mod-chi2} is included as a penalty term to account for prior knowledge of some systematic parameters.


\subsection{Muon KDEs}
\label{section:muon_kde}

After all the filtering steps described in section~\ref{sec:data-processing}, the muon contamination of the data sample is reduced to $\sim 2\%$ of the sample.
This reduces the statistics of muon simulation so much, that the resulting histograms become very sparse as shown in figure~\ref{fig:muon-template-no-kde}.
Such sparse histograms, in which single MC events have to serve as a stand-in for several real data events, are a poor template for what can be expected in data.
To produce a more realistic expectation of the bin counts, the muon histograms are smeared using KDEs as shown in figure \ref{fig:muon-template-with-kde}.
Since the KDE operates on events on the entire zenith and energy range, including events that fall outside the analysis binning, some events bleed into the highest $\cos(\theta_z)$ bin from further above the horizon.
The KDE kernel is mirrored at $\cos(\theta_z) = -1$ to avoid spurious disappearance of events at the edge.

\begin{figure}[H] 
    \centering
    \begin{subfigure}{0.8\textwidth}
        \centering
        \includegraphics[width=\textwidth,trim={0 0 0 0.6cm},clip]{figures/measurement/systematics/muons/muon_hist_no_kde.pdf}
        \caption{Without KDE smoothing}
        \label{fig:muon-template-no-kde}
    \end{subfigure}
    \begin{subfigure}{0.8\textwidth}
        \centering
        \includegraphics[width=\textwidth,trim={0 0 0 0.6cm},clip]{figures/measurement/systematics/muons/plot_maps_muon.pdf}
        \caption{With KDE smoothing}
        \label{fig:muon-template-with-kde}
    \end{subfigure}
    
    \caption{Muon template before (top) and after (bottom) the application of KDE smoothing. The shown values are the average of 20 KDE evaluations on different bootstrap samples.}
    \label{fig:muon-kde-smoothing}
\end{figure}

\subsubsection{KDE error estimates}
To estimate the error on the KDE output, 20 bootstrap samples are drawn separately in each PID channel and the KDE is re-evaluated for each trial.
The expectation value and standard deviation of the expectation in each bin of the histogram is the mean and standard deviation of these samples, respectively.
The samples are always produced with the same initial random seed to ensure that the expectations and errors are reproducible.

% \section{Systematic Uncertainties}

\subsection{Detector Properties}
\label{sec:detector-unc}
Systematic uncertainties of the detector properties are parametrized with the same parameters that have been used in the standard three-flavor analysis. These parameters and their priors are informed by detector calibration studies.
\begin{itemize}
    \item DOM efficiency: A factor that scales the probability that a photon hitting the PMT of a DOM will produce a photo-electron that is measured by the electronics. Nominal value is 1, prior standard deviation is 10\%.
    \item Hole ice: The ice within the column where the holes for the strings have been drilled has different optical properties from the surrounding ice. The effect of this difference is modelled as a variation of the angular acceptance of the DOMs. A Principal Component Analysis was run on several pre-existing models of the angular acceptance curves to find a generic parametrization of their differences. In this analysis, we scale the two most important Principal Components to model the effect of hole ice on the histogram. There are no priors on those components and they are allowed to vary \emph{within the range covered by the produced systematic sets plus a small amount of extrapolation}\footnote{This is different from the verification sample, where parameters were allowed to vary in the entire physically plausible range.}.
    \item Bulk ice: The over-all absorption and scattering coefficients of all ice layers are multiplied by a scaling factor. The nominal value for ice absorption is 1.0 with a prior standard deviation of 5\%. The nominal value for ice scattering is 1.05 with a prior standard deviation of 10\%.
\end{itemize}

To model the effect of the detector uncertainties, several MC sets at different variations of DOM efficiency, hole ice, and bulk ice parameters were produced. These MC sets are used to find a parametrization that will model how the distribution of events in energy, zenith and PID will change as a function of these parameters as described in the following section.

\subsubsection{Method of bin-wise linear fits}
\label{sec:hypersurfaces}

\subsubsection{Method of likelihood-free inference}

A fundamental weakness of linear fit treatment described in section~\ref{sec:hypersurfaces} is that the fitted parameters are only valid for the particular choice of oscillation and flux parameters that they have been fit with. This can be mitigated to a degree by running the fits over a grid in $\Delta m^2_{31}$ and interpolating all fit parameters with a piece-wise linear function between those points, which is sufficient for the three-flavor standard oscillation analysis. However, for the sterile analysis, the required grid would have to have at least five dimensions and the RAM requirement to hold all those parameters would have been prohibitive. This section describes an alternative treatment that entirely decouples the detector response from flux and oscillation weights.

The goal of this treatment of detector systematic effects is to find re-weighting factors for every event in the nominal MC set that correspond to how much more or less likely that particular event would be if the detector properties were different. These re-weighting factors should be independent from any other event weights, especially flux and oscillation weights. To get an intuition, one may look at an event in which the reconstructed energy is larger than the true energy. Such an event is more likely to occur when the DOM efficiency is higher than nominal, and less likely when the DOM efficiency is lower than nominal. This relationship will not change depending on the initial flux of the primary neutrino or oscillation effects, because the detector only reacts to the final state.

Using the discrete MC sets with different variations of detector parameters, one needs to find the re-weighting factors, $R_{ik}$, for each event in the nominal MC set, $i$, such that re-weighting every event by its weighting factor produces the same distribution in energy, zenith angle and PID as the off-nominal MC set, $k$. If the true distributions of event parameters in each MC set was known, this weighting factor could be calculated as the ratio
\begin{equation}
    R_{ik} = \frac{P(X_i|\theta_k)}{P(X_i|\theta_{\rm nominal})} \frac{P(\theta_k)}{P(\theta_{\rm nominal})}\;, \label{eq:ratio-likelihood}
\end{equation}
where $X_i$ are the parameters of the event, and $\theta_k$ are the detector parameters of the off-nominal MC set, $k$. The probability distribution $P(X_i|\theta_k)$ is the distribution of the event parameters in MC set $k$, and $\theta_{\rm nominal}$ contains the values of the detector parameters of the nominal MC set. The second fraction is the ratio of the total normalization of events under different detector parameter values. For instance, if an increase in DOM efficiency increases the total number of events by 10\%, then $P(\theta) / P(\theta_{\rm nominal}) = 1.1$.

In practice, of course, the probability distributions $P(X_i|\theta_k)$ is unknown, but the factors $R_{ik}$  can still be extracted from the available MC sets.
The first step is to apply Bayes' theorem to express $R_i$ as the ratio of the posterior probability distribution of the detector parameters given the event parameters,
\begin{equation}
    R_{ik} = \frac{P(\theta_k|X_i)}{P(\theta_{\rm nominal}|X_i)}\;. \label{eq:ratio-posterior}
\end{equation}
The posterior distributions, $P(\theta_k|X_i)$, can be acquired from a classifier trained to give the posterior probability for an event with parameters $X_i$ to belong to MC set $k$. This means that the task of finding the re-weighting factors can be translated into a \emph{classification task}. Such an inference method, where probability distributions are learned as a ratio of posteriors from a classifier, is also known as a \emph{likelihood-free inference} method.

\subsubsection{K-Neighbors method to calculate posteriors}
In principle, any classifier that provides well-calibrated posterior outputs can be plugged into eq.~\ref{eq:ratio-posterior}. For this analysis, the simple and robust \emph{k-neighbors} method is used.
The K-Neighbors classifier calculates posterior probabilities by finding the set $\mathcal{N}$ of the $N$ nearest neighbors for every event, $i$. This set is defined as the set of $N$ events with the smallest Euclidean distance in the event parameters $X$.  Then, the estimate for the posterior for set $k$ is the fraction of the total weight of the neighbors belonging to set $k$,
\begin{equation}
    P(\theta_k|X_i) = \frac{\sum_{j\in{\mathcal{N}\cup k}} w_j }{\sum_{j\in{\mathcal{N}}} w_j}\;. \label{eq:posterior-knn}
\end{equation}
The weights $w_j$ are the weighted effective area that every simulated MC event corresponds to and correct for the different amount of MC that was produced for every systematic MC set. They do not include neutrino flux or oscillation effects.

While this method is very robust, it is also prone to over-fitting if the number of neighbors is chosen to be too small. On the other hand, if the number of neighbors is too large, it might blur out important features and under-fit. An increased number of neighbors also causes a systematic bias in the probability estimate due to the fact that the distribution across the selected neighbors is not perfectly uniform. This bias is corrected to linear order by re-weighting the events in each neighborhood as described in appendix \ref{apx:knn-correction}. With this bias correction applied, a neighborhood size of 200 per included MC set was found to be a good compromise between bias and overfitting.

\subsubsection{Classification variables}
The input variables passed into the classifier for each event, $X_i$, need to cover all variables that are used in the binning ($E_{\rm reco},\cos(\theta_{\rm reco}),{\rm PID}$) as well as all variables that are used when re-weighting events by flux and oscillation probabilities ($E_{\rm true},\cos(\theta_{\rm true})$)\footnote{Even if other variables can in principle influence the detector response, they do not have to be included as long as their distribution does not change. See also appendix \ref{apx:implicit-marginalization}.}. This gives a total of five input variables that are used for classification. Because the K-neighbors classifier calculates euclidean distances between events in these five dimensions to determine which events are neighbors, all dimensions are transformed to be approximately normally distributed as follows and then scaled to have a unit variance:
\begin{itemize}
    \item Energies are replaced by their logarithm
    \item The zenith angle is used directly, rather than its cosine
    \item The PID, which is the probability output of a BDT, is transformed into the log-odds ratio, $\mathrm{LOR}=\log({\rm PID}) - \log(1-{\rm PID})$. This transformed variable turns the pileup of events near a PID value of one into a long tail.
\end{itemize}
The classifier is fit on the transformed variables separately for each flavor for CC interactions, and the combined set of all NC interactions.

\subsubsection{Calculating event-wise gradients}

The K-Neightbors calculation produces event weights that can re-weight the events of the nominal MC set to imitate the distribution of any other MC set. To be useful in an analysis, however, it is a requirement that this re-weighting can be interpolated to any value of detector parameters between the discrete MC sets. This is accomplished by fitting a vector of gradients, $g_i$, for every event, by minimizing the negative log-likelihood
\begin{equation}
    -\log\mathcal{L} = -\sum_k P_{\rm obs}(\theta_k | X_i) \log P_{k, \rm pred}(g_i, X_i)\;, \label{eq:grad-nllh}
\end{equation}
where the observed probability is $P_{\rm obs}(\theta_k | X_i)$ from the K-neighbors calculation and the  predicted probability $P_{k, \rm pred}(g_i, X_i)$ is
\begin{equation}
    P_{k, \rm pred}(g_i, X_i) = \frac{\exp(\sum_j \theta_{k,j} g_{i,j})}{\sum_l\exp(\sum_j \theta_{l,j} g_j)}\;.
\end{equation}
The motivation for the negative log-likelihood loss in eq.~\ref{eq:grad-nllh} is that minimizing this quantity is equivalent to minimizing the cross-entropy between labels $P_{\rm obs}$ and class predictions  $P_{k, \rm pred}$, and it has been shown that classifiers that minimize the cross-entropy end up learning posterior distributions\cite{NNPosteriors}.

To model non-linear effects, gradients are fit not only to the five detector parameters, but also to their squared values, for a total of ten gradients per event.

\subsubsection{Evaluation}
Once the event-wise gradients for all detector uncertainty parameters have been obtained, all events can be easily re-weighted during a fit for any given set of detector parameters, $\theta$, by multiplying the weight for each event by the ratio
\begin{equation}
    R_i(\theta)=\frac{P(\theta|X_i)}{P(\theta_{\rm nominal}|X_i)}=\exp\left(\sum_j (\theta_j - \theta_{j, \rm nominal}) g_{ij}\right)\;.\label{eq:ultrasurf-eval}
\end{equation}

\subsubsection{Performance}
To verify that the re-weighting according to eq.~\ref{eq:ultrasurf-eval} gives the expected result, they are used to reproduce each systematic MC set and calculate the bin-wise pulls between the reproduction and the systematic set. When the gradients are correct, the pull between the set and its reproduction,
$$p_n = \frac{N_{\mathrm{reprod}, i} - N_{\mathrm{syst}, i}}{\sqrt{\sigma^2_{\rm nominal} + \sigma^2_{\rm syst}}}\;,$$
should follow a standard-normal distribution. Fig.~\ref{fig:ultrasurf-binwise-pulls} shows the result of this test for the four MC sets in which only the DOM efficiency is varied between 90\% and 110\% percent. The spread of the bin-wise pulls closely follows a normal distribution standard deviation of one, as expected, but the total normalization is slightly under-estimated for the MC set 0004 with DOM efficency of 110\%. This is not too concerning for this analysis, since the total normalization is a free parameter without a prior. A similar performance is found for all included MC sets. 

Fig.~\ref{fig:dom-eff-prediction} shows the prediction of the bin count as a function of the DOM efficiency scale, $\epsilon_{\rm DOM}$ at the nominal point (left panel) and for different injected values of the mass splitting $\Delta m^2_{31}$ (right panel) for one arbitrarily chosen bin of the analysis. The prediction matches the shape of the bin count change very well, although it is offset slightly towards lower bin counts. This is expected, since the prediction is based on re-weighting the nominal MC set events without any corrections on the bin count at the nominal point. The error band shown in the figure corresponds to the uncertainty of the nominal set. The right panel of Fig.~\ref{fig:dom-eff-prediction}, demonstrates that the prediction automatically adjusts itself to any injected value of oscillation parameters. This happens despite the fact that the flux and oscillation weights have not been used at all when fitting the event-wise gradients, which demonstrates that the detector response has truly been decoupled from flux and oscillation effects. 

\begin{figure}
    \centering
    \includegraphics[width=0.9\textwidth]{figures/measurement/systematics/detector/ultrasurface_performance_vs_blind_fits_domeff_sets.pdf}
    \caption{Binwise pulls between the nominal set after re-weighting according to eq.~\ref{eq:ultrasurf-eval} and the systematic MC sets 0001, 0002, 0003, and 0004 representing DOM efficiency values of 90\%, 95\%, 105\%, and 110\%, resepctively. The 1D histogram in each row shows the distribution of the pulls over all bins.}
    \label{fig:ultrasurf-binwise-pulls}
\end{figure}

\begin{figure}
    \centering
    \begin{subfigure}{0.45\textwidth}
        \includegraphics[width=\textwidth]{figures/measurement/systematics/detector/dom_eff_prediction.pdf}
        \caption{Prediction at best fit point of three-flavor analysis}
    \end{subfigure}
    \hfill
    \begin{subfigure}{0.45\textwidth}
        \includegraphics[width=\textwidth]{figures/measurement/systematics/detector/dom_eff_mass_splitting_scan.pdf}
        \caption{Prediction at different mass splitting values.}
    \end{subfigure}
    \caption{Prediction of bin counts in one bin of the analysis as a function of the DOM efficiency scale, $\epsilon_{\rm DOM}$. The error band in the left panel corresponds to the error on the nominal MC prediction without errors on the event-wise gradients.}
    \label{fig:dom-eff-prediction}
\end{figure}

\subsection{Atmospheric Neutrino Flux}
\label{section:flux_systs}

The atmospheric neutrino flux can vary depending on the choice of primary cosmic ray (CR) model, assumed meson yield, hadronic interaction (HI) model and atmospheric density model that are used in the calculation. This analysis uses the HKKM model as the baseline flux model and calculate the variation to the overall flux using \href{https://github.com/afedynitch/MCEq}{MCEq version 1.1.3}. The nominal flux, $\Phi_{\rm nom}$, is modified to a systematic flux, $\Phi_{\rm sys}$, so that

$$\Phi_{\mathrm{sys}} = (\Phi_{\mathrm{nom}} \cdot \Delta \Phi_{\mathrm{nom}}) + \bigg( b \cdot \frac{\mathrm{d} \Phi_{\mathrm{nom}}}{\mathrm{d}B} \bigg)$$

The first term is due to the CR flux uncertainty and corresponds to shifting the spectral index of the neutrino flux, with a pivot point at 24 GeV. The uncertainty on meson production is included in the last term, where $b$ is the magnitude of the uncertainty and $ {\rm d} \Phi_{\rm nom} / {\rm d}B $ is the derivative of the Barr modification  (further described below). 

\subsubsection{Uncertainty on Meson Production}

The Barr scheme\sidecite{Barr2006} entails dividing the phase space into a given number of regions, each denoted by a Barr variable. There are eight regions/variables that define the uncertainty on $ \pi^+ $ production, and four regions that define the $K^+$ production. 

As the pion ratio is well-measured, the uncertainty on $ \pi^- $ is defined by the uncertainty on $ \pi^+ $ combined with the uncertainty on the pion ratio. The uncertainty on $ K^- $ production is parametrized separately from the $K^+$ production. Thus the uncertainty from meson production is described by 17 Barr variables (see table \ref{table:mceq_cfg_params}). 
The only modification to the original Barr scheme used in this analysis is that the low-energy $ \pi^+ $ Barr variables A-F are summarized to a single variable, because their impact was found to be highly correlated. Only in the sterile analysis, the prior on the variables with an energy-dependent uncertainty, \texttt{barr\_i\_Pi}, \texttt{barr\_z\_K}, and \texttt{barr\_z\_antiK} by a factor of 5 to 0.61, because it was found that the original priors used in the standard three-flavor analysis greatly under-estimated the impact of these parameters compared to the original Barr 2006 paper.

\subsubsection{Uncertainty on the Cosmic Ray Flux}
The uncertainty on the cosmic ray (CR) flux is implemented as a shift in the spectral index of the neutrino flux
\begin{equation}
    \Delta \Phi = \left( \frac{E}{E_{\rm pivot}}\right)^{\Delta \gamma}\;.
\end{equation}
Other variations to the cosmic ray models were assessed and found to be negligible in their impact.

\begin{table}
\caption{MCEq flux model parameters. Bold numbers indicate that these priors have been inflated to account for high-energy flux uncertainties.}
\label{table:mceq_cfg_params}
\begin{center}
\begin{tabular}{ |l|l| } 
\hline

\textbf{Parameter} & \textbf{Value} \\ \hline

\texttt{table\_file} & Path to pre-computed MCEq \href{https://github.com/IceCubeOpenSource/fridge/tree/master/analysis/common/data/flux}{splines} \\ \hline
\texttt{delta\_index} & $0.0 \pm 0.1$ \\ \hline
\texttt{energy\_pivot} & 24 GeV \\ \hline

\texttt{pion\_ratio} & $0.0 \pm 0.05$ \\ \hline
%\texttt{barr\_a\_Pi} & $0.0 \pm 0.1$ \\ \hline
\texttt{barr\_af\_Pi} & $0.0 \pm 0.63$ \\ \hline
\texttt{barr\_b\_Pi} & $0.0 \pm 0.3$ \\ \hline
\texttt{barr\_c\_Pi} & $0.0 \pm 0.1$ \\ \hline
\texttt{barr\_d\_Pi} & $0.0 \pm 0.3$ \\ \hline
\texttt{barr\_e\_Pi} & $0.0 \pm 0.05$ \\ \hline
\texttt{barr\_f\_Pi} & $0.0 \pm 0.1$ \\ \hline
\texttt{barr\_g\_Pi} & $0.0 \pm 0.3$ \\ \hline
\texttt{barr\_h\_Pi} & $0.0 \pm 0.15$ \\ \hline
\texttt{barr\_i\_Pi} & $0.0 \pm \textbf{0.61}$ \\ \hline

\texttt{barr\_w\_K} &  $0.0 \pm 0.4$ \\ \hline
\texttt{barr\_w\_antiK} &  $0.0 \pm 0.4$ \\ \hline
\texttt{barr\_x\_K} &  $0.0 \pm 0.1$ \\ \hline
\texttt{barr\_x\_antiK} &  $0.0 \pm 0.1$ \\ \hline
\texttt{barr\_y\_K} &  $0.0 \pm 0.3$ \\ \hline
\texttt{barr\_y\_antiK} &  $0.0 \pm 0.3$ \\ \hline
\texttt{barr\_z\_K} &  $0.0 \pm \textbf{0.61}$ \\ \hline
\texttt{barr\_z\_antiK} &  $0.0 \pm \textbf{0.61}$ \\ \hline

\end{tabular}
\end{center}
% \label{table:flux_syst}
\end{table}


\subsection{Neutrino Cross-Sections}
\label{section:xsec_systs}
Two systematic parameters are included to account for uncertainties in the form factors of charged-current quasi-elastic ($M_{A}^{CCQE}$) events and charged-current resonant ($M_{A}^{CCRES}$) events. Both these form factors have a dependency on $Q^2$ of the form:\\

\begin{equation}
    F(Q^{2}) \propto \frac{1}{(1-(Q^{2}/M_{A}^{2})^{2}}
\end{equation}

Where $M_{A}$ is called the \textit{axial mass}, and can be measured experimentally.  The differential cross-section of each event is computed with GENIE at four discrete points around the nominal axial mass value (that is,-2$\sigma$,-1$\sigma$,1$\sigma$ and 2$\sigma$ away from the nominal mass, where $\sigma$ is a fractional uncertainty of 20\%).

In order to apply a continuous variation of that systematic parameter over the course of a minimization, each event is re-weighted by performing a quadratic interpolation between the five discrete values available (ie nominal weight + the 4 re-weighted weights). Figure~\ref{xsec:resonant_mass} show the weights of a handful of $\nu_{e}$ CC events form resonance production, across the allowed range of axial masses, along with their fitted quadratic dependence.The upper panel of figure~\ref{fig:template_xsecsyst} illustrates an example of the varying $M_{A}^{RES}$ on the final level sample.

\begin{figure}
    \centering
    \includegraphics[width=0.5\textwidth]{figures/measurement/systematics/xsec/nue_cc_res_xsec_Ma_systematic.png}
    \caption{Genie interaction weights as a function of the axial mass term systematic $M_{A}$, for 10 $\nu_{e}$ CC events produced via resonance interactions. Each dot represents a discrete point for which the event's cross section is computed in genie. The dashed line represents the quadratic fit made used to interpolate the weight value over the continuous range allowed for the systematic parameter.}
    \label{xsec:resonant_mass}
\end{figure}

\begin{figure}[!t] 
    \centering
    \begin{subfigure}[t]{0.7\textwidth}
        \centering
        \includegraphics[width=0.99\textwidth,trim={0 0 0 0.6cm},clip]{figures/measurement/systematics/xsec/Genie_Ma_RES.pdf}
        \caption{GENIE $M_{A}^\mathrm{RES}$}
    \end{subfigure}
    \begin{subfigure}[t]{0.7\textwidth}
        \centering
        \includegraphics[width=0.99\textwidth,trim={0 0 0 0.6cm},clip]{figures/measurement/systematics/xsec/dis_csms.pdf}
        \caption{DIS CSMS}
    \end{subfigure}  
  \caption{Fractional difference in event rates between (top )$M_{A}^\mathrm{RES}$ (bottom) dis$\_$csms at 1$\sigma$ and at nominal value for both PID bins.
  \label{fig:template_xsecsyst}}
\end{figure}

The uncertainty on the DIS cross-section is primarily given by the disagreement in DIS calculation between CSMS and GENIE cross-sections at energies above 100~GeV. This analysis includes a parameter that interpolates between these two calculations with a linear extrapolation to energies below 100~GeV.
The bottom panel of figure~\ref{fig:template_xsecsyst} illustrates an example of the varying this parameter, DIS, on the final level sample. As expected, the impact of the parameter is largest in the highest energy bins.

There is an additional uncertainty of 20\% on the normalization of NC events to account for uncertainties of the hadronization process and the Weinberg angle.

% Many cross section systematic parameters were tested to see the effect of those on the final verification sample \textcolor{blue}{ and were found to have negligible effect to our analysis so these are not included in this analysis}. A study was performed on the parameters used in the Bodek-Yang model to correct for the parton distribution functions (PDFs) used in the calculation of cross sections in low $Q^2$ region. These DIS events were re-weighted on an event-by-event basis in response to changes in the higher-twist parameters and u valence quark corrections to the GRV98 PDF used in GENIE. Further study was performed on the impact of high-W averaged charged hadronization multiplicity for DIS interactions. It was done by comparing GENIE predictions and bubble chamber data for the averaged charged hadron multiplicity as a function of hadronic mass squared ($W^2$). It was found that modifying PYTHIA6 parameters can achieved better data-MC agreement in the high $W^2$ region.% \cite{Syst_int:PYTHIA6}. 
% Another DIS-related uncertainty studied was its differential cross section. The approach here was to modify the structure function as a function of the Bjorken-x within the uncertainties measured by NuTeV. %\cite{Syst_int:NuTeV}.
% Details of these studies can be found in \href{https://drive.google.com/file/d/1voZ56RCKjDZzPH5qAkvtY3Wkp5EwVi1F/view}{this presentation}.
\subsection{Atmospheric Muons}

Because the muon background contamination is cut to only $\sim$2\%, the impact of muon systematics is generally small. Only the over-all scale is left as a free parameter in the analysis, its impact is shown in figure \ref{fig:weight-scale-syst}. This scale also largely absorbs the effects of DOM efficiency uncertainties, since, to first order, an increase in DOM efficiency leads to a better muon rejection. The spectral index of the muon flux has a very small effect far below the percent-level as shown in figure \ref{fig:delta-gamma-mu-syst} and is therefore kept fixed in the fit.

\begin{figure}[H]
    \centering
    \includegraphics[width=0.7\textwidth,trim={0 0 0 0.6cm},clip]{figures/measurement/systematics/muons/weight_scale.pdf}
    \caption{Impact on the final histograms when the muon normalization is increased by 50\%. The largest impact is seen above the horizon in the mixed PID channel with a change in bin count of 5\%.}
    \label{fig:weight-scale-syst}
\end{figure}

\begin{figure}[H]
    \centering
    \includegraphics[width=0.7\textwidth,trim={0 0 0 0.6cm},clip]{figures/measurement/systematics/muons/delta_gamma_mu.pdf}
    \caption{Impact on the final histograms when the muon spectral index is increased by $1\sigma$.}
    \label{fig:delta-gamma-mu-syst}
\end{figure}

\section{Three-flavor Oscillation Measurement}
\label{sec:three-flavor-measurement}
The first measurement that is run using the data sample described in this work is the measurement of the atmospheric mixing angle $\theta_{23}$ and the mass splitting $\Delta m^2_{32}$. The experimental setup of DeepCore is ideally suited for this measurement, because the first valley of maximum disappearance for muon neutrinos passing through the entire diameter of the Earth is expected to lie between 20~GeV and 30~GeV as shown in \reffig{three-flavor-oscprob}. The parameter $\Delta m^2_{32}$ changes the position of the oscillation valley, while $\theta_{23}$ changes its depth. In the analysis histogram, this disappearance effect is apparent even by eye alone in the PID channel for highly track-like events as shown in \reffig{nominal-hist-null-hypo}. For this measurement, oscillation probabilities are calculated in the three-flavor oscillation scheme including matter effects. The matter profile of Earth is modeled as a shells of constant density following the Preliminary Reference Earth Model (PREM)\sidecite{PREM}. The Monte-Carlo simulated events are weighted in a staged procedure where each stage updates the event weights according to flux, cross-sections and oscillation probabilities\sidecite{PISA}. The oscillation probabilities are calculated using a \textsc{Python} implementation of the Barger~et~al.\sidecite{barger-oscillations} calculation.

\begin{figure}
    \centering
    \includegraphics[width=0.9\linewidth]{figures/measurement/three_flavor/numu_surv_prob_no_sterile_no_text.png}
    \caption{Muon-neutrino survival \todo[inline]{find parameters} probability calculated using}
    \label{fig:three-flavor-oscprob}
\end{figure}

\subsection{Selection of Free Parameters}
\label{sec:std-osc-free-parameters}

When including all plausible sources of systematic uncertainties that are described in \refsec{systematic-uncertainties}, the test statistic from \refeq{mod-chi2} would have to be optimized with respect to 28 nuisance parameters: Five describing the uncertainty of the detector (\refsec{detector-unc}), 18 parameters describing the uncertainties on the hadronic interaction model (\refsec{barr-scheme}) and cosmic ray spectrum, four parameters for uncertainties on the neutrino cross-sections (\refsec{xsec_systs}), and the scale of the muon background (\refsec{atm-muons-systematic}). Together with the two physics parameters, this would require an optimization in 30 dimensions to run the analysis. To reduce this computational burden, the potential bias and its significance that could plausibly be produced by each parameter is assessed, and the value of parameters that are found to have a negligible impact is fixed to its global best-fit value. The impact of each parameter is tested as follows: First, pseudo-data \emph{without} statistical fluctuations is produced from simulation where the value of the parameter to be tested is increased by $1\sigma$ if it has a Gaussian prior, or half-way to its upper boundary if it does not have a prior. The histograms are then fit back while keeping the parameter to be tested fixed at its nominal value. This fit is done once with the physics parameters ($\theta_{23}$ and $\Delta m^2_{31}$) fixed at the value that was used to create the pseudo-data, and once with the physics parameters left free. The difference in the test statistic $\chi^2_{\mathrm{mod}}$ between the free fit and the fit with physics parameters fixed to the truth, $\Delta \chi^2_{\mathrm{mod}}$, is referred to as \emph{mis-modeling}. The p-value of the mis-modeling, calculated under the assumption that it should follow a $\chi^2$-distribution with two degrees of freedom, can be interpreted as the significance with which the analysis would have rejected the true physics value \emph{solely} due to the exclusion of the parameter in question. This test neglects any global offset to the test statistic, since it would not affect the estimate of the confidence limits for the physics parameters.
\begin{figure}
    \centering
    \missingfigure[figwidth=0.8\linewidth,figheight=0.7\linewidth]{Mis-modeling grid produced for one parameter in the mis-modeling ranking test.}
    \caption{Grid of $\Delta \chi^2_{\mathrm{mod}}$ values showing the impact of one parameter being pulled by $1\sigma$.}
    \label{fig:systematic-impact-mismod-example}
\end{figure}
The test described above is repeated for a grid of true values of $\theta_{23}$ and $\Delta m^2_{31}$ spanning the entire range of values that is not strongly excluded by other measurements, producing one value for $\Delta \chi^2_{\mathrm{mod}}$ at each grid point as shown in \reffig{systematic-impact-mismod-example}. The largest value of $\Delta \chi^2_{\mathrm{mod}}$ from the entire grid produced for one parameter represents the \emph{maximum mis-modeling} for that parameter. Taking the maximum mis-modeling for all parameters, one can produce a ranking of the impacts of all parameters of the analysis as shown in \reffig{systematic-impact-mismod-ranking}. Parameters for which the maximum mis-modeling lies below a conservatively chosen value of $\Delta \chi^2_{\mathrm{mod}} < 0.01$ are fixed to their global best-fit value in the analysis, reducing the total number of free parameters to XX\todo{get the exact number of parameters}. The complete list of all parameters of the analysis, their priors and their allowed ranges can be found in \reftab{sys-params-three-flavor} in the appendix.
\begin{figure}
    \centering
    \missingfigure[figwidth=0.6\linewidth, figheight=0.8\linewidth]{Mis-modeling ranking for the three-flavor analysis.}
    \caption{Ranking of $\Delta \chi^2_{\mathrm{mod}}$ values for all nuisance parameters considered for the three-flavor oscillation analysis.}
    \label{fig:systematic-impact-mismod-ranking}
\end{figure}

\subsection{Analysis Checks}
Before running the analysis on real data, its robustness is assessed on pseudo-data produced with MC simulated data sets. Once the robustness on pseudo-data has been established, the analysis is first run \emph{blindly}, that is, without showing the analyzer the results of the physics parameters and only revealing a set of goodness-of-fit variables that has been chosen in advance. Only when the values of these variables lie within the plausible range that can be expected from purely statistical fluctuations are the actual fit values of the measured parameters revealed. 

\subsubsection{Robustness of the minimization}
The free fit of the physics parameters $\theta_{23}$ and $\Delta m^2_{31}$ is run separately once for the lower octant ($\theta_{23} < 45^\circ$) and once for the upper octant ($\theta_{23} > 45^\circ$) to break the degeneracy between the octants. Each fit uses the \textsc{scipy}\cite{2020SciPy-NMeth} implementation of the \textsc{L-BFGS-B} algorithm\cite{l-bfgs-b} to find the parameter values that minimize the $\chi^2_{\mathrm{mod}}$ test statistic. To ensure that the minimization will always converge to the global optimum for any true value of the physics parameters, pseudo-data without statistical fluctuations (also referred to as an \emph{Asimov} test set) is produced on a grid spanning all values that are not strongly excluded by other experiments and a fit is run for each grid point. Since there are no statistical fluctuations, the fit is expected to always converge exactly to the injected true value. As can be seen in the result shown in \reffig{three-flavor-asimov}, the convergence of the minimizer is robust everywhere. 
\begin{figure}
    \centering
    \includegraphics[width=0.9\linewidth]{figures/measurement/three_flavor/asimov_test/inject_recover_map.png}
    \caption{Asimov inject/recover test result for the three-flavor oscillation analysis.}
    \label{fig:three-flavor-asimov}
\end{figure}

\subsubsection{Ensemble tests}
\label{sec:three-flavor-ensemble}
To get expected distributions of the test statistic and parameter fluctuations, the analysis is run on an ensemble of fluctuated pseudo-data. For every trial of the ensemble, the expectation value in every analysis bin is first drawn from a normal distribution centered at the MC expectation with a standard deviation corresponding to the MC uncertainty. Using these sampled expectation values, the bin-count is drawn from a Poisson distribution independently in every bin. This sampling scheme ensures that the fluctuations reflect both the MC uncertainty and the Poisson fluctuations expected in data. A free fit is run on every trial, producing a set of best fit parameters, one value for the total $\Delta \chi^2_{\mathrm{mod}}$ test statistic as well as the contribution of every bin in the analysis histogram to this total value. The distribution for the fit parameters and their pull from their injected values from the ensemble is shown in \reffig{three-flavor-ensemble-param}. The distributions for all parameters are centered on the injected value, demonstrating that the fit is behaving robustly under the expected fluctuations. 

\begin{figure} 
    \centering
    \begin{subfigure}[t]{0.99\textwidth}
        \centering
        \includegraphics[width=0.99\textwidth]{figures/measurement/three_flavor/ensemble_pre_fit/ensemble_fitdist.png}
    \end{subfigure}
    \begin{subfigure}[t]{0.65\textwidth}
        \centering
        \includegraphics[width=0.99\textwidth]{figures/measurement/three_flavor/ensemble_pre_fit/ensemble_pull.png}  
    \end{subfigure}
  \caption{Distributions of fitted values of each parameter considered in the three-flavor analysis (top), and their corresponding pulls from nominal (bottom).
  \label{fig:three-flavor-ensemble-param}}
\end{figure}

\subsection{Results}

\subsubsection{Goodness of Fit}

Before looking at the best fit parameters of the real data fit, the goodness of fit is assessed using the total and bin-wise test statistic distributions. The distribution of the test statistic acquired from the ensemble described in \refsec{three-flavor-ensemble} is shown in \reffig{three-flavor-ts-ensemble} together with the observed test statistic from real data. The observed test statistic is found to lie very well within the expectation with a p-value of 32\%. The bin-wise contribution and the test statistic and its expected distribution are shown in \reffig{three-flavor-binwise-ts}. The histogram shows no apparent regions of particularly bad agreement between data and the the MC expectation, and the distribution of the bin-wise test statistic is in agreement with the distribution expected from pseudo-data trials.

\begin{figure}
    \centering
    \includegraphics[width=0.8\linewidth]{figures/measurement/three_flavor/ensemble_pre_fit/overall_ts_wings_trials.pdf}
    \caption{Observed test statistic value of the three-flavor oscillation analysis compared to expected distribution from ensemble.}
    \label{fig:three-flavor-ts-ensemble}
\end{figure}

\begin{figure*}
    \centering
    \includegraphics[height=0.22\linewidth]{figures/measurement/three_flavor/ensemble_pre_fit/real_fit_binwise_pulls_pre_bugfix.pdf}
    \includegraphics[height=0.22\linewidth]{figures/measurement/three_flavor/ensemble_pre_fit/binwise_ts_wings_trials.pdf}
    \caption{Contribution of every bin to the over-all test statistic in the three-flavor analysis (left) and their observed distribution compared to the expected distribution from pseudo-data trials (right).}
    \label{fig:three-flavor-binwise-ts}
\end{figure*}

\subsubsection{Test for un-physical mixing}
If the real data contains an under-fluctuation in the oscillation valley, it is possible that the fit prefers more than maximal $\nu_\mu$ disappearance, which is physically not possible. This tendency to fit un-physical magnitudes of $\nu_\mu$ disappearance is tested by running a fit in which the oscillation probabilities are calculated with a simplified two-flavor equation in which the scale of the oscillation, $\sin^2(2\theta_{23})$\todo{explain this simplification in the theory section and link here}, is replaced with a scaling factor that is allowed to float freely even to un-physical values where $\sin^2(2\theta_{23}) > 1$. If the true mixing angle is $\theta_{23}=45^\circ$, it is expected that such un-physical best-fit values can occur solely due to random Poisson fluctuations of the data. To quantify this expectation, another ensemble of trials is produced in the way described in \refsec{three-flavor-ensemble}, where the injected true mixing is maximal. The two flavor analysis is run on each trial to produce a distribution of expected values that is shown in \reffig{two-flavor-ensemble}\todo{draw the observed value in the figure}. The results show that, while the real data fit does indeed prefer a slightly un-physical $\nu_\mu$ disappearance, this preference still lies well within the expectation if the true mixing was assumed to be maximal.

\begin{figure}
    \centering
    \includegraphics[width=0.8\linewidth]{figures/measurement/three_flavor/ensemble_pre_fit/two_flav_ensemble_threshold.png}
    \caption{Observed best fit values of the two-flavor fit compared to the distribution from pseudo-data trials.}
    \label{fig:two-flavor-ensemble}
\end{figure}

\subsubsection{Measured Nuisance Parameter Values}
After first having established that the goodness of fit variables and the magnitude of the un-physical disappearance lie within the expectation from pseudo-data trials, the best-fit values of the nuisance parameters are revealed. The results are shown in \reftab{nuisance_params_fittedval}. The pull values show that all parameters fit comfortably within $1\sigma$ of their defined priors. The fit prefers a slightly harder cosmic ray spectrum and a larger muon background than initially expected. The optical efficiency of the DOMs fits to a slightly larger value than nominal with 106\%, while the ice properties stay very close to their initial values.

%The hole ice parameters prefer less acceptance of photons entering the DOMs directly from below, and the shape of their corresponding acceptance curve is close to the best fit result from LED flasher studies\todo{cite hole ice flasher studies} as shown in \reffig{hole-ice-flasher-comparison-three-flavor}.

\begin{table}
    \centering
    \caption{Fitted values of all nuisance parameters from the all-season three-flavor fit. The pull of the best fit value is shown for parameters with a defined prior.}
    \label{tab:nuisance_params_fittedval}
    \begin{tabular}{lSS} \toprule
        Parameter  & {Best Fit Value} &  {Pull ($\sigma$)} \\ \midrule
        delta index & 0.065 & 0.648 \\
        barr af Pi & 0.221  & 0.351 \\
        barr g Pi & -0.053  & -0.175 \\
        barr h Pi & -0.016  & -0.11 \\
        barr w K & 0.079  & 0.198 \\
        barr y K & 0.102  & 0.341 \\
        barr w antiK & -0.010  & -0.0244 \\
        $M_{A}^{CCQE}$ &  0.043 & 0.043 \\
        $M_{A}^{CCRES}$ & 0.607 & 0.607  \\
        dis csms & 0.027  & 0.0267 \\ 
        $A_{eff}$ scale & 0.823 &  \\
        NC normlisation & 1.121 &  0.605 \\
        DOM efficiency & 1.065  & 0.647 \\
        hole ice p0 & -0.267  &  \\
        hole ice p1 & -0.042  &  \\
        ice absorption & 0.973  &  \\
        ice scattering & 0.988 &  \\ 
        Weight scale & 1.371  &  \\
        \bottomrule
    \end{tabular}
\end{table}

% \begin{figure}
%     \centering
%     \includegraphics[width=0.8\linewidth]{figures/measurement/three_flavor/results/hole_ice_curve_best_fit.png}
%     \caption{Acceptance curve corresponding to the best-fit hole ice parameters of the three-flavor analysis compared to the baselind and the results of IceCube flasher studies.}
%     \label{fig:hole-ice-flasher-comparison-three-flavor}
% \end{figure}

\subsubsection{Oscillation parameters}
The fitted values for the three-flavor oscillation parameters are $\theta_{23} = 45.3639$ and $\Delta m^2_{31} = 2.47996 \times10^{-3}$ eV$^2$, which corresponds to $\sin^2\theta_{23} = 0.505$ and $\Delta m^2_{32} = 2.41 \times10^{-3}$ eV$^2$. The 90\% C.L. allowed region for these parameters is shown in \reffig{real_data_contour_three_flavor} compared to measurements from other experiments. The observed 90\% range for $\theta_{23}$ is [40.866, 49.685] and is slightly smaller than the expected Asimov contour at the best fit point. To make sure that this is compatible with random fluctuations, the likelihood is profiled over $\sin^2(\theta_{23})$ for 1000 pseudo-data trials. \reffig{mixing_brazil_band_three_flavor} shows the  68\% (90\%) intervals of the test statistic at each point of the scan over all trials. The observed contour is fully contained in the 68\% band, demonstrating that the narrowed 90\% range for $\theta_{23}$ is fully compatible with expected data fluctuations.

\begin{figure}
    \centering
        \includegraphics[width=0.9\textwidth]{figures/measurement/three_flavor/results/real_data_contour.png}  
  \caption{Contours showing the 90\% C.L. allowed region for the physics parameters of the three-flavor analysis. The red solid(dash) contours represent the observed(expected) sensitivity. The cross shows the best fit value. The bottom and right plots show the 1D likelihood profiles for $sin^2\theta_{23}$ and $\Delta m^2_{32}$, respectively. 
  \label{fig:real_data_contour_three_flavor}}
\end{figure}

\begin{figure}
    \centering
        \includegraphics[width=0.8\textwidth]{figures/measurement/three_flavor/results/mixing_brazil_band.pdf}  
  \caption{Observed contour in $\sin^2(\theta_{23})$ (red) compared to the Asimov expectation (black) and the distribution of 1000 pseudo-data trials (yellow and green bands) produced at the best fit point of the three-flavor oscillation analysis. The observed contour is fully contained within 68\% of the fluctuations of the trials.
  \label{fig:mixing_brazil_band_three_flavor}}
\end{figure}

\subsubsection{Likelihood Coverage}

When drawing the 90\% exclusion contour shown in \reffig{real_data_contour_three_flavor}, it is assumed that Wilks' theorem holds, that is, the distribution of the test statistic follows a $\chi^2$ distribution with two degrees of freedom. If this assumption holds, then the value of the test statistic should lie below the 90\% threshold for 90\% of repeated independent measurements. The relationship between the expected percentiles and the true distribution is referred to as \emph{coverage}. If more than 90\% of repeated measurements fall below the 90\% threshold, the likelihood is said to be \emph{over-covering}. In the reverse case where fewer than 90\% of repeated measurements fall below the 90\% threshold, the likelihood is said to be \emph{under-covering}. The coverage of the likelihood may change depending on the assumed true parameter values. For this measurement in particular, it is expected that the likelihood should over-cover near maximal mixing, because the mixing angle can no longer provide a full degree of freedom. To test the coverage for particular values of $\theta_{23}$ and $\Delta m^2_{31}$, pseudo-data is generated where these values are injected as true values. The bin counts of the pseudo-data are Poisson-fluctuated to create an ensemble of trials. Then, one free fit is run, and another fit where the physics parameters are fixed to their true values. The coverage is then evaluated by counting the fraction of trials for which $\Delta \chi^2_{\mathrm{mod}}$ between these two fits is smaller than the 90\% threshold given by Wilks' theorem. The results are shown in \reffig{three-flavor-coverage} for a range of points in mixing angle and mass splitting. As expected, the likelihood is over-covering near maximal mixing, while there is very little dependence of the coverage on the mass splitting. The likelihood is over-covering for all points in mass splitting in the right panel of \reffig{three-flavor-coverage}, because the injected mixing angle was at the best fit point of the analysis, which is very close to maximal. In conclusion, the 90\% contours shown in \reffig{real_data_contour_three_flavor} are slightly over-conservative in the region close to maximal mixing.

\begin{figure*}
    \centering
    \includegraphics[width=0.45\linewidth]{figures/measurement/three_flavor/coverage_test/coverage_dm_v3.pdf}
    \includegraphics[width=0.45\linewidth]{figures/measurement/three_flavor/coverage_test/coverage_t23_v3.pdf}
    \caption{Fraction of trials below the 90\% threshold expected from Wilks' theorem for a range of points in mass splitting (left) and mixing angle (right).}
    \label{fig:three-flavor-coverage}
\end{figure*}

\section{Search for eV-scale Sterile Neutrinos}
\label{sec:sterile-measurement}

This analysis assumes the extended 3+1 neutrino PMNS framework, with the parameters to be constrained being the oscillation parameters $\theta_{24}$ and $\theta_{34}$. The three-flavor atmospheric oscillation parameters and the CP-violating phase $\delta_{24}$ are treated as nuisance parameters. The mass splitting of the additional mass eigenstate is fixed at $\Delta m^2_{41}=1\;\mathrm{eV^2}$. Because the oscillations at Earth-scale baselines happen on much smaller energy scales than can be resolved by DeepCore, the analysis effectively becomes indifferent to the magnitude of $\Delta m^2_{41}$ and constrains the values of $\theta_{24}$ and $\theta_{34}$ based on the \emph{averaged} oscillation effect. As a consequence, constraints calculated based on the assumption that $\Delta m^2_{41}=1\;\mathrm{eV^2}$ are still valid even if the true mass splitting is much larger. This holds true up to mass splitting values of about $m^2_{41}\gtrapprox100\;\mathrm{eV^2}$, where the heavy mass eigenstate becomes so much slower than the light eigenstates that it would no longer interfere with them and decohere\sidecite{atmo_decoherence}.

\subsection{The 3+1 model}
This analysis probes the "3+1" oscillation model, in which a fourth "sterile" (i.e. non-interacting) neutrino flavor eigenstate $\nu_s$ and mass eigenstate $\nu_4$ with mass splitting $\Delta m^2_{41}$ is added to the standard three-flavor model. This state The PNMS mixing matrix is extended by a fourth row and column, which is parametrized with additional mixing angles $\theta_{14}$, $\theta_{24}$, $\theta_{34}$ and CP violating phases $\delta_{14}$ and $\delta_{24}$ as
\begin{align*}
    U_{3+1} =&\begin{pmatrix} 
    U_{e1}    & U_{e2}    & U_{e3}   &U_{e4}    \\
    U_{\mu1}  & U_{\mu2}  & U_{\mu3} &U_{\mu4}  \\
    U_{\tau1} & U_{\tau2} & U_{\tau3}&U_{\tau4} \\
    U_{s1} & U_{s2} & U_{s3}&U_{s4} \\
    \end{pmatrix}\\
    =&
    R_{34}(\theta_{34})
    \tilde{R}_{24}(\theta_{24}, \delta_{24})
    \tilde{R}_{14}(\theta_{14}, \delta_{14})
    R_{23}(\theta_{23})
    \tilde{R}_{13}(\theta_{13}, \delta_{13})
    R_{12}({\theta_{12})}\;,
\end{align*}
where $R_{kl}$ are rotation matrices. The goal of this analysis is to constrain the matrix elements $U_{\mu4}$ and $U_{\tau4}$ with magnitude $|U_{\mu4}|^2=\sin^2(\theta_{24})$ and  $|U_{\tau4}|^2=\sin^2(\theta_{34})\cos^2\theta_{24}$, respectively, via the measurement of $\nu_\mu$ disappearance.

\subsection{Atmospheric oscillations in the presence of an eV-scale sterile neutrino}
In the presence of an eV-scale sterile neutrino, the standard three-flavor oscillation pattern as a function of neutrino energy and zenith angle is distorted and overlaid with a much faster secondary oscillation pattern. Figure~\ref{fig:numu_survival_0.5eV2_full_range} shows the muon neutrino survival probability in the presence of a fourth mass eigenstate with $\Delta m^2_{41}=0.5\;\mathrm{eV^2}$ and $\theta_{24}=15^\circ$ as a function of the energy and cosine of the zenith angle, where $\cos(\theta_z)=-1$ indicates that the neutrino is coming directly from below and $\cos(\theta_z)=1$ directly from above. For up-going neutrinos, the oscillation pattern induced by $\Delta m^2_{41}$ is only resolved at energies of $>\mathcal{O}(1\;\mathrm{TeV})$. Below 100~GeV, the oscillation length reaches values of as low as $\mathcal{O}(\mathrm{km})$, which makes them unresolvable below the horizon where baselines are of $\mathcal{O}(10000\;\mathrm{km})$.

\subsubsection{Neutrino production height effects}
Above the horizon, the distance from the upper layers of the atmosphere where the neutrinos are produced to the detector is small enough to create a distinct oscillation pattern for a fixed production height as can be seen in the left panel of figure~\ref{fig:numu_survival_0.5eV2_full_range}. Because neutrino production heights vary over a range of tens of kilometers, it is necessary to average the oscillation probability over production heights to get a more realistic expectation. This is done analytically in \texttt{nuSQuIDS} by calculating the averaged vacuum oscillation probability over a uniform distribution. The right panel of figure~\ref{fig:numu_survival_0.5eV2_full_range} shows the oscillation probability with production heights averaged between 10~km and 30~km. The oscillation pattern above the horizon is no longer clearly resolvable, but an average disappearance effect for neutrino energies below 20~GeV remains. The uniform distribution that is assumed to calculate the averaged oscillation probabilities is of course not entirely realistic. For this reason, only events arriving at most slightly above the horizon ($\cos(\theta_z)<0.1$) are included in this analysis
\begin{figure}
    \centering
    \includegraphics[width=0.49\textwidth]{figures/measurement/sterile_analysis/physics/dm41_0.5eV2_th24_15deg_no_filter.png}
    \hfill
    \includegraphics[width=0.49\textwidth]{figures/measurement/sterile_analysis/physics/dm41_0.5eV2_th24_15deg_avg_height_10-30km.png}
    \caption{Muon neutrino survival probability in the presence of a fourth mass eigenstate with $\Delta m^2_{41}=0.5\;\mathrm{eV^2}$ and $\theta_{24}=15^\circ$ with a fixed production height of 20~km (left) and with production heights averaged between 10~km and 30~km (right).}
    \label{fig:numu_survival_0.5eV2_full_range}
\end{figure}

\subsubsection{Oscillation signal for large mass splittings}
In the mass-splitting range where $\Delta m_{41}\approx\mathcal{O}(1\,\mathrm{eV^2})$ and in the energy range of the event sample ($<150\;\mathrm{GeV}$), the presence of a sterile neutrino produces rapid oscillations overlaid on the standard three-flavor oscillation pattern as well as distortions to that pattern itself as shown in figure~\ref{fig:numu_survival_1eV2_analysis_binning_range} for a mixing angle of $\theta_{24}=15^\circ$. The oscillation frequency in energy is too large to be resolved by DeepCore, but the average effect still allows constraining the magnitudes of the $U_{\mu4}$ and $U_{\tau4}$. The precise value of $\Delta m^2_{41}$ has very little influence on the average amplitude of the oscillations and therefore cannot be recovered in this mass splitting regime. The oscillation averages stay approximately constant up to mass splitting values of well above $100\;\mathrm{eV^2}$ where decoherence effects begin to play a role\cite{atmo_decoherence}.
\begin{figure}
    \centering
    \includegraphics[width=0.8\textwidth]{figures/measurement/sterile_analysis/physics/dm41_1.0eV2_th24_15deg_avg_height_10-30km_ana_binning_range.png}
    \caption{$\nu_{\mu}$ survival probability at $\Delta m^2_{41}=1\;\mathrm{eV^2}$ and $\theta_{24}=15^\circ$}
    \label{fig:numu_survival_1eV2_analysis_binning_range}
\end{figure}

\subsubsection{Oscillation signal for small mass splittings}
For mass-splitting values of $\Delta m^2_{41}$ well below $1\;\mathrm{eV^2}$, the oscillation pattern is no longer completely averaged out. Figure~\ref{fig:numu_survival_0.1eV2_analysis_binning_range} shows the muon-neutrino survival probability in the presence of a sterile neutrino state with mass splitting $\Delta m^2_{41}=0.1\;\mathrm{eV^2}$ and mixing angle $\theta_{24}=15^\circ$. The highest oscillation minimum in energy and cosine of the zenith angle (upper right corner of the figure) is large enough to be resolvable with DeepCore. This makes it possible in principle to produce constraints of the mixing matrix elements as a function of the mass-splitting $\Delta m^2_{41}$, although this is beyond the scope of the analysis presented in this thesis.
\begin{figure}
    \centering
    \includegraphics[width=0.8\textwidth]{figures/measurement/sterile_analysis/physics/dm41_0.1eV2_th24_15deg_avg_height_10-30km_ana_binning_range.png}
    \caption{$\nu_{\mu}$ survival probability at $\Delta m^2_{41}=0.1\;\mathrm{eV^2}$ and $\theta_{24}=15^\circ$}
    \label{fig:numu_survival_0.1eV2_analysis_binning_range}
\end{figure}

\subsection{Nuisance oscillation parameters}
Besides the physics parameters $\theta_{24}$ and $\theta_{34}$ to be constrained by the analysis (at a fixed sterile mass splitting $\Delta m^2_{14}$), there are 4 additional mixing angles and 3 CP-violating phases in the 3+1 PNMS matrix as well as the mass splittings $\Delta m^2_{12}$ and $\Delta m^2_{13}$ that influence the oscillation probability. The solar and reactor angles $\theta_{12}$ and $\theta_{13}$ as well as the solar mass splitting $\Delta m^2_{12}$ are constrained by other experiments beyond the sensitivity of this analysis and are fixed at their current global best fit point\sidecite{nufit}. The effect of the standard three-flavor CP violating phase $\delta_{\mathrm{CP}}=\delta_{13}$ is negligible and it is fixed to zero. The mixing angle $\theta_{14}$ and the phase $\delta_{14}$ are also fixed to zero, since recent reactor data constrains $|U_{e4}|^2 = \sin^2(\theta_{14})$ to $\mathcal{O}(10^{-3})$, which is well below the sensitivity of this analysis\sidecite{global_unitarity_Hu}. The only nuisance parameters that remain free are the standard 3-flavor atmospheric oscillation parameters $\theta_{23}$ and $\Delta m^2_{13}$ as well as the sterile CP-violating phase $\delta_{24}$. The effect of $\delta_{24}$ is to shift the oscillation pattern of muon neutrinos as shown in figure~\ref{fig:sterile-cp-phase-effect}. The effect for \emph{anti}neutrinos runs in the opposite direction. Because neutrinos and antineutrinos are nearly indistinguishable in DeepCore, the combined effect of $\delta_{24}$ is a smearing of the oscillation minimum. Additionally, the sign of $\cos(\delta_{24})$ is approximately degenerate with the neutrino mass hierarchy effect. It's therefore expected that the analysis will produce very similar results for NO and IO when $\delta_{24}$ is free. Table~\ref{tab:oscillation-parameters} gives an overview over all oscillation parameters in the 3+1 model and their treatment in this analysis. 
\begin{figure}
    \centering
    \includegraphics[width=0.7\textwidth]{figures/measurement/sterile_analysis/physics/Muon_neutrino_survival_probability_with_steriles_1D.png}
    \caption{Muon neutrino survival probability for a directly up-going neutrino as a function of energy in the presence of sterile neutrinos at different values of the sterile CP violating phase.}
    \label{fig:sterile-cp-phase-effect}
\end{figure}

\begin{table}
    \centering
    \begin{tabular}{@{}cccp{0.35 \linewidth}@{}}\toprule
        \textbf{parameter} & \textbf{nominal value} & \textbf{fixed?} & \textbf{comment} \\
        \midrule
        $\theta_{14}$ & $0^\circ$ & fixed &  Constr. by reactor data \\
        $\theta_{24}$ & -- & free & Physics parameter\\
        $\theta_{34}$ & -- & free & Physics parameter\\
        $\theta_{12}$ & $33.82^\circ$ & fixed & Constrained by reactor and solar data \\
        $\theta_{13}$ & $8.61^\circ$ & fixed & Constrained by reactor and accelerator data \\
        $\theta_{23}$ & $49.6^\circ$ & free & Atm. mixing angle \\
        $\delta_{13}$ & $0^\circ$ & fixed &  Negligible effect \\
        $\delta_{14}$ & $0^\circ$ & fixed &  No effect when $\theta_{14} = 0^\circ$ \\
        $\delta_{24}$ & -- & free & Smears osc. minimum \\
        $\Delta m^2_{21}$ & $7.39\times10^{-5}\;\mathrm{eV^2}$ & fixed & Constrained by reactor and solar data \\
        $\Delta m^2_{31}$ & $2.525\times10^{-3}\;\mathrm{eV^2}$ & free & Atm. mass splitting \\
        $\Delta m^2_{41}$ & $1\;\mathrm{eV^2}$ & fixed & Averaged out above $1\;\mathrm{eV^2}$ \\
        \bottomrule
    \end{tabular}
    \caption{Oscillation parameters of the 3+1 model and their treatment in this analysis.}
    \label{tab:oscillation-parameters}
\end{table}

\subsection{Oscillation Probability Calculation with nuSQuIDS}
\label{sec:nusquids}

This analysis uses as customized version of \href{https://github.com/ts4051/nuSQuIDS}{nuSQuIDS} that is optimized for the integration into \href{https://github.com/icecube/pisa}{PISA}. The basic principle behind nuSQuIDS is to calculate state transition probabilities in the Interaction (Dirac) Picture of quantum mechanics, where the full, time dependent Hamiltonian is split into the time-independent vacuum oscillation part $H_0$ and the time-dependent interaction part $ H_1(t)$:
$$ H(t) = H_0 = H_1(t)$$ 
In this picture, operators evolve with $ H_0$ as
$$
\bar{O}_I(t)=e^{iH_0t}O_Se^{-iH_0t}\,,
$$
while state densities evolve with the interaction Hamiltonian $ H_1(t)$
$$ 
\partial_t\bar{\rho}_I(t)=-i[\bar{H}_{1, I}(t), \bar{\rho}_I(t)]\,.
$$

This state evolution is solved via numerical integration in nuSQuIDS, which is computationally expensive. However, because fast oscillations inside $ H_0$ only play a sub-leading role, this difficult calculation does not have to be performed at every point in the analysis space. It is sufficient to calculate state densities at a selected set of points, referred to as \emph{nodes}, and to then interpolate the densities between them. The fast oscillations between the nuSQuIDS nodes are recovered when the probabilities for each flavor, $i$, are projected out with the trace operation on the state density with the (time-evolved) projection operator for that state:
$$ 
p_i(t)=\mathrm{Tr}(\underbrace{\bar{\Pi}^{(i)}(t)}_{\mathrm{proj.\,op.}}\bar{\rho}_I(t))
$$

\subsubsection{Node placement}
The nodes where the difficult state integration is calculated need not be placed dense enough in the analysis space to resolve the fast oscillations due to sterile neutrinos, but they do need to resolve matter effects. The state densities change most rapidly (as function of energy) for neutrinos that traverse a lot of matter at low energies. Additionally, there is a sharp break at $\cos(\theta_{\mathrm{zenith}})=-0.84$ where neutrinos begin to pass through the core. For this reason, the nuSQuIDs nodes are concentrated in three places:
\begin{itemize}
    \item the energy region between 2 GeV and 10 GeV
    \item within a small interval around $\cos(\theta_{\mathrm{zenith}})=-0.84$
    \item the region below $\cos(\theta_{\mathrm{zenith}})=-0.84$
\end{itemize}
Figure~\ref{fig:nusquids-nodes} shows the optimized placement of nuSQuIDS nodes as black dots.

\begin{figure}
    \centering
    \includegraphics[width=0.7\textwidth]{figures/measurement/sterile_analysis/nusquids/0.1eV_sterile_only_height_avg_optim_nodes.png}
    \caption{Optimized placement of nuSQuIDS nodes (black dots) with extra dense node spacing in the three critical regions. The injected value for $\Delta m^2_{41}$ is $0.1\;\mathrm{eV^2}$.}
    \label{fig:nusquids-nodes}
\end{figure}

\subsubsection{Production height averaging}
At eV-scale mass splittings, oscillations are fast enough that significant oscillations occur even at 10-km-scale distances. If the production height is assumed to be fixed at an exact position, a strong oscillation pattern appears above the horizon that could falsely produce a very high sensitivity in the analysis, driven entirely by events above the horizon. In reality, production heights can vary within the atmosphere, which smears out the oscillations. For this analysis, an analytical averaging method was implemented in nuSQuIDS that assumes a uniform distribution of propagation distances between two points. The start and end point depends on the zenith angle and corresponds to the intersection of the neutrino path with a height of 10 km and 30 km, respectively.

\subsubsection{Low-pass filtering}
The fast oscillations in the presence of sterile neutrinos are filtered with a low-pass filter both during the state evolution and the calculation of probabilities. This step dramatically increases the speed at which oscillation probabilities can be evaluated.

Vacuum oscillations enter the differential equation governing the state evolution via the time evolution of the interaction Hamiltonian $\bar{H}_{1, I}(t)$. At low energies they cause tiny, but extremely fast oscillations of the time derivative that the numerical integrator has to keep track of by drastically reducing the step size, slowing down the calculation. To mitigate this problem and increase performance, a low-pass filter is applied when calculating the RHS of the differential equation.

\begin{figure}
    \centering
    \includegraphics[width=0.7\textwidth]{figures/measurement/sterile_analysis/nusquids/Dm41_0.5eV2_th24_15deg_avg_height_10-30km_lp_belowhor.png}
    \caption{Muon neutrino survival probability in the presence of a sterile neutrino after application of both height averaging and low-pass filtering.}
    \label{fig:nusquids-low-pass-filtering}
\end{figure}

In the presence of sterile oscillations, transition probabilities usually have to be calculated for every single simulated event to average them out in the analysis binning. Because the MC set has millions of events, doing this would be very expensive even when using nuSQuIDs' state interpolation feature. In this analysis, a low-pass filter as a function of energy is applied when the transition probabilities are projected out from the state densities to average out very fast oscillations. With this filtering, it is possible to calculate oscillation probabilities on a fine binning with ~20k bins. One caveat is that this filtering is not appropriate to apply above the horizon because propagation distances there are short enough that oscillation probabilities don't average out completely. It is therefore only applied below the horizon as shown in figure~\ref{fig:nusquids-low-pass-filtering}.


\subsection{Sterile signal in analysis binning}
The change in bin counts with respect to the nominal expectation for different combinations of sterile mixing angles at $\Delta m^2_{41}=1\;\mathrm{eV^2}$ is shown in figure~\ref{fig:oscillation-effects-ana-binning}. While a pull in only $\theta_{34}$ by $20^\circ$ has only a very small effect (see middle panels of figure~\ref{fig:oscillation-effects-ana-binning}), the combination of both angles can greatly amplify the signal. The CP-violating phase $\delta_{24}$ only plays a role when both angles $\theta_{24}$ and $\theta_{34}$ are non-zero. The sensitivity of this analysis to $\theta_{34}$ is entirely due to matter effects experienced by neutrinos passing through the dense core of the Earth.

\begin{figure}
    \centering
    \includegraphics[width=0.7\textwidth]{figures/measurement/sterile_analysis/oscillation_signal/pull_theta_combinations_20deg_dcp24.pdf}
    \caption{Signal in the analysis binning produced by different combinations of $\theta_{24}$, $\theta_{34}$, and $\delta_{24}$ as a fraction of the Poisson error in each bin. The mass splitting of the sterile state is $\Delta m^2_{41}=1\;\mathrm{eV^2}$.}
    \label{fig:oscillation-effects-ana-binning}
\end{figure}

\subsection{Selection of Free Parameters}

To reduce the computational cost of optimizing the likelihood in a high-dimensional space, the impact of each nuisance parameter in consideration is assessed and its value fixed if it is found to be negligible. Since the standard three-flavor oscillation model is a nested hypothesis within the 3+1 model, the parameter selection described in section~\ref{sec:std-osc-free-parameters} is taken as a starting point for the parameter selection of this analysis.
Starting from that selection, a test is run to determine if a paraemter is entirely dominated by its prior. For this test, 200 trials are run where all nuisance parameters are sampled randomly according to their prior. Then,  Asimov pseudo-data is produced at that point, and this pseudo-data is fit back with the default fit settings. The resulting pairs of true injected parameter values and fitted values for every free parameter are shown in a scatter plot in figure~\ref{fig:parameter-ensemble-result}. If a parameter is entirely dominated by its prior, it will fit back to its nominal value regardless of the injected value. For parameters where this is the case, their value is fixed to their nominal value during a fit. Parameters to which this applies are framed in red in figure~\ref{fig:parameter-ensemble-result}. The framed parameters are (from top to bottom, and from left to right in the bottom row): \texttt{barr\_w\_K}, \texttt{barr\_w\_antiK}, \texttt{barr\_g\_Pi}, \texttt{pion\_ratio}, \texttt{barr\_h\_Pi}. The test shown in the figure was run before the priors on \texttt{barr\_i\_Pi}, \texttt{barr\_z\_K}, and \texttt{barr\_z\_antiK} have been inflated as described in section~\ref{section:flux_systs}. After that change,  \texttt{barr\_i\_Pi} was also added to the set of free parameters of the analysis. The full list of free parameters with their respective ranges and priors is shown in table~\ref{tab:all-parameters}.

\begin{figure*}
    \centering
    \includegraphics{figures/measurement/sterile_analysis/param_selection/Screen Shot 2022-05-30 at 13.12.49.png}
    \caption{Result of the ensemble test with randomly injected nuisance parameters. This test was run before the priors on \texttt{barr\_i\_Pi}, \texttt{barr\_z\_K}, and \texttt{barr\_z\_antiK} have been inflated as described in section~\ref{section:flux_systs}. Parameters framed in red have been deemed to be negligible for the analysis. The color scale shows the likelihood difference between the free fit and a fit in which the physics parameters ($\theta_{24}$, $\theta_{34}$ have been fixed to their true value. If this number is negative, the trial is circled in red. In such cases, the minimizer failed to find the correct global optimum in the free fit.}
    \label{fig:parameter-ensemble-result}
\end{figure*}

\begin{table}
    \centering
    \begin{tabular}{cccl}\toprule
        \textbf{parameter} & \textbf{nominal value} & \textbf{range} & \textbf{prior} \\
        \midrule
        $\theta_{24}$ &    $0^\circ$ &  $0^\circ$ to $45^\circ$  & uniform \\
        $\theta_{34}$ &    $0^\circ$ &  $0^\circ$ to $45^\circ$  & uniform \\
        $\theta_{23}$ & $49.6^\circ$ & $20^\circ$ to $70^\circ$  & uniform \\
        $\delta_{24}$ &    $0^\circ$ &  $0^\circ$ to $180^\circ$ & uniform \\
        $\Delta m^2_{31}$ & $2.525\times10^{-3}\;\mathrm{eV^2}$ & $(2\,\rm{to}\, 3)\times10^{-3}\;\mathrm{eV^2}$ & uniform \\
        \midrule
        $\Delta \gamma_\nu$ & 0.0 & $-3\sigma$ to $+3\sigma$ &  $\sigma=0.1$  \\
        $\rm{Barr}, \: a-f_{\pi^+}$ & 0.0 & $-3\sigma$ to $+3\sigma$ &  $\sigma=0.63$  \\
        $\rm{Barr}, \: i_{\pi^+}$ & 0.0 & $-3\sigma$ to $+3\sigma$ &  $\sigma=0.61$  \\
        $\rm{Barr}, \: y_{K^+}$ & 0.0 & $-3\sigma$ to $+3\sigma$ &  $\sigma=0.3$  \\
        \midrule
        $M_{A,QE}$ & 0.0 & $-2\sigma$ to $+2\sigma$ &  $\sigma=1.0$  \\
        $M_{A,res}$ & 0.0 & $-2\sigma$ to $+2\sigma$ &  $\sigma=1.0$  \\
        DIS & 0.0 & $-3\sigma$ to $+3\sigma$ &  $\sigma=1.0$  \\
        $N_{\nu, NC}$ & 1.0 & 0.5 to 1.5 &  $\sigma=0.2$  \\
        \midrule
        $N_{\nu}$ & 1.0 & 0.5 to 2.0 & uniform \\ 
        $N_{\mu}$ & 1.0 & 0 to 3 & uniform \\ 
        \midrule
        $\epsilon_{\rm{DOM}}$ & 1.0 & 0.85 to 1.15 & $\sigma=0.1$ \\ 
        $\rm{ice \; absorption}$ & 1.0 & 0.85 to 1.15 & $\sigma=0.05$ \\ 
        $\rm{ice \; scattering}$ & 1.05 & 0.85 to 1.15 & $\sigma=0.1$ \\
        $\rm{hole \; ice}, \: p_0$ & 0.101569 & -1.1 to 0.5 & uniform \\
        $\rm{hole \; ice}, \: p_1$ & -0.049344 & -0.15 to 0.1 & uniform \\
        \bottomrule
    \end{tabular}
    \caption{List of all free parameters in this analysis with their respective ranges and priors (if applicable).}
    \label{tab:all-parameters}
\end{table}


\setchapterpreamble[u]{\margintoc}
\chapter{Summary and Future Outlook}

\section{Summary of Results}
\label{sec:summary}

This work presented two oscillation measurements of atmospheric neutrinos using the DeepCore sub-array of the IceCube Neutrino Observatory. Both measurements were based on a newly developed data sample of \num{21914} well-reconstructed, track-like events in the energy range between \SI{5}{\giga\electronvolt} and \SI{150}{\giga\electronvolt}. The selection process for these events was described in \refch{data-sample} and consisted of several filtering steps that remove background due to detector noise and atmospheric muons. At the final filter level, the contribution of muon background was reduced to $\sim2\%$ and the noise background was entirely negligible and the sample consisted almost entirely of neutrino interactions. The zenith angle of every event was reconstructed using a geometric method and the energy of every interactions was estimated with a likelihood that only takes into account whether or not a sensor in the array has observed light. %While these reconstruction methods are less accurate than more sophisticated methods that were developed more recently, they are also less affected by the precise modelling of the detector properties.
Several quantities produced by the reconstruction algorithms such as the length of the reconstructed track and the goodness-of-fit were fed into a Boosted Decision Tree (BDT) that calculated a particle ID (PID) number for every event that estimates the probability that it originated from a muon neutrino interaction. Both data and simulated pseudo-data were binned in zenith angle, energy, and PID. A minimization algorithm then calculated the best-fit neutrino oscillation parameters by re-weighting the simulated events to match the histograms of the observed data as closely as possible.

\subsection{Three-Flavor Oscillations}
\label{sec:summary-three-flavor}

The first data analysis shown in this work was a measurement of the atmospheric mixing angle and mass splitting in the three-flavor neutrino oscillation model assuming normal mass ordering. This measurement is complementary to oscillation analyses of accelerator neutrinos and the most precise measurement using atmospheric neutrinos to date. The result
\begin{align*}
    \sin^2\theta_{23} &= 0.507_{-0.053}^{+0.050}\\
    \Delta m^2_{32} &= 2.42_{-0.75}^{+0.77} \times10^{-3}\;\mathrm{eV}^2.
\end{align*}
is consistent with previous DeepCore measurements and current global fits.

\subsection{Sterile Neutrino Search}
\label{sec:summary-sterile-osc}

The second measurement presented in this work was the search for eV-scale sterile neutrinos. The search was performed under the "3+1" model, where the PNMS matrix is extended by an additional row and column to accomodate the mixing of a fourth neutrino mass eigenstate. The measurement used the same data sample as the three-flavor fit and the same likelihood function calculated in an identical binning. The major technical difference between the analyses was that the neutrino oscillation calculation for the sterile neutrino model was done using a customized version of the \textsc{nuSQuIDS} package. It allowed a computationally efficient way of calculating flavor transition probabilities in the presence of a heavy fourth mass eigenstate that produces a very fast oscillation pattern. The customizations that were developed specifically for this work were the addition of low-pass filters that can analytically produce oscillation probabilities where the contributions due to that heaviest mass eigenstate are averaged out. Another major technical development that separates the sterile neutrino analysis from the three-flavor fit is the introduction of a novel method of incorporating uncertainties in the detector response in a way that is fully decoupled from neutrino oscillation probabilities.

\subsubsection{Result}
The analysis constrained the $|U_{\mu4}|$ and $|U_{\tau4}|$ elements of the extended PNMS matrix to\todo{extract precise numbers}
\begin{equation}
    \begin{aligned}
        \abs{U_{\mu4}}^2 &< X \\
        \abs{U_{\tau4}}^2 &< Y \\
    \end{aligned}
\end{equation}
at 90\% C.L. while marginalizing over the CP violating phase $\delta_{24}$. This result is valid for both normal ordering and inverted ordering thanks to the approximate degeneracy between the mass ordering and the sign of $\cos(\delta_{24})$. The confidence limits were calculated using Wilks' theorem assuming two degrees of freedom. Spot-checks of the likelihood distributions showed that these limits err on the conservative side. More stringent limits could be obtained by correcting the critical values of the likelihood according to Feldman and Cousins\cite{Feldman_1998}, but the computational expense was deemed too high and the conservative limits sufficient for the purposes of this work. This result is a substantial improvement over the previous DeepCore result and provides the most stringent limit on $\abs{U_{\tau4}}^2$ to date. The constraint on $\abs{U_{\mu4}}^2$ is competitive with other experiments and has the potential to further increase the tension between appearance and disappearance datasets in global fits of the 3+1 sterile neutrino model described in \refsec{global-anomalies}.

\section{Future Outlook}

The measurement presented in this thesis used only the fraction of the available DeepCore data that could be reconstructed with older reconstruction methods that are optimized for robustness. The purpose of this analysis was not to achieve the highest possible sensitivity, but to verify the integrity of the newly developed data selection techniques and to act as a test bed for newly developed methods of treating uncertainties in the detector properties. Once the tools that were developed for this analysis are combined with more capable reconstruction methods, this new data sample will provide constraints on oscillation parameters that are more stringent than those of any other atmospheric neutrino oscillation measurement and rival the precision of the most recent accelerator experiments.

\subsection{Reconstruction Improvements}
A substantial increase in the sensitivity of both analyses could be achieved by using the table-based reconstruction method described in \cite{lowen-reco-paper}. This algorithm can provide an estimate for the energy and zenith angle for nearly all events passing the Level 5 event filter described in \refsec{data-processing} and has a much higher resolution than the reconstruction method used in this analysis, substantially increasing the statistical power of the analysis. The projected sensitivity that could be achieved with this method is shown in \reffig{retro-sensitivity}.\todo{Compare SANTA sensitivity, add retro sensitivity} With the increased statistical power and resolution also comes a larger burden to accurately model the properties of the neutrino flux, particle interactions and detector properties. The work to bring data and simulation into agreement with the more powerful sample is still ongoing at the time of writing this thesis.
\begin{figure}
    \centering
    \includegraphics[width=0.7\linewidth]{figures/summary/Sterile_mixing_sensitivity_90pct_retro.png}
    \caption{Projected sensitivity of the sterile neutrino search when using the table-based reconstruction method.\label{fig:retro-sensitivity}}
\end{figure}

\subsection{Treatment of systematic uncertainties}
\subsubsection{Detector uncertainties}
In order to perform the sterile oscillation search, a novel method of interpolating between different MC sets has been developed that allows re-weighting individual events based on changes in the detector response. The need for this development arose from the discovery that the effect of a change in detector properties in each bin of the analysis is coupled to the values of the neutrino oscillation parameters. In the three-flavor analysis, this problem was addressed by interpolating the gradients of the bin counts with respect to detector parameters over a grid of points in the mass splitting parameter $\Delta m^2_{31}$ as described in \refsec{hypersurfaces}. For the sterile analysis, however, the dimensionality of the grid needed to cover all oscillation parameters made this approach unfeasible. The new method, described in \refsec{ultrasurfaces}, completely decouples the detector response from oscillation parameters and thereby eliminates the need for any interpolation. This property makes it universally applicable to any oscillation study probing arbitrarily complex oscillation phenomena. In addition, the method opens the door to using unbinned likelihood functions in future analyses, which was impossible in the past because previous methods such as the one described in \sidecite{multisim} are inherently binned. In the near future, this new method of calculating event-wise weights based on posterior estimates will be expanded upon and has the potential to become the new standard treatment for detector uncertainties for neutrino telescopes.

\subsubsection{Atmospheric flux}
The treatment of atmospheric neutrino flux uncertainties used in this work is based on the Barr blocks method that was published in 2006\cite{Barr2006}. Since then, new methods of modeling variations of the atmospheric flux have been developed that decrease the overall relative uncertainty by up to 40\% and also provide a data-driven parametrization of the flux variations\sidecite{Fedynitch_2022}. Incorporating these developments into the oscillation data analysis has the potential to improve the agreement between data and simulation and to increase the sensitivity.

\chapter{Conclusion}

The phenomenon of neutrino oscillations is the first clear evidence of physics beyond the standard model and raises several questions that could hitherto not be answered. It is clear that neutrinos must have non-zero masses for the phenomenon to occur, but it is unclear how large those masses are and how they acquire these masses in the first place. The data analysis presented in this work probed the oscillation pattern of neutrinos that are produced in the atmosphere of the Earth for signs of heavy neutrino mass eigenstates that do not interact with the Weak force. These additional mass eigenstates could be a byproduct of the process that generates neutrino masses and might explain the smallness of the masses of the active neutrino flavors. The search for these states presented in this work was made under the assumption that the mass splitting between them and the active states is of $\order{\SI{1}{\electronvolt\squared}}$ based on experimental anomalies that have been found in some accelerator experiments and measurements of the neutrino flux originating in radioactive decay.

The results of this work are the first to be obtained using a newly developed eight-year data sample of IceCube's DeepCore sub-array. This sample is the result of an effort by the IceCube Collaboration over several years to improve the detector calibration.


\appendix % From here onwards, chapters are numbered with letters, as is the appendix convention
\pagelayout{wide} % No margins
\addpart{Appendix}
\pagelayout{margin} % Restore margins
% image credits!
\setchapterstyle{lines}
\labch{appendix}

\setchapterstyle{lines}
\chapter{Mathematical derivations}
\label{ch:derivations}

Detailed mathematical derivations that are too involved for the main text go here.


\section{Detector systematics via Likelihood-free Inference}

\subsection{Linear correction for KNN bias}
\label{apx:knn-correction}
The probability estimate from a K-Neighbors classifier with a large number of neighbors shows a systematic bias due to the fact that the distribution of the samples in each neighborhood are not uniform.
This bias is corrected by reweighting the samples inside the neighborhood in such a way that their "center of gravity" is located at the point where the KNN is queried.
This is done under the assumption that the weight should be a linear function of the coordinate of each sample after subtracting the query coordinate. In one dimension, this leads to the condition
\begin{equation}
    \sum_{i\in \mathcal{N}} w_i x_i = \sum_{i\in \mathcal{N}} (1 + gx_i) x_i = 0\;,\label{eq:knn-correction-condition}
\end{equation}
where the index $i$ runs over all samples in the neighborhood around the query point, $w_i$ is the weight to be assigned to each sample and $x_i$ is the position of the sample relative to the query point.
The weight is replaced in the second equality with the linear function $w_i = (1 + gx_i)$ with the gradient $g$.
The condition in \refeq{knn-correction-condition} is solved for the gradient as
\begin{equation}
\begin{aligned}
    \sum_{i\in \mathcal{N}} (1 + gx_i) x_i &= 0 \\
    \Leftrightarrow \sum_{i\in \mathcal{N}} x_i &= -g \sum_{i\in \mathcal{N}} x_i^2 \\
    \Leftrightarrow -\frac{\sum_{i\in \mathcal{N}} x_i}{\sum_{i\in \mathcal{N}} x_i^2} &= g\;.
\end{aligned}
\end{equation}
The gradient $g$ thus found is the ratio of the first and second moments of the distribution of samples in the neighborhood.
The non-uniformity of the distribution of events around the query point is corrected to first order by applying the weights $w_i = (1 + gx_i)$ to every event.
In higher dimensions, the same calculation is done independently in each dimension and the sample weights for all dimensions are multiplied.
%
%\subsection{Implicit marginalization}
%\label{apx:implicit-marginalization}
%As the input to our classifier that produces re-weight ratios for detector uncertainties, we only use variables that are either used to re-bin events, or those that are used as input into flux and oscillation weights.
%This set of variables is sufficient to fully decouple the detector response from other physics weights, even if other variables (such as e.g.
%track length) can in principle influence the detector response to any particular event.
%The reason why this set of variables is sufficient is that the classifier \emph{implicitly marginalizes} over all variables that are not included.



%----------------------------------------------------------------------------------------
\backmatter % Denotes the end of the main document content
\setchapterstyle{plain} % Output plain chapters from this point onwards
\chapter*{Acknowledgements}
\addcontentsline{toc}{chapter}{Acknowledgements} % Add the preface to the table of contents as a chapter

Here I will thank everyone who helped me along the way.

\begin{flushright}
	\textit{Alexander Trettin}
\end{flushright}
\chapter*{Panel Images}
\addcontentsline{toc}{chapter}{Panel Images} % Add the preface to the table of contents as a chapter

Here I will give credit to the images I might use as chapter title headers.

% Chapter \ref{ch:theory} - \textbf{The Crab Nebula at X-ray/optical/infra-red/radio wavelengths}. Credit: NASA/CXC/SAO (X-ray), Paul Scowen and Jeff Hester (Arizona State University) and the Palomar Observatories (optical), 2MASS/UMass/IPAC- Caltech/NASA/NSF (infrared), and NRAO/AUI/NSF (radio)

%----------------------------------------------------------------------------------------
%	BIBLIOGRAPHY
%----------------------------------------------------------------------------------------
% The bibliography needs to be compiled with biber using your LaTeX editor, or on the command line with 'biber main' from the template directory
\defbibnote{bibnote}{Here are the references in citation order.\par\bigskip} % Prepend this text to the bibliography
\printbibliography[heading=bibintoc, title=Bibliography, prenote=bibnote] % Add the bibliography heading to the ToC, set the title of the bibliography and output the bibliography note

\newpage
TODO: Declare independent work.
% \includegraphics{declaration_independent_work}

\end{document}
