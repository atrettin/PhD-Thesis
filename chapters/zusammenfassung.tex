\chapter{Zusammenfassung}

Neutrinooszillationen sind das einzige Phänomen jenseits des Standardmodells, das experimentell mit hoher statistischer Signifikanz bestätigt wurde.
Diese Arbeit präsentiert eine Messung der atmosphärischen Neutrinooszillationen unter Verwendung von acht Jahren an Daten, die zwischen 2011 und 2019 vom IceCube DeepCore-Detektor aufgenommen wurden.
Die Ereignisauswahl wurde im Vergleich zu früheren DeepCore-Messungen verbessert, wobei ein besonderes Augenmerk auf ihre Robustheit gegenüber systematischen Unsicherheiten in den Detektoreigenschaften gelegt wurde.
Die Oszillationsparameter werden über eine Maximum-Likelihood-Fit an gebinnte Daten in der gemessenen Energie und Zenitwinkel geschätzt, wobei die Erwartungswerte aus gewichteten simulierten Ereignissen abgeleitet werdem.
Diese Arbeit diskutiert den Simulations- und Datenauswahlprozess sowie die statistischen Methoden, die verwendet werden, um einen genauen Erwartungswert unter variablen Detektoreigenschaften und anderen systematischen Unsicherheiten zu liefern.
Die Messung wird zunächst unter Verwendung des Standardmodells der Drei-Flavor-Oszillation durchgeführt, wobei das atmosphärische Massensplitting und der Mischwinkel auf $\Delta m^2_{32} = 2.42_{-0.75}^{+0.77} \times10^{-3};\mathrm{eV}^2$ und $\sin^2\theta_{23} = 0.507_{-0.053}^{+0.050}$ geschätzt werden.
Das Drei-Flavor-Modell wird dann um einen zusätzlichen Masseneigenzustand erweitert, der einem sterilen Neutrino mit Massensplitting $\Delta m^2_{41} = 1;\mathrm{eV}^2$ entspricht und mit den aktiven $\nu_\mu$- und $\nu_\tau$-Flavorzuständen mischen kann.
Es wird kein signifikantes Signal eines sterilen Neutrinos beobachtet, und die Mischungsamplituden zwischen den sterilen und aktiven Zuständen werden auf $|U_{\mu 4}|^2 < 0.0534$ und $|U_{\tau 4}|^2 < 0.0574$ bei 90\% C.L.
begrenzt.
Diese Grenzwerte sind um den Faktor zwei bis drei strenger als das vorherige DeepCore-Ergebnis, und die Einschränkung von $|U_{\tau 4}|^2$ ist die stärkste der Welt.
