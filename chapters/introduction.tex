\setchapterstyle{kao}
\setchapterpreamble[u]{\margintoc}
\chapter{Introduction}
\labch{intro}

The neutrino is a nearly massless and electrically neutral particle whose existence was first conjectured by Pauli in the 1930s to explain the fact that the energy spectrum of radiation from nuclear beta decay was continuous. If the only particles produced by the decay were the nucleus and the beta particle, then the energy of the beta particle would have been fixed by conservation of energy and momentum. It was a bold proposition at the time, because there were no observable traces of this particle and Pauli himself feared that it might be unobservable. His fears proved unwarranted when the first direct experimental observation of neutrinos was made in 1956 by Cowan and Reines\cite{cowan:1956} by detecting the distinct signature of so-called "inverse" beta-decay reaction
\begin{equation}
    \bar{\nu} + p \rightarrow n + e^+
\end{equation}
inside a water tank close to a nuclear reactor. About a decade later, in 1960, the Homestake experiment was able to measure the flux of neutrinos from the Sun. However, the observed rate of electron neutrinos was lower than what was expected from nuclear fusion reactions inside the Sun, leading to the \emph{solar neutrino problem}. The muon neutrino was discovered in 1962 by an experiment at the Brookhaven National Laboratory\cite{PhysRevLett.9.36} and the tau neutrino in 2000 by the DONUT experiment at Fermilab\cite{Kodama_2001}, completing the current picture of the Standard Model (SM) with three generations of leptons. In this model, neutrinos are described as spin-$\nicefrac{1}{2}$ fermions that only interact via the Weak nuclear force. The Weak force only interacts with left-handed chiral neutrinos and right-handed chiral antineutrinos, and no other neutrino states have so far been observed. This description requires that neutrinos are massless, because the Higgs mechanism that produces the masses of all other particles in the SM requires an interaction involving both right-handed and left-handed chiral fields.

The solution to the solar neutrino problem accepted today is that neutrinos have the ability to oscillate from one flavor to another. In this way, the electron neutrinos that are initially produced by the Sun can turn into a different flavor to which the Homestake experiment was not sensitive. This phenomenon of \emph{neutrino oscillations} was first demonstrated by the Super-Kamiokande experiment for muon neutrinos that are produced in the Earth's atmosphere\cite{PhysRevLett.81.1562}. In 2002, the SNO experiment provided the first direct evidence that this flavor conversion was also happening to electron neutrinos from the Sun\cite{PhysRevLett.89.011301}. The existence of neutrinos oscillations has profound implications, because it means that neutrinos cannot be massless.

Neutrino oscillations can be explained by postulating that the flavor eigenstates with which the Weak force interacts are mixtures of different mass eigenstates. The mass eigenstates can be described as wave packets with slightly different frequencies that overlap. These wave packets travel at different speeds due to their mass differences and therefore interfere with one another, leading to neutrino oscillations. However, the SM provides no explanation of how neutrinos acquire their masses. The Higgs mechanism requires couplings of both left-handed and right-handed chiral fields to the Higgs field, but there are no right-handed neutrinos in the SM. The extreme lightness of the neutrinos compared to other particles suggests that the process that produces them might be different altogether from the process that generates the masses of all other particles. Neutrino oscillations, therefore, are direct evidence of physics beyond the Standard Model (BSM) and motivate the search for new particles and forces that might be involved in the process of neutrino mass genereration.

This work describes a neutrino oscillation measurement using the IceCube Neutrino Observatory, a neutrino detector located at the geographic South Pole. It uses 5160 optical sensors deployed in a volume of one cubic kilometer of the Antarctic glacier to measure faint flashes of of Cherenkov light that is produced when neutrinos interact with the ice. It can detect neutrinos in a wide energy range starting from atmospheric neutrinos at a few GeV up to the PeV energy range of cosmic neutrinos. The data analysis presented in this work uses observations of the DeepCore sub-array of IceCube that is specifically optimized for the detection of neutrinos that are produced in the atmosphere of the Earth. These neutrinos consist mostly of muon neutrinos that travel through the Earth and have a chance to oscillate into other neutrino flavors before they are detected at the South Pole. After collecting tens of thousands of neutrinos over the course of several years, the muon neutrino survival probability can be estimated to a high precision. The results allow inferences about the mass differences between different neutrino states and can be probed for signs of BSM physics.

\Cref{ch:stdmodel} of this thesis summarizes how neutrinos are described in the Standard Model. A particular emphasis of this chapter is the famous Higgs mechanism by which all particles in the SM acquire their mass and how this mechanism can be expanded to include neutrino masses. In \refch{massoscillations}, we will describe how the mass differences between neutrinos lead to neutrino oscillations and how these oscillations manifest in experimental observations. The chapter also describes the experimental anomalies of neutrino oscillation measurements that motivate the search for additional heavy neutrino states of the eV scale. The IceCube Neutrino Observatory is introduced in \refch{icecube} with special focus on the DeepCore sub-array. The chapter will also describe in detail how the DeepCore data is filtered to produce a data sample with a high purity of muon neutrinos. The data sample is then used for two different measurements: The first, described in \refch{measurement-three-flavor}, measures the parameters that characterize the muon neutrino survival probability in the picture of three oscillating neutrino flavors. The second measurement, shown in \refch{measurement-sterile}, is probing the observed oscillation pattern for signs of an additional heavier neutrino mass eigenstate that is associated with an otherwise non-interacting \emph{sterile} neutrino. The result of this measurement places limits on the amount of mixing that is allowed between the hypothetical sterile neutrino state and the three known active neutrino flavors. With these constraints, this work provides another puzzle piece in the search for the origin of neutrino masses.

