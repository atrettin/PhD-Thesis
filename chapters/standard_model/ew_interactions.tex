%\section{Neutrino Properties}

%Neutrinos in the Standard Model are spin-$\nicefrac{1}{2}$ particles and interact solely via the Weak force, which is mediated by the exchange of $W^{\pm}$ and $Z^0$ bosons with masses of 80.4 and 91.2~GeV, respectively. Due to the large mass of the exchange bosons, the Weak force can only act on extremely short distances and interactions are generally much more rare than for the Strong and electromagnetic forces. For this reason, neutrinos can pass undisturbed over cosmological distances and penetrate enormous amounts of matter, which makes them excellent messenger particles for astronomy. On the other hand, this also means that they are very difficult to detect. Indeed, when their existence was first proposed by Pauli in the 1930s to explain the  continuous energy spectrum of electrons produced in  radioactive beta decays, it was thought that they might be entirely unobservable. It was not until 1956 that the first neutrino was detected, by Frederick Reines and Clyde Cowan, using a nuclear reactor as a source.
%%Since then, neutrinos have been detected from a variety of sources, including the Sun, the atmosphere, nuclear reactors, and accelerators.
%
%\subsection{Quantum Numbers and Helicity}
%
%Neutrinos are electrically neutral and have a lepton number that is defined empirically and results in the three known neutrino flavors $\nu_e$, $\nu_\mu$ and $\nu_e$. The neutrino fields of each flavor form a doublet with the charged lepton of the same lepton number, that is, the electron, the muon and the tauon, and all Weak interactions conserve the lepton number.

\section{Neutrino Interactions}
\label{sec:ew-interactions}

\subsection{Weak interactions after symmetry-breaking}
Neutrino interactions with matter are described by Weak force interactions after electroweak symmetry-breaking described in \refsec{ew-symmetry-breaking}.
The Lagrangian for these interactions can be written as the sum of the neutral-current (NC) and charged-current (CC) interactions. The NC part describes the exchange of neutral $Z^0$ bosons, which couples to all quarks and leptons except for right-handed neutrinos. For leptons, the NC Lagrangian reads
\begin{marginfigure}[*-4]
\centering
\begin{subfigure}[t]{0.49\linewidth}
    \begin{tikzpicture}
        \begin{feynman}[small]
        \vertex (a) {\(\pbar{\nu}_\mathcal{l}\)};
        \vertex [below right=of a.south] (center);
        \vertex [above right=of center.south] (b) {\(\pbar{\nu}_\mathcal{l}\)};
        \vertex [below=of center] (c) {\(Z^0\)};

        \diagram* {
          (a) -- [fermion] (center) -- [fermion] (b),
          (center) -- [boson] (c)
        };
        \end{feynman}
    \end{tikzpicture}
    \caption{neutrinos}
\end{subfigure}
\begin{subfigure}[t]{0.49\linewidth}
    \begin{tikzpicture}
        \begin{feynman}[small]
        \vertex (a) {\(\mathcal{l}^\pm\)};
        \vertex [below right=of a.south] (center);
        \vertex [above right=of center.south] (b) {\(\mathcal{l}^\pm\)};
        \vertex [below=of center] (c) {\(Z^0\)};

        \diagram* {
          (a) -- [fermion] (center) -- [fermion] (b),
          (center) -- [boson] (c)
        };
        \end{feynman}
    \end{tikzpicture}
    \caption{charged leptons}
\end{subfigure}
\caption{Neutral-current lepton interaction vertices.}
\label{fig:nc-vertices}
\end{marginfigure}
\begin{equation}
  \mathcal{L}_\mathrm{NC,L} = -\frac{g}{2 c_W^2} \sum_{\mathcal{l}=e,\mu,\tau} (\bar{\nu}_{\mathcal{l}, L} \gamma^\mu \nu_{\mathcal{l}, L} + (2 s_W^2 - 1) \bar{e}_{\mathcal{l}, L} \gamma^\mu e_{\mathcal{l}, L} + 2s_w^2 \bar{e}_{\mathcal{l}, R} \gamma^\mu e_{\mathcal{l}, R}) Z^0_\mu\;, \label{eq:ew-nc-lagrangian}
\end{equation}
where $\nu$ denotes a neutrino field, $e$ a lepton field, and the subscripts $L$ and $R$ denote left-handed and right-handed fields, respectively. Since the Lagrangian is written in the flavor basis, the neutrino fields are superpositions of mass eigenstates. The coefficient $s_W$ ($c_W$) is the sine (cosine) of the Weinberg angle and $g$ is the coupling constant that determines the overall strength of the electroweak force. This Lagrangian leads to the trilinear couplings shown in \reffig{nc-vertices}. The couplings to quarks have the same form as those to the charged leptons up to a difference in coupling strength\sidenote{Quark mixing has no effect on neutral current interactions due to the GIM mechanism.}. Neutral-current interactions conserve both the electric charge and lepton number, such that a neutral-current interaction of a neutrino will always produce a neutrino of the same flavor.

The charged-current (CC) part of the Weak Lagrangian in the flavor basis is
\begin{marginfigure}
\centering
\begin{subfigure}[t]{0.49\linewidth}
    \begin{tikzpicture}
        \begin{feynman}[small]
        \vertex (a) {\(\bar{\nu}_\mathcal{l}\)};
        \vertex [below right=of a.south] (center);
        \vertex [above right=of center.south] (b) {\(\mathcal{l}^{+}\)};
        \vertex [below=of center] (c) {\(W^-\)};

        \diagram* {
          (a) -- [fermion] (center) -- [fermion] (b),
          (center) -- [boson] (c)
        };
        \end{feynman}
    \end{tikzpicture}
    \caption{$W^-$ vertex}
\end{subfigure}
\begin{subfigure}[t]{0.49\linewidth}
    \begin{tikzpicture}
        \begin{feynman}[small]
        \vertex (a) {\(\nu_\mathcal{l}\)};
        \vertex [below right=of a.south] (center);
        \vertex [above right=of center.south] (b) {\(\mathcal{l}^{-}\)};
        \vertex [below=of center] (c) {\(W^+\)};

        \diagram* {
          (a) -- [fermion] (center) -- [fermion] (b),
          (center) -- [boson] (c)
        };
        \end{feynman}
    \end{tikzpicture}
    \caption{$W^+$ vertex}
\end{subfigure}
\caption{Charged-current lepton interaction vertices.}
\label{fig:cc-vertices}
\end{marginfigure}
\begin{equation}
    \mathcal{L}_\mathrm{CC} = -\frac{g}{\sqrt{2}} \sum_{\mathcal{l}=e,\mu,\tau} \bar{\nu}_{\mathcal{l},L} \gamma^\mu e_{\mathcal{l},L} W^+_\mu + \bar{e}_{\mathcal{l},L} \gamma^\mu \nu_{\mathcal{l},L} W^-_\mu\;\mathrm{+h.c.}\;.\label{eq:ew-cc-lagrangian}
\end{equation}
In contrast to neutral current interactions, the charged current interactions couple exclusively to left-handed fields\sidenote{The left-handed fields in the flavor basis are superpositions of mass eigenstates that may contain a (charge-conjugated) right-handed Majorana component as described in \refsec{neutrino-masses}}. The associated lepton interaction vertices are shown in \reffig{cc-vertices}.

The weak CC interactions with quarks are affected by quark mixing as a result of their mass generation via the Higgs mechanism. After electroweak symmetry breaking, the mass eigenstates and the flavor eigenstates of quarks are not identical but are instead mixed with a unitary matrix, $V$, that is also called the Cabbibo-Kobayashi-Maskawa (CKM) matrix. In the basis of mass eigenstates, the Lagrangian for weak CC interactions with quarks is
\begin{equation}
    \mathcal{L}_\mathrm{CC,Q} = \frac{g}{\sqrt{2}} \sum_{\alpha=1}^3 \sum_{\beta=1}^3 
    \bar{u}_{\alpha,L}\gamma^\mu V_{\alpha \beta} d_{\beta,L} W_\mu^+
     + \bar{d}_{\alpha,L}\gamma^\mu V_{\beta \alpha}^* u_{\beta,L} W_\mu^-
    \;\mathrm{+h.c.}\;,
\end{equation}
where the indices $\alpha$ and $\beta$ run over the generations. 
\subsection{Neutrino Interactions with Nuclei}
\label{sec:neutrino-xsec}


