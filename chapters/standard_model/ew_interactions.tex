\section{Neutrino Interactions}
\label{sec:ew-interactions}

Left-handed neutrinos and right-handed antineutrinos can interact with quarks and leptons via the exchange of $Z^0$ (neutral-current) and $W^\pm$ (charged-current) bosons. In practice, neutrinos are observed to either scatter off of electrons in reactions such as 
\begin{equation}
    \nu_e + e^- \rightarrow \nu_e + e^-\;,
\end{equation}
or to interact with nucleons. While the calculation of electron scattering is straight-forward from the electroweak Lagrangian, the scattering off nuclei is rather complicated and requires different approximations depending on the energy scale. This chapter first describes the scattering processes that are most important for the purpose of neutrino detection at energies of \SI{>1}{\giga\eV}, where the total cross-section is dominated by interactions with nuclei. It then briefly summarizes the process of coherent forward scattering that influences neutrino oscillations during propagation through bulk material.

\subsection{Weak interactions after symmetry-breaking}
Neutrino interactions with matter are described by Weak force interactions after electroweak symmetry-breaking described in \refsec{ew-symmetry-breaking}.
The Lagrangian for these interactions can be written as the sum of the neutral-current (NC) and charged-current (CC) interactions. The NC part describes the exchange of neutral $Z^0$ bosons, which couples to all quarks and leptons except for right-handed neutrinos. For leptons, the NC Lagrangian reads
\begin{marginfigure}
\centering
\begin{subfigure}[t]{0.49\linewidth}
    \begin{tikzpicture}
        \begin{feynman}[small]
        \vertex (a) {\(\pbar{\nu}_\mathcal{l}\)};
        \vertex [below right=of a.south] (center);
        \vertex [above right=of center.south] (b) {\(\pbar{\nu}_\mathcal{l}\)};
        \vertex [below=of center] (c) {\(Z^0\)};

        \diagram* {
          (a) -- [fermion] (center) -- [fermion] (b),
          (center) -- [boson] (c)
        };
        \end{feynman}
    \end{tikzpicture}
    \caption{neutrinos}
\end{subfigure}
\begin{subfigure}[t]{0.49\linewidth}
    \begin{tikzpicture}
        \begin{feynman}[small]
        \vertex (a) {\(\mathcal{l}^\pm\)};
        \vertex [below right=of a.south] (center);
        \vertex [above right=of center.south] (b) {\(\mathcal{l}^\pm\)};
        \vertex [below=of center] (c) {\(Z^0\)};

        \diagram* {
          (a) -- [fermion] (center) -- [fermion] (b),
          (center) -- [boson] (c)
        };
        \end{feynman}
    \end{tikzpicture}
    \caption{charged leptons}
\end{subfigure}
\caption{Neutral-current lepton interaction vertices.}
\label{fig:nc-vertices}
\end{marginfigure}
\begin{equation}
  \mathcal{L}_\mathrm{NC,L} = -\frac{g}{2 c_W^2} \sum_{\mathcal{l}=e,\mu,\tau} (\bar{\nu}_{\mathcal{l}, L} \gamma^\mu \nu_{\mathcal{l}, L} + (2 s_W^2 - 1) \bar{e}_{\mathcal{l}, L} \gamma^\mu e_{\mathcal{l}, L} + 2s_w^2 \bar{e}_{\mathcal{l}, R} \gamma^\mu e_{\mathcal{l}, R}) Z^0_\mu\;, \label{eq:ew-nc-lagrangian}
\end{equation}
where $\nu$ denotes a neutrino field, $e$ a lepton field, and the subscripts $L$ and $R$ denote left-handed and right-handed fields, respectively. Since the Lagrangian is written in the flavor basis, the neutrino fields are superpositions of mass eigenstates. The coefficient $s_W$ ($c_W$) is the sine (cosine) of the Weinberg angle and $g$ is the coupling constant that determines the overall strength of the electroweak force. This Lagrangian leads to the trilinear couplings shown in \reffig{nc-vertices}. The couplings to quarks have the same form as those to the charged leptons up to a difference in coupling strength\sidenote{Quark mixing has no effect on neutral current interactions due to the GIM mechanism.}. Neutral-current interactions conserve both the electric charge and lepton number, such that a neutral-current interaction of a neutrino will always produce a neutrino of the same flavor.

The charged-current (CC) part of the Weak Lagrangian in the flavor basis is
\begin{marginfigure}
\centering
\begin{subfigure}[t]{0.49\linewidth}
    \begin{tikzpicture}
        \begin{feynman}[small]
        \vertex (a) {\(\bar{\nu}_\mathcal{l}\)};
        \vertex [below right=of a.south] (center);
        \vertex [above right=of center.south] (b) {\(\mathcal{l}^{+}\)};
        \vertex [below=of center] (c) {\(W^-\)};

        \diagram* {
          (a) -- [fermion] (center) -- [fermion] (b),
          (center) -- [boson] (c)
        };
        \end{feynman}
    \end{tikzpicture}
    \caption{$W^-$ vertex}
\end{subfigure}
\begin{subfigure}[t]{0.49\linewidth}
    \begin{tikzpicture}
        \begin{feynman}[small]
        \vertex (a) {\(\nu_\mathcal{l}\)};
        \vertex [below right=of a.south] (center);
        \vertex [above right=of center.south] (b) {\(\mathcal{l}^{-}\)};
        \vertex [below=of center] (c) {\(W^+\)};

        \diagram* {
          (a) -- [fermion] (center) -- [fermion] (b),
          (center) -- [boson] (c)
        };
        \end{feynman}
    \end{tikzpicture}
    \caption{$W^+$ vertex}
\end{subfigure}
\caption{Charged-current lepton interaction vertices.}
\label{fig:cc-vertices}
\end{marginfigure}
\begin{equation}
    \mathcal{L}_\mathrm{CC} = -\frac{g}{\sqrt{2}} \sum_{\mathcal{l}=e,\mu,\tau} \bar{\nu}_{\mathcal{l},L} \gamma^\mu e_{\mathcal{l},L} W^+_\mu + \bar{e}_{\mathcal{l},L} \gamma^\mu \nu_{\mathcal{l},L} W^-_\mu\;\mathrm{+h.c.}\;.\label{eq:ew-cc-lagrangian}
\end{equation}
In contrast to neutral current interactions, the charged current interactions couple exclusively to left-handed fields\sidenote{The left-handed fields in the flavor basis are superpositions of mass eigenstates that may contain a (charge-conjugated) right-handed Majorana component as described in \refsec{neutrino-masses}}. The associated lepton interaction vertices are shown in \reffig{cc-vertices}.

The weak CC interactions with quarks are affected by quark mixing as a result of their mass generation via the Higgs mechanism. After electroweak symmetry breaking, the mass eigenstates and the flavor eigenstates of quarks are not identical but are instead mixed with a unitary matrix, $V$, that is also called the Cabbibo-Kobayashi-Maskawa (CKM) matrix. In the basis of mass eigenstates, the Lagrangian for weak CC interactions with quarks is
\begin{equation}
    \mathcal{L}_\mathrm{CC,Q} = \frac{g}{\sqrt{2}} \sum_{\alpha=1}^3 \sum_{\beta=1}^3 
    \bar{u}_{\alpha,L}\gamma^\mu V_{\alpha \beta} d_{\beta,L} W_\mu^+
     + \bar{d}_{\alpha,L}\gamma^\mu V_{\beta \alpha}^* u_{\beta,L} W_\mu^-
    \;\mathrm{+h.c.}\;,
\end{equation}
where the indices $\alpha$ and $\beta$ run over the generations. 

\subsection{Neutrino Interactions with Nuclei}
\label{sec:neutrino-xsec}

At energies of \SI{>= 1}{\giga\eV}, the total cross-section of neutrinos is dominated by interactions with nuclei, while scattering off electrons can be effectively neglected. There are three processes that each have different characteristic energy ranges. The descriptions of these processes and their cross-sections follow those in \sidecite{Formaggio:2012aa}.

\subsubsection{Charged-current quasi-elastic scattering}
At energies below \SI{1}{GeV}, neutrinos do not resolve the inner structure of a nucleon and the scattering process can be described as an interaction with the nucleon as a whole as
\begin{equation}
\begin{aligned}
    \nu_\ell + n &\rightarrow p + \ell^-\;,\\
    \bar{\nu}_\ell + p &\rightarrow n + \ell^+\;.
\end{aligned}
\end{equation}
The differential cross-section for this process as a function of the neutrino energy $E_\nu$ is 
\begin{equation}
    \frac{\drm\sigma}{\drm Q^2} = \frac{G_F^2 M^2 |V_{ud}|^2}{8\pi E^2_\nu}
    \left[
        A \pm \frac{s-u}{M^2}B + \frac{(s-u)^2}{M^4}C 
    \right]\;,\label{eq:ccqe-xsec}
\end{equation}
in which $V$ is the CKM matrix, $G_F$ is the Fermi constant, $Q^2$ is the squared four-momentum transfer and $M$ is the nucleon mass. The $\pm$ sign is positive for neutrinos and negative for antineutrinos. The variables $s$ and $u$ are Mandelstam variables that are functions of the momentum transfer and the factors $A$, $B$ and $C$ are functions of the form factors of the nucleon. In practice, these factors depend largely only on the vector ($F_1$ and $F_2$) and axial-vector ($F_A$) form factor, the latter of which is 
\begin{equation}
    F_A(Q^2) = \frac{g_A}{\left(1 + \frac{Q^2}{M_A^2}\right)^2}\;,\label{eq:axial-mass-form-factor}
\end{equation}
where $M_A$ is the \emph{axial mass}. With the vector form factors and the coupling constant $g_A$ are well constrained from electron scattering and nuclear beta decay, respectively, the only major source of uncertainty on this cross-section comes from that on the axial mass. 
\subsubsection{Neutral-current elastic scattering}
Cross-sections of neutral-current interactions of the form 
\begin{equation}
\begin{aligned}
    \nu p &\rightarrow \nu p \\
    \nu n &\rightarrow \nu n
\end{aligned}
\end{equation}
are described by an equation that has the same form as \refeq{ccqe-xsec}, albeit with different coupling constants, and their most important uncertainties can be parametrized with the same axial mass as in \refeq{axial-mass-form-factor}. For this reason, experiments have usually measured the ratio between CC and NC interactions.

\subsubsection{Resonant Scattering}
