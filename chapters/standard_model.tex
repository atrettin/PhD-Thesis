\setchapterpreamble[u]{\margintoc}
\setchapterstyle{kao}
% \setchapterimage[6.5cm]{figures/artwork/Alexander_Trettin_illustrations_of_wave_functions_and_particle__21cc39d2-0bda-4230-9b29-9bfd8707df01.png}
\chapter{Neutrinos in the Standard Model}

\labch{stdmodel}

The Standard Model (SM) of particle physics is a relativistic quantum field theory based on the gauge symmetry group $\mathrm{SU}(3)_C \times \mathrm{SU}(2)_L \times \mathrm{U}(1)_Y$, where the sub-scripts $C$, $L$ and $Y$ correspond to the conserved quantities \emph{color}, \emph{left-handed chirality} and \emph{weak hypercharge}, respectively. In this model, all matter particles are described as fermions, that is, excitations of Dirac-type fermion fields permeating space-time. The forces acting between fermions are mediated by an exchange of bosons, and all interactions must preserve the over-all symmetry of the theory. Since its completion in the early 1970s, it has been shown to an impressive degree of precision that it accurately describes the interactions between elementary particles due to the Strong Force, the Weak Force and the electromagnetic force. It can also explain how quarks and leptons acquire their masses via the Higgs mechanism, whose by-product, the Higgs boson, was detected at the LHC in 2012\todo{cite Higgs discovery}. Despite its success, the Standard Model has some known shortcomings such as its incompatibility with General Relativity and inability to explain cosmological phenomena most commonly interpreted as Dark Matter and Dark Energy.  Most relevantly for this work, it predicts that neutrinos should be massless and therefore does not allow for neutrino oscillations. Since neutrino oscillations can be experimentally observed at very high statistical significance\todo{cite nobel prize SuperK}, it is clear that the SM has to be extended in such a way that neutrino masses can be accommodated. There are several candidate theories for such an extension, but none of them could so far be confirmed experimentally. This chapter will first describe the properties and interactions of neutrinos as they are described by the SM. Then, it will show the mathematical formulation of neutrino oscillations, and finally it will describe some of the simplest extensions to the SM that could explain how neutrinos acquire their mass.

\section{Standard Model particles}

The elementary particles of the SM are organized into fermions and bosons, where fermions make up the observable matter while bosons are the particles that mediate forces. The number of force-mediating bosons is determined by the generators of the symmetry groups that all interactions must obey, while the strength of each force is determined by a \emph{coupling constant} that has to be estimated experimentally. There are eight massless gluons that correspond to the generators of the $\mathrm{SU}(3)_C$ group and mediate the Strong force. All Strong interactions conserve the so-called \emph{color} charge of the involved particles. The symmetry group $\mathrm{SU}(2)_L \times \mathrm{U}(1)_Y$ is the combined symmetry of the \emph{electroweak} force and produces the gauge boson fields $W_1$, $W_2$, $W_3$ and $B$. The electroweak symmetry group is broken into the $\mathrm{U}(1)_Q$ group by interactions of fermions with the Higgs field (further described below) that mixes the $W$ and $B$ fields into massive $W^\pm$ and $Z^0$ bosons and a massless photon $\gamma$ such that
\begin{align}
    Z &= \cos \theta_W W_3 - \sin \theta_W B \\
    \gamma &= \sin \theta_W W_3 + \cos \theta_W B\\
    W^\pm &= \frac{1}{\sqrt{2}} (W_1 \mp iW_2)\;,\label{eq:ew-boson-definitions}
\end{align}
where $\theta_W$ is the so-called \emph{Weinberg angle}.
\begin{margintable}
    \caption{Fermions in the Standard Model. The electric charge, Q, is the conserved charge of the $\mathrm{U}(1)_Q$ symmetry group.}
    \label{tab:fermions-sm}
    \centering
    \begin{tabular}{ccccc} \toprule
    & \multicolumn{3}{c}{generation} & \\ \cmidrule{2-4}
    & 1 & 2 & 3 & Q \\ \midrule
    \multirow{4}{*}{\rotatebox[origin=c]{90}{quarks}}\\
    & u & c & t & $+\nicefrac{2}{3}$ \\
    & d & s & b & $-\nicefrac{1}{3}$ \\
    \\ \midrule
    \multirow{4}{*}{\rotatebox[origin=c]{90}{leptons}}\\
    & $\nu_e$ & $\nu_\mu$ & $\nu_\tau$ & 0 \\
    & $e$ & $\mu$ & $\tau$ & $-1$ \\
    \\ \bottomrule
    \end{tabular}
\end{margintable}
The fermions of the SM are divided into quarks and leptons. Quarks participate in all strong, weak and electromagnetic interactions and are always found in combinations that form baryons (protons, neutrons) or mesons (kaons, pions). The leptons, on the other hand, do not participate in strong interactions. Charged leptons are massive and participate in both the weak and the electromagnetic force, while neutral leptons, the neutrinos, are massless and participate only in weak interactions. All fermions can be grouped into three \emph{generations} of quarks and leptons that are only distinguished by their masses, leading to a convenient arrangement of quarks and leptons into a $3\times4$ scheme as shown in \reftab{fermions-sm}. For each (massive) fermion, there exists a left-handed and a right-handed component. The left-handed components of each generation form a doublet of the $\mathrm{SU}(2)_L$ group with weak isospin $\frac{1}{2}$, while the right-handed components are singlets. The right-handed and left-handed fields for one generation and their charges are summarized in \reftab{fermions-one-generation}.
\begin{margintable}
    \caption{Eigenvalues of the weak isospin $I$, of its third component $I_3$ and the hypercharge $Y = 2(Q - I_3)$ for one generation of fermions. Reproduced from \cite{giunti-kim-neutrino}.}
    \label{tab:fermions-one-generation}
    \centering
    \begin{tabular}{cccc} \toprule
    & $I$ & $I_3$ & $Y$ \\ \midrule
    \multirow{2}{*}{$L_L \equiv \begin{pmatrix} \nu_{eL} \\ e_L \end{pmatrix}$} & \multirow{2}{*}{$\nicefrac{1}{2}$} & $\nicefrac{1}{2}$ & \multirow{2}{*}{-1}\\
    & & $-\nicefrac{1}{2}$ & \\ \midrule
    $e_R$ & 0 & 0 & -2 \\ \midrule
    \multirow{2}{*}{$Q_L \equiv \begin{pmatrix} u_L \\ d_L \end{pmatrix}$} & \multirow{2}{*}{$\nicefrac{1}{2}$} & $\nicefrac{1}{2}$ & \multirow{2}{*}{$\nicefrac{1}{3}$}\\
    & & $-\nicefrac{1}{2}$ & \\ \midrule
    $u_R$ & 0 & 0 & $\nicefrac{4}{3}$ \\
    $d_R$ & 0 & 0 & $-\nicefrac{2}{3}$ \\\bottomrule
    \end{tabular}
\end{margintable}

\subsection{Electroweak Symmetry Breaking}
\labsec{ew-symmetry-breaking}
The process of breaking the $\mathrm{SU}(2)_L \times \mathrm{U}(1)_Y$ symmetry group deserves special attention for the purposes of this work, because it is the process by which the exchange bosons of the Weak force acquire their mass. If the symmetry was unbroken, as it is the case for the $\mathrm{SU(3)}$ group of the Strong force, then the exchange bosons would all remain massless, just like the gluons. To simplify the discussion, the process can be illustrated using only the first generation of SM fermions. The starting point is to introduce the Higgs doublet of complex scalar fields
\begin{equation}
    \Phi = \begin{pmatrix}
        \Phi^+ \\
        \Phi^0
    \end{pmatrix}\;,\label{eq:higgs-doublet}
\end{equation}
where $\Phi^+$ is charged and $\Phi^0$ is neutral\sidenote{In a more general discussion, the Higgs doublet would be written down without assigning the charges a priori, they would be derived later. See \cite{schwartz_2013} for a more rigorous derivation.}. The Lagrangian describing the dynamics of this field,
\begin{equation}
    \mathcal{L}_\mathrm{Higgs} = (D_\mu \Phi^\dag)(D^\mu \Phi) - \lambda \left( \Phi^\dag \Phi - \frac{v^2}{2} \right)^2\;,\label{eq:higgs-lagrangian}
\end{equation}
with the covariant derivative
\begin{equation}
    D_\mu \Phi= \partial_\mu \Phi - ig W_\mu^a \tau^a \Phi - \frac{1}{2}ig'B_\mu \Phi
\end{equation}
is invariant under $\mathrm{SU}(2)_L \times \mathrm{U}(1)_Y$ symmetry and adds a quartic self-interaction potential with the parameters $\lambda$ and $v$, where $\lambda$ is taken to be positive, such that the potential is bounded from below. The fields $W_\mu^a$ in the covariant derivative correspond to the gauge bosons of the $\mathrm{SU}(2)_L$ group whose generators are $\tau^a = \frac{\sigma^a}{2}$, where $\sigma^a$ are the Pauli matrices. The $B_\mu$ field is the boson of the $\mathrm{U}(1)_Y$ group. The factors $g$ and $g'$ are the $\mathrm{SU}(2)_L$ and $\mathrm{U}(1)_Y$ coupling constants, respectively, and are related to the Weinberg angle by
\begin{equation}
    \tan \theta_W = \frac{g'}{g}\;.\label{eq:weinberg-angle}
\end{equation}
Because the potential has a minimum that is not at zero, the field $\Phi$ acquires a non-zero \emph{vacuum expectation value} (VEV) where $\Phi^\dag \Phi = \frac{v^2}{2}$. Since the vacuum is electrically neutral, this VEV can only come from the neutral part, $\Phi^0$, of the doublet and can be written as
\begin{equation}
    \Phi_\mathrm{VEV} = \frac{1}{\sqrt{2}}\begin{pmatrix}
        0\\
        v
    \end{pmatrix}\;.
\end{equation}
This vacuum expectation value is no longer symmetric under the $\mathrm{SU}(2)_L \times \mathrm{U}(1)_Y$ group, but it still is symmetric under the $\mathrm{U}(1)_Q$ group in which electric charge is conserved. To see what happens to the Lagrangian, the field $\Phi$ can be expressed in the unitary gauge as a variation around the VEV such that
\begin{equation}
    \Phi(x) = \begin{pmatrix}
        0 \\
        v + H(x)
    \end{pmatrix}\;.
\end{equation}
Plugging this into the Lagrangian in \refeq{higgs-lagrangian} and re-writing the $W_\mu^i$ and $B_\mu$ fields in terms of $Z$ and $W^\pm$ using the relationships given in \refeq{ew-boson-definitions} and \refeq{weinberg-angle} we find
\begin{align}
    \mathcal{L}_\mathrm{Higgs} = &\hspace{1em}\frac{1}{2}(\partial H)^2 - \lambda v^2 H^2 - \lambda v H^3 - \frac{\lambda}{4}H^4 \\
    &+ \frac{g^2v^2}{4} W_\mu^\dag W^\mu + \frac{g^2 v^2}{8\cos^2\theta_W}Z_\mu Z^\mu \label{eq:boson-mass-terms}\\
    &+ \mathrm{Higgs\;vertices}\;,
\end{align}
where Higgs vertices are 3-vertices and 4-vertices between the Higgs field and the $W$ and $Z$. The notable part is that the $W$ and $Z$ bosons have acquired a kinetic term in \refeq{boson-mass-terms} with a mass that is proportional to the VEV of the Higgs field, giving massive exchange bosons to the Weak force\sidenote{The massless photon field is found by expanding the full electroweak Lagrangian in the same way, which we neglect here for the sake of brevity.}.

\subsection{Charged Fermion Masses}
\label{sec:charged-fermion-masses}
In Quantum Electrodynamics, a Lorentz\todo{check spelling}-invariant mass term for spin-$\frac{1}{2}$ fermions can be written as a product of left-handed and right-handed Weyl spinors, also known as the Dirac mass
\begin{equation}
    \mathcal{L}_\mathrm{Dirac} = m (\bar{\Psi}_R \Psi_L - \bar{\Psi}_L \Psi_R)\;.
\end{equation}
However, such a term is not invariant under $\mathrm{SU}(2)_L \times \mathrm{U}(1)_Y$ and therefore cannot be added to the SM Lagrangian directly. Fortunately, masses for fermions can be recovered if we add a Yukawa coupling term between the fermions and the Higgs field, such as
\begin{equation}
    \mathcal{L}_\mathrm{Yuk} = -y \bar{L}_L \Phi e_R + \mathrm{h.c.}\;,
\end{equation}
where $L_L$ denotes the $\mathrm{SU}(2)_L$ doublet listed in \reftab{fermions-one-generation} and $y$ is the Yukawa coupling constant. When the VEV is inserted into this term, it produces a mass term $-m_e (\bar{e}_L e_R + \bar{e}_R e_L)$ with $m_e = \frac{y}{\sqrt{2}}v$ for the charged leptons and the down-type quarks $d$, $s$, and $b$.  A similar term that is also invariant under $\mathrm{SU}(2)_L$ and generates masses for the up-type quarks is $-y \bar{L}_L \tilde{\Phi} u_R$, where we defined $\tilde{\Phi} \equiv i \sigma_2 \Phi$. When all possible terms of this form for all generations of quarks are put together, the generations mix among each other in a way that is very similar to neutrino mixing, which is beyond the scope of this discussion.
%Putting all possible terms of this form for all generations of quarks together, the Lagrangian for quark masses is
%\begin{equation}
%   \mathcal{L}_\mathrm{mass} = -Y_{ij}^d \bar{Q}^i H d_R^j - Y_{ij}^u \bar{Q}^i \tilde{H} u_R^j + \mathrm{h.c.}\;,
%\end{equation}
%which contains two matrices of coupling constants, $Y^d$ and $Y^u$ with indices $i$ and $j$ enumerating the generations. After symmetry breaking, the mass terms become
%\begin{align}
%   \mathcal{L}_\mathrm{mass} &= -\frac{v}{\sqrt{2}}\left[  Y^d_{ij} \bar{d}_L^i d_R^j +  Y^u_{ij} \bar{u}_L^i u_R^j \right] + \mathrm{h.c.} \\
%   &=  -\frac{v}{\sqrt{2}}\left[ \bar{d}_L Y_d d_R + \bar{u}_L Y_u u_R \right] + \mathrm{h.c.}\;.
%\end{align}
%These terms are not the masses of the physically observed quarks, because the matrices $Y_{d,u}$ can in principle have off-diagonal terms. To get well-defined quark masses, therefore, the $Y_{d,u}$ matrices have to be diagonalized with a unitary transformation such that
%\begin{equation}
%    Y_{u,d} Y_{u,d}^\dag = U_{u,d} M_{u,d}^2 U_{u,d}^\dag\;,
%\end{equation}
%where the $M_{u,d}$ are diagonal mass matrices. This transformation introduces two different bases in which the Lagrangian can be written: In the \emph{mass basis}, the mass terms are diagonal

\subsection{Neutrino Masses}

The Higgs mechanism described in \refsec{charged-fermion-masses} necessitates both left-handed and right-handed Weyl spinors to interact with the Higgs field. Since there are no right-handed neutrinos in the SM, it predicts that they should be massless, in contradiction to experimental evidence. However, if we add right-handed neutrino fields into the model, then neutrino masses can be generated in a way that is tantalizingly similar to that of up-type quarks by adding interactions of the form $Y_{ij}^\nu \bar{L}^i \tilde{\Phi}\nu_R^j$ to the Lagrangian. Such a right-handed field would be uncharged with respect to all symmetry groups of the SM and would therefore not interact with any other particle and is hence referred to as a \emph{sterile neutrino}. Because neutrinos are electrically neutral, another possibility for a mass term that is allowed by the symmetry of the SM is the so-called \emph{Majorana mass}, $m \nu_R^c \nu_R$, in which $\nu_R^c=\nu_R^T \sigma_2$ is the charge conjugate Weyl spinor. The most general Lagrangian including all Yukawa couplings and Majorana mass terms of the lepton sector is
\begin{equation}
    \mathcal{L}_\mathrm{mass} = -Y_{ij}^e \bar{L}^i \Phi e_R^j - Y_{ij}^\nu \bar{L}^i \tilde{\Phi} \nu_R^j - iM_{ij}(\nu_R^i)^c \nu_R^j + \mathrm{h.c.}\;,
\end{equation}
where the indices $i$ and $j$ run over the generations $e$, $\mu$, and $\tau$ and the matrix $Y_{ij}^e$ contains the Yukawa coupling constants.

%\section{Neutrino Properties}

%Neutrinos in the Standard Model are spin-$\nicefrac{1}{2}$ particles and interact solely via the Weak force, which is mediated by the exchange of $W^{\pm}$ and $Z^0$ bosons with masses of 80.4 and 91.2~GeV, respectively. Due to the large mass of the exchange bosons, the Weak force can only act on extremely short distances and interactions are generally much more rare than for the Strong and electromagnetic forces. For this reason, neutrinos can pass undisturbed over cosmological distances and penetrate enormous amounts of matter, which makes them excellent messenger particles for astronomy. On the other hand, this also means that they are very difficult to detect. Indeed, when their existence was first proposed by Pauli in the 1930s to explain the  continuous energy spectrum of electrons produced in  radioactive beta decays, it was thought that they might be entirely unobservable. It was not until 1956 that the first neutrino was detected, by Frederick Reines and Clyde Cowan, using a nuclear reactor as a source.
%%Since then, neutrinos have been detected from a variety of sources, including the Sun, the atmosphere, nuclear reactors, and accelerators.
%
%\subsection{Quantum Numbers and Helicity}
%
%Neutrinos are electrically neutral and have a lepton number that is defined empirically and results in the three known neutrino flavors $\nu_e$, $\nu_\mu$ and $\nu_e$. The neutrino fields of each flavor form a doublet with the charged lepton of the same lepton number, that is, the electron, the muon and the tauon, and all Weak interactions conserve the lepton number.

\section{Weak Interactions After Symmetry-Breaking}
\label{sec:ew-interactions}

Neutrino interactions with matter are described by Weak force interactions after electroweak symmetry-breaking described in \refsec{ew-symmetry-breaking}.
The Lagrangian for these interactions can be written as the sum of the neutral-current (NC) and charged-current (CC) interactions. The NC part describes the exchange of neutral $Z^0$ bosons, which couples to all quarks and leptons except for right-handed neutrinos. For leptons, the NC Lagrangian reads
\begin{equation}
  \mathcal{L}_\mathrm{NC,L} = -\frac{g}{2 c_W^2} \sum_{\mathcal{l}=e,\mu,\tau} (\bar{\nu}_{\mathcal{l}, L} \gamma^\mu \nu_{\mathcal{l}, L} + (2 s_W^2 - 1) \bar{e}_{\mathcal{l}, L} \gamma^\mu e_{\mathcal{l}, L} + 2s_w^2 \bar{e}_{\mathcal{l}, R} \gamma^\mu e_{\mathcal{l}, R}) Z^0_\mu\;, \label{eq:ew-nc-lagrangian}
\end{equation}
where $\nu$ denotes a neutrino field, $e$ a lepton field, and the subscripts $L$ and $R$ denote left-handed and right-handed fields, respectively. Since the Lagrangian is written in the flavor basis, the neutrino fields are superpositions of mass eigenstates. The coefficient $s_W$ ($c_W$) is the sine (cosine) of the Weinberg angle and $g$ is the coupling constant that determines the overall strength of the electroweak force. This Lagrangian leads to the trilinear couplings shown in \reffig{nc-vertices}. The couplings to quarks have the same form as those to the charged leptons up to a difference in coupling strength.
\begin{figure}
\centering
\begin{subfigure}{0.3\linewidth}
    \begin{tikzpicture}
        \begin{feynman}[small]
        \vertex (a) {\(\pbar{\nu}_\mathcal{l}\)};
        \vertex [below right=of a.south] (center);
        \vertex [above right=of center.south] (b) {\(\pbar{\nu}_\mathcal{l}\)};
        \vertex [below=of center] (c) {\(Z^0\)};

        \diagram* {
          (a) -- [fermion] (center) -- [fermion] (b),
          (center) -- [boson] (c)
        };
        \end{feynman}
    \end{tikzpicture}
    \caption{neutrinos}
\end{subfigure}
\begin{subfigure}{0.3\linewidth}
    \begin{tikzpicture}
        \begin{feynman}[small]
        \vertex (a) {\(\mathcal{l}^\pm\)};
        \vertex [below right=of a.south] (center);
        \vertex [above right=of center.south] (b) {\(\mathcal{l}^\pm\)};
        \vertex [below=of center] (c) {\(Z^0\)};

        \diagram* {
          (a) -- [fermion] (center) -- [fermion] (b),
          (center) -- [boson] (c)
        };
        \end{feynman}
    \end{tikzpicture}
    \caption{charged leptons}
\end{subfigure}
\caption{Neutral-current lepton interaction vertices.}
\label{fig:nc-vertices}
\end{figure}
Neutral-current interactions conserve both the electric charge and lepton number, such that a neutral-current interaction of a neutrino will always produce a neutrino of the same flavor.

The charged-current (CC) part of the Weak Lagrangian in the flavor basis is
\begin{equation}
    \mathcal{L}_\mathrm{CC} = -\frac{g}{\sqrt{2}} \sum_{\mathcal{l}=e,\mu,\tau} \bar{\nu}_{\mathcal{l},L} \gamma^\mu e_{\mathcal{l},L} W^+_\mu + \bar{e}_{\mathcal{l},L} \gamma^\mu \nu_{\mathcal{l},L} W^-_\mu\;.\label{eq:ew-cc-lagrangian}
\end{equation}
In contrast to neutral current interactions, the charged current interactions couple exclusively to left-handed fields\sidenote{The left-handed fields in the flavor basis are superpositions of mass eigenstates that may contain a (charge-conjugated) right-handed Majorana component as described in \refsec{neutrino-masses}}. The associated lepton interaction vertices are shown in \reffig{cc-vertices}.
\begin{figure}
\centering
\begin{subfigure}{0.3\linewidth}
    \begin{tikzpicture}
        \begin{feynman}[small]
        \vertex (a) {\(\bar{\nu}_\mathcal{l}\)};
        \vertex [below right=of a.south] (center);
        \vertex [above right=of center.south] (b) {\(\mathcal{l}^{+}\)};
        \vertex [below=of center] (c) {\(W^-\)};

        \diagram* {
          (a) -- [fermion] (center) -- [fermion] (b),
          (center) -- [boson] (c)
        };
        \end{feynman}
    \end{tikzpicture}
    \caption{Coupling to $W^-$}
\end{subfigure}
\begin{subfigure}{0.3\linewidth}
    \begin{tikzpicture}
        \begin{feynman}[small]
        \vertex (a) {\(\nu_\mathcal{l}\)};
        \vertex [below right=of a.south] (center);
        \vertex [above right=of center.south] (b) {\(\mathcal{l}^{-}\)};
        \vertex [below=of center] (c) {\(W^+\)};

        \diagram* {
          (b) -- [fermion] (center) -- [fermion] (a),
          (center) -- [boson] (c)
        };
        \end{feynman}
    \end{tikzpicture}
    \caption{Coupling to $W^+$}
\end{subfigure}
\caption{Charged-current lepton interaction vertices.}
\label{fig:cc-vertices}
\end{figure}
The weak CC interactions with quarks couple up-type quarks to down-type quarks with the Lagrangian
\begin{equation}
    \mathcal{L}_\mathrm{CC,Q} = \frac{g}{\sqrt{2}}\bar{u}_L\gamma^\mu d_L W_\mu\;.
\end{equation}

\subsection{Neutrino Cross-Sections}
\label{sec:neutrino-xsec}


\section{Neutrino Sources}
\section{Neutrino Sources}

\subsection{Solar neutrinos}
\label{sec:solar-nu}

Here we describe how solar neutrinos come to be.
