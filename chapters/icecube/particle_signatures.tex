\section{Particle Signatures in IceCube}

All particle signatures in IceCube can be approximated as being combinations of compact \emph{cascades} that are produced by hadronic and electromagnetic showers (see section~\ref{sec:had-showers} and \ref{sec:em-showers}), and elongated \emph{tracks} that are only produced by muons travelling through the detector.

\subsection{Neutrinos}

At energies above 10~GeV, nearly all neutrino interactions are due to Deep Inelastic Scattering (DIS) and therefore always produce at least a hadronic cascade originating at the interaction vertex. In Neutral-Current (NC) interactions, this hadronic cascade is the only visible part of the interaction. Charged-current (CC) interactions also produce a lepton of the same flavor of the primary neutrino. For electron-neutrinos, this creates an electromagnetic (EM) shower that also originates at the interaction vertex. While the direction of the EM shower and the hadronic shower might not be exactly the same, they are in practice not distinguishable by the detector and can therefore be approximated as a single cascade-like signature. For muon-neutrinos, a muon is produced in the CC interaction which can then travel a significant distance through the ice beyond the extent of the initial hadronic shower, creating a track-like signature that sticks out of the cascade reminiscent of a lollipop. Charged-current interactions of tau-neutrinos produce a tauon that decays after a short distance, creating a second EM shower at the point of its decay. At TeV-scale energies, the distance covered by the tauon before its decay can be large enough to make the separation between the two showers resolvable, creating a \emph{double-bang} signature consisting of two cascades. At energies below 100~GeV that are more relevant to this work, however, the two cascades are too close together to be cleanly separable and they are effectively approximated as a single cascade as well. About 17\% of tauons produce a muon opon decay, which creates a short track-like signature as well.

\subsection{Atmospheric muons}

Atmospheric muons at energies below 100~GeV, where the dominant process of energy loss is ionization, create a single long track-like signature that can pass through the entire volume of the IceCube sensor array. At energies above 100~GeV, radiative energy losses become more relevant that create a series of stochastically distributed cascades along the muon's trajectory. The fraction of total energy loss that is concentrated in these cascades is referred to as \emph{stochasticity}. Atmospheric muons also often arrive in bundles originating from a single interaction of cosmic rays with the upper atmosphere. Within such a bundle, stochastic energy losses of individual muons may average out over its trajectory, such that the bundle as a whole can be approximated as a single long track with a relatively low stochasticity.
