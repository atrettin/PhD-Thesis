\setchapterpreamble[u]{\margintoc}
\chapter{Summary and Future Outlook}

\section{Summary of Results}
\label{sec:summary}

This work presented two oscillation measurements of atmospheric neutrinos using the DeepCore sub-array of the IceCube Neutrino Observatory. Both measurements were based on a newly developed data sample of \num{21914} well-reconstructed, track-like events in the energy range between \SI{5}{\giga\electronvolt} and \SI{150}{\giga\electronvolt}. The selection process for these events was described in \refch{data-sample} and consisted of several filtering steps that remove background due to detector noise and atmospheric muons. At the final filter level, the contribution of muon background was reduced to $\sim2\%$ and the noise background was entirely negligible and the sample consisted almost entirely of neutrino interactions. The zenith angle of every event was reconstructed using a geometric method and the energy of every interactions was estimated with a likelihood that only takes into account whether or not a sensor in the array has observed light. %While these reconstruction methods are less accurate than more sophisticated methods that were developed more recently, they are also less affected by the precise modelling of the detector properties. 
Several quantities produced by the reconstruction algorithms such as the length of the reconstructed track and the goodness-of-fit were fed into a Boosted Decision Tree (BDT) that calculated a particle ID (PID) number for every event that estimates the probability that it originated from a muon neutrino interaction. Both data and simulated pseudo-data were binned in zenith angle, energy, and PID. A minimization algorithm then calculated the best-fit neutrino oscillation parameters by re-weighting the simulated events to match the histograms of the observed data as closely as possible.

\subsection{Three-Flavor Oscillations}
\label{sec:summary-three-flavor}

The first data analysis shown in this work was a measurement of the atmospheric mixing angle and mass splitting in the three-flavor neutrino oscillation model assuming normal mass ordering. This measurement is complementary to oscillation analyses of accelerator neutrinos and the most precise measurement using atmospheric neutrinos to date. The result
\begin{align*}
    \sin^2\theta_{23} &= 0.507_{-0.053}^{+0.050}\\
    \Delta m^2_{32} &= 2.42_{-0.75}^{+0.77} \times10^{-3}\;\mathrm{eV}^2.
\end{align*}
is consistent with previous DeepCore measurements and current global fits.

\subsection{Sterile Neutrino Search}
\label{sec:summary-sterile-osc}


\chapter{Conclusion}
