\chapter{Abstract}

Neutrino oscillations are the only phenomenon beyond the Standard Model that has been confirmed experimentally to a very high statistical significance.
This work presents a measurement of atmospheric neutrino oscillations using eight years of data taken by the IceCube DeepCore detector between 2011 and 2019.
The event selection has been improved over that used in previous DeepCore measurements with a particular emphasis on its robustness with respect to systematic uncertainties in the detector properties.
The oscillation parameters are estimated via a maximum likelihood fit to binned data in the observed energy and zenith angle, where the expectation is derived from weighted simulated events.
This work discusses the simulation and data selection process, as well as the statistical methods employed to give an accurate expectation value under variable  detector properties and other systematic uncertainties.
The measurement is first performed first under the standard three-flavor oscillation model, where the atmospheric mass splitting and mixing angle are estimated to be $\Delta m^2_{32} = 2.42_{-0.75}^{+0.77} \times10^{-3}\;\mathrm{eV}^2$ and $\sin^2\theta_{23} = 0.507_{-0.053}^{+0.050}$, respectively.
The three-flavor model is then extended by an additional mass eigenstate corresponding to a sterile neutrino with mass splitting $\Delta m^2_{41} = 1\;\mathrm{eV}^2$ that can mix with the active $\nu_\mu$ and $\nu_\tau$ flavor states.
No significant signal of a sterile neutrino is observed and the mixing amplitudes between the sterile and active states are constrained to $|U_{\mu 4}|^2 < 0.0534$ and $|U_{\tau 4}|^2 < 0.0574$ at 90\% C.L.
These limits are more stringent than the previous DeepCore result by a factor between two and three and the constraint on $|U_{\tau 4}|^2$ is the strongest in the world.
